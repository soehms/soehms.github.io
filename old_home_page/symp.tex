\documentclass[twoside,12pt]{article}
%\usepackage{babel}
\usepackage{latexsym,amssymb,amsfonts}
\topmargin-1.1cm
%\topmargin-2.7cm
\parindent0cm
\headheight0mm
\textwidth16.0cm
\textheight24.1cm
\tabcolsep-4mm
\oddsidemargin-2mm
\evensidemargin-2mm
\pagestyle{myheadings}
\markboth{\centerline{Sebastian Oehms}}{\centerline{Symplectic q-Schur Algebras}}


\begin{document}

\newtheorem{prop}{Proposition}[section]
\newtheorem{thm}[prop]{Theorem}
\newtheorem{cor}[prop]{Corollary}
\newtheorem{lem}[prop]{Lemma}
\newtheorem{Def}[prop]{Definition}
\newtheorem{rem}[prop]{Remark}

\newenvironment{Pf}{{\sc Proof: }}{$\Box$\\}
\newenvironment{PfS}{{\sc Sketch of Proof: }}{$\Box$\\}



\newcommand{\N}{{\mathbb N}\,}
\newcommand{\F}{{\mathbb F}\,}
\newcommand{\Z}{{\mathbb Z}\,}
\newcommand{\C}{{\mathbb C}\,}
\newcommand{\K}{{\mathbb K}\,}
\newcommand{\Q}{{\mathbb Q}\,}
\newcommand{\X}{{\mathbb X}\,}
\newcommand{\Y}{{\mathbb Y}\,}
\newcommand{\I}{{\mathbb I}\,}




\newcommand{\Ab}{\\[-2mm]}
\newcommand{\pfr}{\rightarrow}
\newcommand{\pfl}{\leftarrow}
\newcommand{\gdw}{\Longleftrightarrow}
\newcommand{\ohne}{\backslash}

\newcommand{\Dim}[2]{{\rm dim}_{#1}{(#2)}}      
\newcommand{\Min}[1]{{\rm min}(#1)}      
\newcommand{\Char}[1]{{\rm char}(#1)}   
\newcommand{\Ker}[1]{{\rm ker}(#1)}   
\newcommand{\Ima}[1]{{\rm im}(#1)}   
\newcommand{\mg}[1]{\underline{#1}}           %Menge 1,... , #1 
\newcommand{\tmg}[2]{P({#1},{#2})}     %#2-elm Teilmengen von 1,... ,#1
\newcommand{\tlmg}[3]{P({#1},{#2},{#3})}  %#2-elm Teilmengen von #3,... ,#1



\newenvironment{tableau}[1]{\begin{array}{*{#1}{|p{2mm}}|}
}{\end{array}}
\newcommand{\tline}[3]{#1  &\multicolumn{#2}{c}{} \\ \cline{1-#3}}








% Kapitel 1



\newcommand{\kom}{\Delta}                     % Komultiplikation
\newcommand{\koe}{\epsilon}                   % Koeins
\newcommand{\cmd}[1]{\tau_{#1}}               % Komodulstrukturabb.
\newcommand{\mult}{\bigtriangledown}          % Multiplikation
\newcommand{\eins}{\mbox{i}}                  % Eins
\newcommand{\md}[1]{\eta_{#1}}                % Modulstrukturabb.
\newcommand{\ap}{\gamma}                      % Antipode


\newcommand{\id}[1]{{\rm id}_{#1}}            % Identit"at
\newcommand{\gr}{R}                           % Grundring
\newcommand{\er}{S}                           % Erweiterungsring
\newcommand{\nt}{V}                           % Natuerlicher Modul
\newcommand{\Ed}{{\cal E}}                    % Abbk. f"ur \grntend
\newcommand{\rEd}{{\cal E}_r}                 % Abbk. f"ur \grntrtsend
\newcommand{\gEd}[1]{{\cal E}_{#1}}           % wie \rEd f"ur bel. r
\newcommand{\dual}[1]{{#1}^*}                 % dualer Modul
\newcommand{\dEd}{\dual{\Ed}}                 % dualer Modul zu \grntend
\newcommand{\rdEd}{\dual{\rEd}}               % dualer Modul zu \grntrtsend
\newcommand{\gdEd}[1]{\dual{\gEd{#1}}}        % wie \rdEd f"ur bel. r


\newcommand{\bs}[1]{v_{#1}}                   % Basis von \nt
\newcommand{\es}[2]{e_{#1#2}}                 % Basis von \Ed

\newcommand{\ts}[2]{{#1}^{\otimes #2}}        % mehrfache Tensorprodukte
\newcommand{\ntts}[1]{\nt^{\otimes #1}}       % "         " mit nat. Modul
\newcommand{\ntrts}{\nt^{\otimes r}}          % r-fache   " "   nat Mod.

\newcommand{\End}[2]{{\rm End}_{#1}{(#2)}}          % Endomrphismenring
\newcommand{\Hom}[3]{{\rm Hom}_{#1}{(#2,#3)}}       % Hom-Menge
\newcommand{\tr}{tr}                          % Spurabb.
\newcommand{\tens}[1]{{\cal T}(#1)}           % Tensoralgebra


\newcommand{\mind}[2]{I(#1,#2)}               %  Multiindizes
\newcommand{\nmind}[1]{\mind{n}{#1}}          % Menge der n Multiindizes
\newcommand{\mmind}[1]{\mind{m}{#1}}          % Menge der m Multiindizes
\newcommand{\ibf}[1]{{\bf #1}}                % Multiindex
\newcommand{\komp}[2]{\Lambda(#1, #2)}        % Komposition
\newcommand{\skomp}[2]{\Lambda^{\rm s}(#1, #2)}     % sympl. Komposition
\newcommand{\nkomp}[1]{\komp{n}{#1}}          % Komposition in n Teile
\newcommand{\nrkomp}{\komp{n}{r}}             % Komposition r in n Teile
\newcommand{\prt}[2]{\Lambda^+(#1, #2)}       % Partition in Teile
\newcommand{\sprt}[2]{\Lambda^{{\rm s}+}(#1, #2)}   % sympl. Partition in Teile
\newcommand{\nprt}[1]{\prt{n}{#1}}            % Partition in n Teile
\newcommand{\nrprt}{\prt{n}{r}}               % Partition r in n Teile
\newcommand{\rprt}[1]{\Lambda^+(#1)}          % Partition


\newcommand{\frt}[1]{{\cal M}(#1)}            % FRT-Konstruktion zu
\newcommand{\gfrt}[2]{{\cal M}(#1)_{#2}}      % FRT-Konstr. homog. Summ.
\newcommand{\rfrt}[1]{{\cal M}(#1)_r}         % FRT-Konstr. r-homog. Summ.

\newcommand{\greg}[2]{{#2}\left[#1\right]}    % Regul"arer Funktionenring
\newcommand{\reg}[1]{\greg{#1}{\gr}}          % Reg. Funktionenring "uber  \gr

\newcommand{\gmhg}[2]{{\rm M}_{#1}(#2)}       % Matrixhalbgruppe Typ A
\newcommand{\gspm}[2]{{\rm SpM}_{#1}(#2)}     % Matrixmonoid Typ C
\newcommand{\mmsp}[2]{{\rm SpH}_{#1}(#2)}     % nichtinvertierbare sympl. Matr.
\newcommand{\gom}[2]{{\rm OM}_{#1}(#2)}       % Matrixmonoid Typ B, D
\newcommand{\qmhg}[2]{{\rm M}_{#1,q}(#2)}     % Quanten mhg
\newcommand{\qspm}[2]{\gspm{#1,q}{#2}}        % Quanten spm
\newcommand{\qom}[2]{\gom{#1,q}{#2}}          % Quanten om
\newcommand{\qmo}[2]{\gmo{#1,q}{#2}}          % Quanten om
\newcommand{\mhg}[1]{\gmhg{#1}{\gr}}          % Matrixhalbgruppe Typ A
\newcommand{\spm}[1]{\gspm{#1}{\gr}}          % Matrixmonoid Typ C
\newcommand{\om}[1]{\gom{#1}{\gr}}            % Matrixmonoid Typ B, D
\newcommand{\gln}[2]{{\rm GL}_{#1}(#2)}       % Generelle lineare Gruppe
\newcommand{\gsp}[2]{{\rm GSp}_{#1}(#2)}      % Generelle symplektische Gr
\newcommand{\go}[2]{{\rm GO}_{#1}(#2)}        % Generelle orthogonale Gr
\newcommand{\zgsp}[2]{\overline{\gsp{#1}{#2}}} % Matrixmon. C Zariski-Abs.
\newcommand{\zgo}[2]{\overline{\go{#1}{#2}}}   % Matrixmon. B,D Zariski-Abs.
\newcommand{\sln}[2]{{\rm SL}_{#1}(#2)}       % Spezielle lineare Gruppe
\newcommand{\spn}[2]{{\rm Sp}_{#1}(#2)}       % Symplektische Gruppe
\newcommand{\on}[2]{{\rm O}_{#1}(#2)}         % Orthogonale Gruppe

\newcommand{\symg}[1]{{\cal S}_{#1}}          % Symetrische Gruppe
\newcommand{\asymg}[1]{\gr symg{#1}}          % Symetrische Gruppen-Algebra

\newcommand{\zopf}[1]{{\cal Z}_{#1}}          % Zopf Gruppe
\newcommand{\azopf}[1]{\gr \zopf{#1}}         % Zopf Gruppen-Algebra
\newcommand{\zerz}[1]{\sigma_{#1}}            % Zopf Gruppen Erzeuger

\newcommand{\frei}[1]{{\cal F}_{\gr,#1}}      % Freie-Algebra
\newcommand{\heck}[2]{{\cal H}_{#1,#2}}       % Hecke-Algebra
\newcommand{\bmw}[2]{{\cal C}_{#1,#2}}        % BMW-Algebra
\newcommand{\br}[2]{{\cal B}_{#1,#2}}         % Brauer-Algebra
\newcommand{\falg}{{\cal A}}                  % Algebren-Familie
\newcommand{\gfalg}[1]{{\cal A}_{#1}}         % Algebren-Familie homogene Teile
\newcommand{\rfalg}{{\cal A}_r}               % Algebren-Fam. r-homogene Teile
\newcommand{\dar}{\rho}                       % Darstellung der Familie
\newcommand{\rdar}{\rho_r}                    % r-te Darstellung der Familie
\newcommand{\gdar}[1]{\rho_{#1}}              % #1-te Darstellung der Familie
\newcommand{\il}{s}                           % linke Inlusion
\newcommand{\ir}{t}                           % rechte Inlusion
\newcommand{\gil}[1]{s_{#1}}                  % #1-te linke Inlusion
\newcommand{\gir}[1]{t_{#1}}                  % #1-te rechte Inlusion
\newcommand{\kerz}[2]{\isod([#1,#2])}         % Erzeuger Komm. Koideal
\newcommand{\ma}{\frt{\falg}}
\newcommand{\rma}{\rfrt{\falg}}

\newcommand{\ZK}[1]{M(#1)}                    % Zentralisatorkoalgebra
\newcommand{\ZA}[1]{C(#1)}                    % Zentralisatoralgebra
\newcommand{\KK}[1]{K(#1)}                    % Kommutatorkoideal
\newcommand{\MA}{\ZK A}                       % Zentralisatorkoalgebra von A
\newcommand{\tfq}[1]{\overline{#1}}           % Torsionsfreier Quotient
\newcommand{\TMA}{\tfq{\MA}}                  % Torsionsfreier Quotient von \MA
\newcommand{\Tma}{\tfq{\ma}}                  % Torsionsfreier Quotient von \ma
\newcommand{\Trma}{\tfq{\rma}}                % Torsionsfreier Quotient \rma
\newcommand{\CA}{\ZA A}                       % Zentralisatoralgebra von A
\newcommand{\CCA}{\ZA{\CA}}                   % Doppelzentralisator von A
\newcommand{\KA}{\KK A}                       % Kommutatorkoideal von A
\newcommand{\comp}[1]{{#1}^{\bot}}            % duales Komplement
\newcommand{\ausw}[1]{{\rm Ev}_{#1}}          % Auswerteabbildung
\newcommand{\merw}[2]{{#1}^{#2}}              % Erweiterter Modul
\newcommand{\smerw}[1]{\merw{#1}{\er}}        % Erweiterter Modul mit \er
\newcommand{\zerw}[2]{{#1}_{#2}}              % Zentralisator nach Erweiterung
\newcommand{\szerw}[1]{\zerw{#1}{\er}}        % Zentr. nach Erweiterung mit \er
\newcommand{\ierw}[2]{\merw{#1}{{#2}^{\succ}}}
                                              % Eingebetteter erw. Modul
\newcommand{\iserw}[1]{\ierw{#1}{\er}}        % Eingebett. erw. Modul \er
\newcommand{\einb}{\iota}                     % Einbettungsabbildungen
\newcommand{\geinb}[1]{\einb_{#1}}            % Einbettungsabb. konkret
\newcommand{\seinb}[1]{\smerw{(\geinb{#1})}}  % induzierte konkr. Einbettung
\newcommand{\duerw}[1]{\psi_{#1}}             % Isom. zwischen W*S und WS*
\newcommand{\auerw}[1]{\psi'_{#1}}            % Isom. zwischen W**S und WS**

\newcommand{\gloc}[2]{{#1}_{#2}}              % Lokalisation an Primideal
\newcommand{\rloc}[1]{\gr_{#1}}               % Lokalis. an Primideal in \gr
\newcommand{\Spec}[1]{{\rm Spec}(#1)}         % Spektrum eines Rings.

\newcommand{\qan}[1]{[#1]_q}                  % q-Analog
\newcommand{\ean}[1]{[#1]_{\epsilon}}         % q-Analog f"ur Zahl epsilon
\newcommand{\yan}[1]{\{#1\}_{y}}              % y-Analog 
\newcommand{\Xan}[1]{\{#1\}_{X}}              % y-Analog 
\newcommand{\ybin}[2]{\{{#1\atop #2}\}_{y}}   % y-analoger Binomialkoeff 
\newcommand{\iyan}[1]{\{#1\}_{y^{-1}}}        % y-Analog f"ur y^{-1}




% Kapitel 2


\newcommand{\An}[1]{A_{#1}(n)}
\newcommand{\rAn}[2]{A_{#1}(n,#2)}
\newcommand{\sAn}[1]{A^{{\rm s}}_{#1}(n)}
\newcommand{\rsAn}[2]{A^{{\rm s}}_{#1}(n,#2)}
\newcommand{\qAn}[1]{A_{q,#1}(n)}
\newcommand{\qrAn}[2]{A_{#1,q}(n,#2)}
\newcommand{\qsAn}[1]{A^{{\rm s}}_{#1,q}(n)}
\newcommand{\qrsAn}[2]{A^{{\rm s}}_{#1,q}(n,#2)}
\newcommand{\qsBn}[1]{A^{{\rm s}}_{#1,q}(n)}
\newcommand{\qrsBn}[2]{ A^{{\rm s}}_{#1,q}(n,#2)}
\newcommand{\rsBn}[2]{A^{{\rm s}}_{#1}(n,#2)}
\newcommand{\qshAn}[1]{A^{{\rm sh}}_{#1,q}(n)}
\newcommand{\qrshAn}[2]{A^{{\rm sh}}_{#1,q}(n,#2)}
\newcommand{\sch}[2]{S_{#1}(n,#2)}            % Schur-Algebra
\newcommand{\ssch}[2]{S^s_{#1}(n,#2)}         % sympl. Schur-Algebra
\newcommand{\qsch}[2]{S_{#1,q}(n,#2)}        % q-Schur-Algebra
\newcommand{\qssch}[2]{S^{\rm s}_{#1,q}(n,#2)}     % sympl. q-Schur-Algebra


\newcommand{\spT}{J}
\newcommand{\spq}{d}

\newcommand{\BMW}{Bir\-man-Mu\-ra\-ka\-mi-Wenzl }
\newcommand{\ggr}{\Z[BMW]}
\newcommand{\pgr}{\Z[B]}                      % Grundring Brauer-Algebra plus
\newcommand{\pars}[1]{P_n^{\rm s}(#1)}        % sympl. BMW-Tripel
\newcommand{\paro}[1]{P_n^{\rm o}(#1)}        % orth.. BMW-Tripel
\newcommand{\agr}{{\cal Z}'}
\newcommand{\zgr}{\Z[H]}
\newcommand{\cgr}{{\cal Z}}
\newcommand{\ocgr}{\cgr'}
\newcommand{\gk}[2]{g_{#1|#2}}
\newcommand{\ek}[2]{e_{#1|#2}}


\newcommand{\spg}[1]{s_{#1}}          % Elementarspiegelung in \symg r
\newcommand{\Mla}{M^{\lambda}}        % Permutationsmoduln
\newcommand{\gewr}{({\ntrts})^{\lambda}}   % Gewichtsraum
\newcommand{\ysym}{\symg{\lambda}}    % Yang Untergruppe
\newcommand{\xsym}{\symg\mu }            % Yang Untergruppe
\newcommand{\dsym}[1]{{\cal D}_{#1}}  % Nebenklassen vertr. Yang Untergruppe
\newcommand{\dysym}{\dsym{\lambda}}   
\newcommand{\dxsym}{\dsym\mu }           
\newcommand{\ddysym}{\dsym{\lambda ,\mu }} % Doppelnebenklassen vertr.



\newcommand{\aqdar}{\rho_r}
\newcommand{\bqdar}{\rho^{\rm o}_r}
\newcommand{\cqdar}{\rho^{\rm s}_r}
\newcommand{\rcqdar}[1]{\rho^{\rm s}_{#1}}


\newcommand{\raqfalg}{\falg_r}           % Bild \aqdar
\newcommand{\rbqfalg}{\falg^{\rm o}_r}   % Bild \cqdar
\newcommand{\rcqfalg}{\falg^{\rm s}_r}   % Bild \bqdar 
\newcommand{\aqfalg}{\falg}              % zugehoerige Familien.
\newcommand{\bqfalg}{\falg^{\rm o}}
\newcommand{\cqfalg}{\falg^{\rm s}}










% Kapitel 3


%\newcommand{\naA}[2]{{\bigwedge}_{#1}(#2)}
%\newcommand{\nraA}[3]{{\bigwedge}_{#1}(#2,#3)}
%\newcommand{\aA}{\naA{\gr}{n}}
%\newcommand{\raA}[1]{\nraA{\gr}{n}{#1}}
\newcommand{\snaA}[2]{{\bigwedge}^{\rm s}_{#1}(#2)}
\newcommand{\snraA}[3]{{\bigwedge}^{\rm s}_{#1}(#2,#3)}
\newcommand{\saA}{\snaA{\gr}{n}}
\newcommand{\sraA}[1]{\snraA{\gr}{n}{#1}}
\newcommand{\qnaA}[2]{{\bigwedge}_{#1,q}(#2)}
\newcommand{\qnraA}[3]{{\bigwedge}_{#1,q}(#2,#3)}
\newcommand{\aA}{\qnaA{\gr}{n}}
\newcommand{\raA}[1]{\qnraA{\gr}{n}{#1}}
\newcommand{\qsnaA}[2]{{\bigwedge}^{\rm s}_{#1,q}(#2)}
\newcommand{\qsnraA}[3]{{\bigwedge}^{\rm s}_{#1,q}(#2,#3)}

\newcommand{\qsaA}{\qsnaA{\gr}{n}}

\newcommand{\qsraA}[1]{\qsnraA{\gr}{n}{#1}}




\newcommand{\vaA}[1]{\qnaA{#1}{n}}
\newcommand{\vgaA}[2]{\qnraA{#1}{n}{#2}}
\newcommand{\vnaA}[1]{{\qsnaA{#1}{n}}}
\newcommand{\vngaA}[2]{\qsnraA{#1}{n}{#2}}
\newcommand{\vtaA}[2]{\vaA{#1}[#2]}
%\newcommand{\aA}{\vaA{}}
\newcommand{\gaA}[1]{\vgaA{\gr}{#1}}
\newcommand{\taA}[1]{\vtaA{\gr}{#1}}
%\newcommand{\raA}{\gaA{r}}
\newcommand{\naA}{\vnaA{}}
\newcommand{\nraA}{\vngaA{}{r}}
\newcommand{\vsA}[1]{{\sf \large S}_{#1}}
\newcommand{\vgsA}[2]{\vsA{#1}(#2)}
\newcommand{\sA}{\vsA{}}
\newcommand{\rsA}{\vgsA{r}}
\newcommand{\frtsym}{A^{\rm s}(n)}
\newcommand{\kfrtsym}{\qsAn{\gr}{1}}
\newcommand{\gfrtsym}{A^{\rm sh}(n)}
\newcommand{\rgfrtsym}[1]{A^{\rm sh}(n,#1)}
\newcommand{\rfrtsym}[1]{A^{\rm s}(n,#1)}
\newcommand{\rkfrtsym}[1]{\qrsAn{\gr}{1}{#1}}
\newcommand{\shape}[1]{[#1]}                 %Diagramm von Partition
\newcommand{\lshape}{\shape{\lambda}}        %Diagramm von #1
\newcommand{\gtab}[1]{T^{#1}}                %Grund Tableau von #1
\newcommand{\stabz}[1]{Z(\gtab{#1})}         %Zeilenstabilisator
\newcommand{\stabs}[1]{S(\gtab{#1})}         %Spaltenstabilisator

% Tableau mengen

\newcommand{\gsmind}[2]{I_{#1}^{#2}}          %Standardtableau
\newcommand{\asmind}[1]{\gsmind{#1}{}} 
\newcommand{\smind}[1]{\gsmind{#1}{<}} 
\newcommand{\symind}[1]{\gsmind{#1}{\rm sym}}
\newcommand{\symmind}[1]{\gsmind{#1}{\rm mys}} 
\newcommand{\ssmind}[1]{\gsmind{#1}{\rm col}} 

\newcommand{\itab}[2]{\gtab{#1}_{\ibf{#2}}}  %Tableau zu Multiindex
\newcommand{\col}[2]{C_{#1}(#2)}             %Spalte eines Tableaus
\newcommand{\row}[2]{R_{#1}(#2)}             %Zeile eines Tableaus
\newcommand{\bidet}[3]{\gtab{#1}(#2:#3)}     %Bideterminante klassisch
\newcommand{\qbidet}[3]{\gtab{#1}_q(#2:#3)}  %Bideterminante quantum
\newcommand{\pbidet}[3]{t^{#1}(#2:#3)}       %PraeBideterminante klassisch
\newcommand{\pqbidet}[3]{t_q^{#1}(#2:#3)}    %PraeBideterminante quantum
\newcommand{\qdet}[2]{{\rm det}_q(#1, #2)}
\newcommand{\Det}[2]{{\rm det}(#1, #2)}
\newcommand{\libf}[3]{\ibf{#1}^{#2}_{#3}}    %Summmand Multi-Index bez. Komp. 

\newcommand{\xes}[2]{x_{#1 #2}}              %Monome in $\frtsym$
\newcommand{\yes}[2]{X_{#1 #2}}              % Unbestimmte
\newcommand{\B}{{\bf B}}                     %Basis von $\frtsym$
\newcommand{\BB}{{\bf C}}                    %Teil von Basis von $\frtsym$
\newcommand{\stdard}[1]{L(#1)}               % Standardmodul (weyl)
\newcommand{\strich}[1]{{#1}^{+}}
\newcommand{\kstrich}[1]{{#1}^{-}}
\newcommand{\ostrich}[1]{{#1}^{\circ}}
\newcommand{\sign}[1]{{\rm sign}(#1)}        %Signum einer Perm.

\newcommand{\rg}[1]{{\rm rg}(#1)}
\newcommand{\aeq}[1]{<#1>}
\newcommand{\spi}[1]{{#1}^\times }

%\newcommand{\K}{{\cal K}}                    % Halbkoaslgebra in straighening
\newcommand{\M}{{\cal N}}                    % Ordnungsmenge
\newcommand{\fM}[1]{[]#1[]}                  % Ordnungsfunktion 
\newcommand{\ltperm}[2]{\ll _{#1}^{#2}}         % lt-Permutation operator.  
\newcommand{\Uilt}[1]{U_{\ibf i,l#1}}        % Aufspann  alle lt permut















% Kapitel 4

\newcommand{\dbidet}[3]{D^{#1}_{#2,#3}}
\newcommand{\cbidet}[3]{C^{#1}_{#2,#3}}



%%% Local Variables: 
%%% mode: latex
%%% TeX-master: "doc"
%%% End: 


\thispagestyle{empty}

\begin{center} 
{\bf \huge Symplectic q-Schur Algebras}
 \\[3mm]
%preliminary version, september 1997, revised february 2000\\[3mm]
{\bf Sebastian Oehms, Mathematisches Institut B, Universit{\"a}t Stuttgart
\footnote{E-Mail: seba@mathematik.uni-stuttgart.de}}
\end{center}


\section{Introduction}

The general linear group $GL_n(K)$ operates on $\ntrts$ 
the $r$-fold tensor space of its 
natural module $V$. Its group algebra factored by the kernel of this operation
is called the {\em Schur algebra} and denoted $S(n,r)$. By place permutation 
the symmetric group $\symg r$ operates on $\ntrts$, too. Moreover, both
actions centralize each other. This fact is known as 
{\em Schur-Weyl-Duality}.\Ab

This situation admits a $q$-analogue which has been introduced
by {\em R.\ Dipper} and {\em G.\ James} in \cite{dj}. Here, instead of 
the symmetric group you have to take the {\em Iwahori-Hecke algebra} of type A.
Its centralizer is called the {\em $q$-Schur algebra}.
There are various generalizations of this theory for instance by
{\em Dipper}, {\em James} and {\em A.\ Mathas} 
\cite{dja} who replaced the {\em Iwahori-Hecke algebras} by 
{\em Ariki-Koike algebras} leading to so called 
{\em cyclotomic $q$-Schur algebras}. 
On the other hand
the original $q$-Schur algebra can be obtained (up to Morita
equivalence, cf.\ \cite{dj2}) using constructions 
from the theory of {\em quantum groups}
\cite{dd}. In this paper we will apply these constructions to obtain
$q$-Schur algebras which are related to the {\em symplectic groups}. We
will denote them by $\ssch{q} r$.
Setting
the deformation parameter $q=1$, we obtain classical symplectic
Schur algebras in the sense of {\em S.\ Donkin} \cite{donk3}. 
The main result in this paper is that the symplectic
$q$-Schur algebras are {\em cellular} in the sense of {\em J.\ Graham} and
{\em G.\ Lehrer} \cite{graham} and  {\em integrally quasi-hereditary}
as algebras over the ring of integer Laurent polynomials. Furthermore we show
that they are the centralizers of the {\em \BMW algebras} acting on
$\ntrts$.

In order to obtain the cellular basis we introduce
a quantum symplectic version
of {\em bideterminants}.
In \cite{ofrt} the author has presented a symplectic version of
the famous  {\em straightening formula} for bideterminants
in the classical case. Here, we will
develop the fundamental calculus for quantum symplectic bideterminants
and give a quantized version of that straightening
formula. This formula is powerful enough to imply almost all results
of the paper.
\Ab

The {\em standard modules} (or {\em cell
representations}) of $\ssch{q} r$
are indexed by pairs $(\lambda, l)$ consisting of
an integer $0 \leq l \leq \frac r2$ and a partition
$\lambda \in \prt m{r-2l}$ of $r-2l$ in not more than $m$-parts. Here
$n=2m$ is the dimension of the natural module of the symplectic group.
The part of the basis corresponding to $(\lambda , l)$  is labelled by pairs of
{\em $\lambda$-symplectic standard tableaux} in the sense of
{\em R.C.\ King} \cite{king}, or more precisely by a reversed version of
them.\Ab

The material of this paper is taken from my doctoral thesis \cite{doc} 
arranged in a completely reorganized form. Further it contains some
improvements. For example the second part of Theorem \ref{BMW centralizer} 
has been proved 
in \cite[4.1.2]{doc} under restrictions only and the proof of
Proposition  \ref{Rel nicht spiegelsymplektisch}
is more direct and shortened, here.\Ab








\section{Quantum Symplectic Monoids}
\label{symp monoids}

Let $\gr$ be a noetherian integral domain and $q\in \gr$ an invertible
element.   Let $\nt$ be a free
$\gr$-module. Fixe a basis $\{\bs 1, \ldots , \bs n\}$ and let
$\es ij$ denote the corresponding basis of matrix units for
$\Ed:=\End{\gr}{\nt}$. We define two endomorphisms $\beta$ and $\gamma$ on
$\nt \otimes \nt$ identifying $\End{\gr}{\nt\otimes\nt}$ with
$\Ed\otimes\Ed$ (we write simply $\otimes$ instead of $\otimes_{\gr}$ if no
ambiguity can arise):

\[
\beta:=\sum_{1\leq i\leq n} (q^2\es ii \otimes \es ii
+\es i{i'} \otimes
\es{i'}i) + q\sum_{1\leq i\neq j,j'\leq n} \es ij \otimes
\es ji + \] \[
+(q^2-1)\sum_{1\leq j < i\leq n} (\es ii \otimes \es jj
-q^{\rho_i -\rho_j}\epsilon_i\epsilon_j  \es i{j'} \otimes
\es {i'}j),
\]
and

\[
\gamma: =\sum_{1\leq i,j\leq n} q^{\rho_i -\rho_j}\epsilon_i\epsilon_j
\es i{j'} \otimes\es {i'}j,
\]

Here, the exponents are given by 
$(\rho_1, \ldots , \rho_n)=(m, m-1, \ldots, 1, -1,\ldots ,-(m-1),
-m)$, whereas $\epsilon_i:=sign(\rho_i)$. 
Further, $i':=n-i+1$ defines an involution on $\mg n:=\{1,\ldots ,
n\}$. Thus
\[ (1',2', \ldots , n')=(n,n-1, \ldots , 1). \]
There are slightly more general versions of these endomorphisms
involving additional parameters. We may omit them without loss
of generality (see \cite[Satz 2.5.8]{doc}).
For $r\in \N$ write $\mg r:=\{1, \ldots , r\}$. A {\em multi-index} is a 
map $\ibf i:\mg r\pfr \mg n$ frequently denoted as an $r$-tuple
$\ibf i=(i_1, \ldots , i_r)$ where $i_j\in \mg n$. The set of
all such multi-indices will be denoted by $\nmind r$. We define

\[ \bs{\ibf i}:=\bs{i_1}\otimes \bs{i_2} \otimes \ldots \otimes
\bs{i_r} \in \underbrace{\nt \otimes \nt \otimes \ldots \otimes \nt}_{
  r\mbox{-times}}=: \ntrts. \]

An endomorphism $\mu$ of $\ntrts$ may be given by its coefficients
$\mu_{\ibf i\ibf j}$ 
with respect to the basis
$\{ \bs{\ibf i}|\; \ibf i \in \nmind r\}$ of $\ntrts$, that is

\[ \mu (\bs{\ibf j}) = \sum_{\ibf i\in \nmind r}\mu_{\ibf i\ibf
  j}\bs{\ibf i}. \]

Let $F_{\gr}(n):=\gr\langle\yes 11,\yes 12, \ldots , \yes nn\rangle$ be the
free algebra generated by the $n^2$ symbols $\yes ij$ for $i,j \in
\mg n$. This is a graded algebra, an $\gr$-basis of the $r$-th homogeneous part 
$F_{\gr}(n,r)$ is the set

\[ \{ \yes{\ibf i}{\ibf j}:=\yes{i_1}{j_1}\ldots \yes{i_2}{j_2}\ldots
\yes{i_r}{j_r}|\; \; \ibf i,\ibf j \in \nmind r\}. \]

To simplify notation we introduce a new convention to write down frequently used
elements of $F_{\gr}(n)$ an its quotients in a convenient way.
For an endomorphism $\mu$ on $\ntrts$ we write

\newcommand{\no}{\wr}

\begin{equation}\label{Notation End}
\mu \no \yes{\ibf i}{\ibf j} :=\sum_{\ibf k\in \nmind r} 
\mu_{\ibf i\ibf k}\yes{\ibf k}{\ibf j} \; \; \mbox{ and } \; \;
\yes{\ibf i}{\ibf j}\no\mu :=\sum_{\ibf k\in\nmind r} \yes{\ibf i}{\ibf
  k}\mu_{\ibf k\ibf j}.
\end{equation}

We will denote the residue classes of $\yes ij$ in any quotient of
$F_{\gr}(n)$ by $\xes ij$. The
residue class $\xes{\ibf i}{\ibf j}$ of $\yes{\ibf i}{\ibf j}$ then
clearly has a similar expression in the $\xes ij$ as the $\yes{\ibf i}{\ibf j}$ 
do in the $\yes ij$. The above introduced convention will be used
for $\xes{\ibf i}{\ibf j}$ accordingly.\Ab

The object of our investigations is given by the following definition:


\[ \qsBn{\gr}:=F_{\gr}(n)/\left< \beta\no\yes{\ibf i}{\ibf j}-\yes{\ibf
  i}{\ibf j}\no\beta,\;\gamma \no\yes{\ibf i}{\ibf j}-\yes{\ibf
  i}{\ibf j}\no\gamma |\; \; \ibf i,\ibf j \in \nmind 2\right>. \]

Here the brackets $\langle\rangle$ denote the ideal generated by the enclosed
elements and $\beta, \gamma$ are the endomorphisms on $\nt \otimes
\nt$ defined above.
Since this ideal in the definition is homogeneous, the algebra 
$\qsBn{\gr}=\bigoplus_{r\in \N_0}\qrsBn{\gr} r$ is again graded.
Here, $\qrsBn{\gr} r$ is the $\gr$-linear span of the elements
$\xes{\ibf i}{\ibf j}$ for $\ibf i,\ibf j\in \nmind r$. 
The algebra $\qsBn{\gr}$ can be identified
with a generalized $FRT$-construction with respect to the subset
$N:=\{\beta, \gamma\} \subseteq \Ed\otimes \Ed$ denoted
${\cal  M}_{\gr}(N)$ in \cite[section 5]{ofrt}.
It has been pointed out there
that it possesses the
structure of a bialgebra where comultiplication and augmentation 
on the generators $\xes{\ibf i}{\ibf j}$ are given by

\begin{equation} \label{kom_koe} 
\kom (\xes{\ibf i}{\ibf j})=\sum_{\ibf k\in \nmind r} 
\xes{\ibf i}{\ibf k}  \otimes \xes{\ibf k}{\ibf j} ,\; \; \koe
(\xes{\ibf i}{\ibf j})
=\delta_{\ibf i\ibf j}.
\end{equation}

In particular, the homogeneous summands $\qrsBn{\gr} r$ are subcoalgebras.
Furthermore, the tensor space $\ntrts$ is a 
$\qsBn{\gr}$ (resp.\ $\qrsBn{\gr} r$)-(right-)comodule. The structure map
$\cmd r:\ntrts \pfr \ntrts \otimes \qrsBn{\gr} r$ is defined by

\[ \cmd r (\bs{\ibf j})=\sum_{\ibf i\in \nmind r} 
\bs{\ibf i}\otimes \xes{\ibf i}{\ibf j}. \]

Now, if $q^2-1$ is an invertible element in $\gr$ the endomorphism
$\gamma$ is known to be in the algebraic span of $\beta$, explicitly
one has $\gamma=\frac{q^2\beta^{-1}-\beta}{q^2-1}+1$. Thus, by
\cite[Corollary 2.3]{ofrt} the relations $\gamma\no\xes{\ibf i}{\ibf j}=
\xes{\ibf i}{\ibf j}\no\gamma$ are redundant in this case. The reader may
check that under these circumstances
our bialgebra $\qsBn{\gr}$ is identical to the matrix bialgebra of the
usual FRT-construction 
$F_{\gr}(n)/\left< \beta i\no\yes{\ibf i}{\ibf j}-\yes{\ibf
  i}{\ibf j}\no\beta,\;\; \ibf i,\ibf j \in \nmind 2\right>$
connected with the symplectic group for example denoted 
 ${\cal F}_{\beta}(M_n)$ in \cite[7.3 c]{charipr}.\Ab

On the other hand, if $q^2-1$ is not invertible we really need to add the
relations $\gamma \no \xes{\ibf i}{\ibf j}=\xes{\ibf i}{\ibf j}\no\gamma$.
For instance, it has been proved in \cite[Corollary 6.2]{ofrt} that
setting $q=1$ the bialgebra $\qsBn{\gr}$
is the coordinate ring of the symplectic monoid scheme $\gspm n{\gr}$
which is defined by

\[ \gspm n{\gr}:= \{ A \in \gmhg n{\gr}|\;\exists \;\; d(A)\in {\gr},\;
A^t\spT A=A\spT A^t=\spq (A)\spT \}.\]

Here, $\spT$ is the Gram-matrix of the canonical
skew bilinear form, that is $\spT=(\spT_{ij})_{i,j \in \mg n}$ where
$\spT_{ij}:=\epsilon_i\delta_{ij'}$. The regular 
function $\spq :\gspm n{\gr} \pfr {\gr}$ is
called the {\em coefficient of dilation} (cf. \cite{doty}). 
On the other hand,
in this case the bialgebra of the usual FRT-construction 
equals $\An{\gr}=\gr[\xes 11, \xes 12, \ldots , \xes nn]$, the
commutative polynomial ring in the $\xes ij$, which is just the
coordinate ring of the  monoid scheme $\gmhg n{\gr}$ of $n \times n$-matrices.
Consequently the bialgebra of the usual FRT-construction 
contains $(q^2-1)$-torsion elements considered over the 
ground ring $\gr=\Z[q,q^{-1}]$ of integer 
Laurent polynomials in $q$.\Ab

Let us write down a couple of consequent relations holding in $\qsBn{\gr}$.
For this purpose the algebraic span of the $\ntrts$-endomorphisms


\[ \beta_i := \id{\ntts{i-1}} \otimes \beta
\otimes\id{\ntts{r-i-1}}\;\;\mbox{ and }\;\;
\gamma_i :=\id{\ntts{i-1}} \otimes \gamma
\otimes\id{\ntts{r-i-1}}\;\; i=1, \ldots , r-1
\]


in $\End{\gr}{\ntrts}$ will be denoted by $\rfalg$ (for all $r>1$). 
According to \cite[section 1, 5]{ofrt}
in $\qrsBn{\gr} r$ the following relations hold for all $r>1$:

\begin{equation} \label{Rel MA}
\mu \no\xes{\ibf i}{\ibf j} = \xes{\ibf i}{\ibf j} \no\mu \; \; 
\mbox{ for all } \mu \in \rfalg, \;
\ibf i,\ibf j \in \nmind r.
\end{equation}

The reader should also note that by \cite[Lemma 2.2]{ofrt} all elements
of $\rfalg$ must be  morphisms of $\qrsBn{\gr} r$-comodules.







\section{Quantum Symplectic Bideterminants}
\label{bideterminanten}


Let $p,r\in\N$ be positive integers and $\komp pr$ denote the set of
compositions of $r$ into $p$ parts. These are $p$-tuples $\lambda
=(\lambda_1,\ldots , \lambda_p)$ of non negative integers
$\lambda_i\in\N_0$ summing up to $r$. To each composition $\lambda \in
\komp pr$ there corresponds a parabolic subgroup in the symmetric
group $\symg r$, called the {\em standard Young}
subgroup. We will denote it by $\symg{\lambda}$. It is the subgroup
fixing the sets $\{1, 2, \ldots, \lambda_1\},\; \{\lambda_1+1,
\lambda_1 +2, \ldots, \lambda_1+\lambda_2\},\ldots $.
Now, let $w \in \symg r$ be given by a  reduced expression
$w=s_{i_1}s_{i_2}\ldots s_{i_t}$, where  the $s_i=(i,i+1)$ are the simple
transpositions. We define endomorphisms

\[ \beta (w):=
\beta_{i_1}\beta_{i_2} \ldots \beta_{i_t} \in \rEd:=\End{\gr}{\ntrts}
\]

for $r>1$ and set $\beta (w)=\id{\nt} \in \Ed$ for $r=1$. It is easy
to see that this definition is independent of the choice of the
reduced expression for $w$ since any two of them
can be transformed into each other using the braid relations. But
$\beta$ satisfies the {\em quantum
Yang-Baxter equation} which is just the second type braid relation 

\[ \beta_i\beta_{i+1}\beta_i=\beta_{i+1}\beta_i\beta_{i+1} \]

in the case $i=1$. The latter one obviously implies the relations for $i>1$,
whereas the first type braid relations $\beta_i\beta_j=\beta_j\beta_i$ for
$|i-j|>1$ hold trivially. Observe,

\[ \beta (ww') =\beta (w)\beta (w') \;\;\mbox{ if } \;\; l(ww')=l(w)+l(w').\]

where $l(w)$ denotes the length of $w$, that is the number of transpositions
in a reduced expression. Setting $y:=q^2$ and using our notation 
(\ref{Notation End}) we associate a {\em quantum symplectic bideterminant}
to each triple consisting of a composition $\lambda$ of $r$ and a pair of
multi-indices $\ibf i, \ibf j\in \nmind r$ by

\begin{equation} \label{def qbidet}
\pqbidet{\lambda}{\ibf i}{\ibf j} := \sum_{w \in \symg{\lambda}}
(-y)^{-l(w)}\beta (w)\no\xes{\ibf i}{\ibf j}=
\sum_{w \in \symg{\lambda}} (-y)^{-l(w)}\xes{\ibf i}{\ibf j}\no
\beta (w) .
\end{equation}

The equality therein follows from (\ref{Rel MA}) applied to
$\mu=\beta (w)$. Using the abbreviation
$\kappa_{\lambda}:=\sum_{w \in \symg{\lambda}} (-y)^{-l(w)}\beta
(w)\;\; \in \rEd $ we also may write
$\pqbidet{\lambda}{\ibf i}{\ibf
  j}=\kappa_{\lambda}\no\xes{\ibf i}{\ibf j}=\xes{\ibf i}{\ibf
  j}\no\kappa_{\lambda}$. If $q$ is set to $1$, we obtain

\[ \xes{\ibf i}{\ibf j}\no\beta (w)=\xes{\ibf i}{(\ibf jw^{-1})}\;
\mbox{ and } \; \beta (w) \no\xes{\ibf i}{\ibf j}=\xes{(\ibf iw)}{\ibf j} \]

since then $\beta (w)_{\ibf k\ibf j}=\delta_{\ibf k\ibf jw^{-1}}=
\delta_{\ibf kw\;\ibf j}$. Therefore, in this case our quantum
symplectic bideterminants coincide with ordinary bideterminants which
are defined as products of minor $\lambda_i\times \lambda_i$-determinants,
one factor for each entry
$\lambda_i$ of the composition $\lambda$. In keeping with familar notation
we set 
\[ \qbidet{\lambda}{\ibf i}{\ibf j}:=\pqbidet{\lambda'}{\ibf i}{\ibf j} \]
to each partition $\lambda \in \prt np$. By a partition we mean a composition
$\lambda=(\lambda_1, \ldots ,\lambda_p)$ ordered decreasingly ($\lambda_1\geq
\lambda_2 \geq \ldots \geq \lambda_p\geq 0$) and by $\lambda'$ we denote
the dual of the partition $\lambda$, 
that is $\lambda'=(\lambda_1', \ldots, \lambda_s')$ where $s=\lambda_1$
and $\lambda'_i:=|\{j|\; \lambda_j \geq i\}|$. Using this notation 
one obtains precisely the classical
bideterminant $ \bidet{\lambda}{\ibf i}{\ibf j}$ (as defined
in \cite[2.4]{martin} for instance) when $q$ is set to $1$.
Observe that the capital $T$ notation is more restricted since
not all compositions occur as duals of partitions. This makes it necessary
to consider $\pqbidet{\lambda}{\ibf i}{\ibf j}$ as well for technical 
reasons.\Ab

It should be remarked that the well known quantum determinants
corresponding to the general linear groups
(see for example \cite[4.1.2, 4.1.7]{dd},
\cite[p. 236]{charipr}, \cite[p. 152]{tak}, \cite[p. 157]{hay1})
can be defined in a similar
way using the quantum Yang-Baxter operator of type $A$ instead of our
$\beta$. In contrast, explicit expressions for 
quantum symplectic bideterminants become very complicated for $r > 2$
(apart from the case $\lambda =\alpha_r:=(r)\in \prt 1r$ in which the
bideterminants $\qbidet{\alpha_r}{\ibf i}{\ibf j}$
just are the monomials $\xes{\ibf i}{\ibf j}$). Denoting the fundamental weights
by $\omega_r:=(1,1, \ldots , 1)\in \prt rr$
one obtains a single $r\times r$-minor determinant. If $r=2$,  explicit
expressions are for example


\[
\left| \begin{array}{cc} \xes{k}{i} & \xes{k}{j} \\
\xes{l}{i} & \xes{l}{j}
\end{array} \right|_q:=  \qbidet{\omega_2}{(k,l)}{(i,j)}=
\xes{k}{i}\xes{l}{j}-q^{-1}\xes{k}{j}\xes{l}{i}\] 

if $k<l, i<j, i\neq j'=n-j+1$ and

\[ 
\left| \begin{array}{cc} \xes{k}{i} & \xes{k}{i'} \\
\xes{l}{i} & \xes{l}{i'}
\end{array} \right|_q:= \qbidet{\omega_2}{(k,l)}{(i,i')}=
\xes{k}{i}\xes{l}{i'}-q^{-2}\xes{k}{i'}\xes{l}{i}
-(q^{-2}-1)\sum_{j=1}^{i-1}q^{j-i}\xes{k}{j'}\xes{l}{j},
\]

in the cases $k<l, i\leq m$. The calculation of
$\qbidet{\omega_3}{(j,k,l)}{(i,i',i)}$ for $j<k<l, i\leq m$
is really hard work. Note that such a bideterminant might be
different from
zero even though it contains two identical columns. 
%While the first example looks quite similar to 
%quantum determinants of general linear groups or even ordinary determinants,
%the last fact seems to be very strange even when compared to quantum general
%linear groups.\Ab




\section{Quantum Coefficient of Dilation}
\label{dilatationskoeffizient}

In the definition of the symplectic monoid $\gspm n{\gr}$ we have introduced a
function called the {\em coefficient of dilation}. This is neccessarily
a regular function. Now we will define its quantization which will be
called the {\em quantum coefficient of dilation}. 
Using
notation (\ref{Notation End}) we see that

\[ -q^{-\rho_k-\rho_l}\epsilon_k\epsilon_l\;\;\gamma\no
\xes{(k,k')}{(l,l')}=
 q^{-\rho_l}\epsilon_l \sum_{i=1}^nq^{\rho_i}\epsilon_i
\xes il\xes{i'}{l'}.\]

is independent on $k$, whereas

\[
-q^{-\rho_k-\rho_l}\epsilon_k\epsilon_l \;\;
\xes{(k,k')}{(l,l')}\no\gamma =
-q^{-\rho_k}\epsilon_k \sum_{i=1}^nq^{\rho_i}\epsilon_i
\xes ki\xes{k'}{i'}.
\]

is independent of $l$. But, as $\gamma\no
\xes{(k,k')}{(l,l')}= \xes{(k,k')}{(l,l')}\no\gamma$ 
according to (\ref{Rel MA}),
both expressions coincide and consequently are independent of both 
$k$ and $l$. Thus, the element

\newcommand{\qcD}{d_q}

\begin{equation} \label{gruppenahnlich 1}
\qcD := -q^{-\rho_k-\rho_l}\epsilon_k\epsilon_l\;\; \gamma\no
\xes{(k,k')}{(l,l')}= -q^{-\rho_k-\rho_l}
\epsilon_k\epsilon_l \;\;\xes{(k,k')}{(l,l')}\no\gamma
\end{equation}

is well defined in $\qsBn{\gr}$. In fact it is a grouplike element of
this bialgebra. More precisely it is the coefficient function of the
one dimensional subcomodule of $\nt\otimes \nt$ that is spanned by the
tensor

\[\dual{J}:=\sum_{i=1}^n\epsilon_iq^{\rho_i}\bs i\otimes\bs{i'} 
\in \nt\otimes \nt.\]

To see this, note that 
$\dual{J}=\gamma(-q^{-\rho_l}\epsilon_l\bs l\otimes \bs{l'})$ for each $l$
and that $\gamma$ is a morphism of $\qsBn{\gr}$-comodules. One calculates

\[ \cmd{2}(\dual{J} )=\cmd 2\circ\gamma(-q^{-\rho_l}\epsilon_l\bs l\otimes \bs{l'})=
\gamma \otimes \id{}(
\sum_{i,k=1}^n(\bs i \otimes \bs{k})\otimes (-q^{-\rho_l}\epsilon_l\;\;
\xes{(i,k)}{(l,l')}))= \] \[
\dual{J}\otimes q^{-\rho_l}\epsilon_l\sum_{k=1}^n
q^{\rho_k}\epsilon_k \;\xes{(k,k')}{(l,l')}) = \dual{J} \otimes \qcD.\]

In order to obtain a formula which connects $\qcD$ with quantum symplectic
$2\times 2$-determinants one easily
verifies the following equation using induction on $l$:

\begin{equation}\label{summe bis l}
\sum_{i=1}^l q^{-i}\xes{k}{i}\xes{l}{i'}\no \beta = 
\sum_{i=1}^l q^{i-2l}\xes{k}{i'}\xes{l}{i}.
\end{equation}

For $l=m$ we deduce the connection formula

\begin{equation}\label{g und qdet}
\sum_{i=1}^mq^{-i}\qbidet{\omega_2}{(k,l)}{(i,i')}=
\left\{\begin{array}{ll}
q^{-k}\qcD & k=l'\leq m, \\
0 & k\neq l'.
\end{array} \right. 
\end{equation}

















\section{The Symplectic $q$-Schur algebra}
\label{qSchur}

Remember that $\qrsAn{\gr}{r}$ is a coalgebra for each
$r$. Therefore, its dual $\gr$-module inherits the structure of an
$\gr$-algebra. We define

\[ \qssch{\gr}{r}:=\Hom{\gr}{\qrsAn{\gr}{r}}{\gr} \]

and call it the {\em symplectic $q$-Schur algebra}. Two linear forms $\mu, \nu
\in \qssch{\gr}{r}$ are multiplied by convolution, that is

\[ \mu \nu (a):= (\mu \otimes \nu) \circ \kom (a) \]

for all $a \in \qrsAn{\gr}{r}$.   The reader may verify that 
one obtains the symplectic Schur algebra in the classical situation
as defined in \cite{ofrt}. This also is identical to the symplectic
Schur algebra in the sense of {\em S.\ Donkin}, respectively {\em S.\ Doty}
(\cite{donk1} respectively \cite{doty}). 
One aim is to show that the construction
is stable under base changes and that it is a free $\gr$-module. 
Both facts follow when we have shown that $\qrsAn{\gr}{r}$ is free as 
an $\gr$-module.  Further
we want to initiate the study of the representation theory of this algebra. 
The easiest way to do this is to
check that the axioms of a {\em cellular algebra} given by {\em J.\ Graham}
and {\em G.\ Lehrer} in \cite{graham} hold. These axioms
are as follows:\Ab

Let $A$ be an associative unital algebra over a commutative unital ring $\gr$
together with a partially ordered finite set
$\Lambda$ and finite sets $M(\lambda)$ to each $\lambda \in \Lambda$ (the 
set of 
``$\lambda$-tableaux''). $A$ is called a {\em cellular algebra} if the following
properties hold:

\begin{description}
\item[(C1)] $A$ possesses an $\gr$-basis $\{C^{\lambda}_{S,T}|\; \lambda \in \Lambda
, \; S,T \in M(\lambda) \}$.
\item[(C2)] $A$ posesses an $\gr$-linear
involution $^*$ which is an algebra anti-automorphism
such that  ${C^{\lambda}_{S,T}}^*=
C^{\lambda}_{T,S}$ holds for all $\lambda \in \Lambda$ and $S,T\in M(\lambda)$.
\item[(C3)] For all $a \in A, \lambda \in \Lambda$ and $S,T \in M(\lambda)$
the congruence relation

\[ aC^{\lambda}_{S,T}\equiv \sum_{S'\in M(\lambda)} r_a(S',S)C^{\lambda}_{S',T}
\; \; \mbox{ mod } A(<\lambda ), \]

holds, where the elements $r_a(S',S) \in R$ are independent of $T$ and  $A(<\lambda)$
is defined as the $\gr$-linear span of basis elements $C^\mu _{U,V}$ where $\mu < \lambda$
and $U,V \in M(\mu )$.
\end{description}

Starting with these axioms the representation theory of $A$ is developed in
\cite{graham} along the following lines. To each
$\lambda \in \Lambda$ a standard module $W(\lambda)$ is defined on a
free $\gr$-basis $\{C^{\lambda}_S|\; S \in M(\lambda)\}$.
An element $a \in A$ acts on it via
$aC^{\lambda}_S=\sum_{S'\in M(\lambda)}r_a(S',S)C^{\lambda}_{S'}$.
Each $W(\lambda)$ possesses a symmetric
bilinear form $\phi_{\lambda}$ for which the formula
 $\phi_{\lambda}(a^*x,y)=\phi_{\lambda}(x,ay)$ is valid
for all $a \in A$ and $x,y \in W(\lambda)$. In the case where $\gr$ is a field
and $\phi_{\lambda}\neq 0$,
the radical of $W(\lambda)$ is the same as the radical of the bilinear form
$\phi_{\lambda}$. The simple head $L_{\lambda}$
of $W(\lambda)$ then is absolutely
irreducible. In this way a complete set of pairwise non-isomorphic
simple $A$-modules $\{L_{\lambda}|\;\lambda \in \Lambda_0\}$ can be
obtained. Here we have set
$\Lambda_0:=\{\lambda \in \Lambda|\; \phi_{\lambda}\neq 0\}$.\Ab

Denoting the multiplicity of  $L_\mu $ in $W(\lambda)$ by $d_{\lambda\mu }$ to
each $\lambda \in \Lambda$ and $\mu \in \Lambda_0$ 
Graham and Lehrer show that $d_{\lambda\mu }=0$ for $\lambda \leq \mu $ and
$d_{\lambda\lambda}=1$. To each order refining the given partial order on
$\Lambda$ the corresponding decomposition matrix
$D=(d_{\lambda\mu })_{\lambda \in \Lambda , \mu \in \Lambda_0}$ 
is unitriangular.
The Cartan-matrix $C$ can be calculated as
$C=D^tD$. The theory also supplies a criterion to decide whether $A$ is
semisimple or quasi-hereditary. In the first case we must have
${\rm rad}(\phi_{\lambda})=(0)$ for all $\lambda \in \Lambda$ whereas in the
second case $\Lambda_0=\Lambda$ will do.\Ab

Examples of  cellular algebras are the Brauer centralizer algebras
$\br{\gr, x} r$,
Ariki-Koike-Hecke-algebras, Temperley-Lieb and Jones algebras (\cite{graham}).
R.M.\ Green (\cite{greenr}) constructs a $q$-analogue of the
{\em codeterminant basis}  (in the sense of \cite{green2})
for the classical Schur algebra $\sch{\gr} r$
which is cellular as well. 
The corresponding standard modules $W(\lambda)$ are precisely the
$q$-Weyl modules in the sense of \cite{dj2} (see \cite{greenr},
Proposition 5.3.6). \Ab

It should be remarked that the finiteness of $\Lambda$ is not postulated in the
original definition. Since this property is valid in our example we impose this
restriction to avoid unnecessary trouble (cf. discussion in \cite{steffen},
section 3).\Ab

Since we have defined the symplectic $q$-Schur algebra as the dual module
of a coalgebra we now translate the concept of cellular algebras
to coalgebras:\Ab

Let $K$ be a coalgebra over a commutative unital ring $\gr$,
together with a partially ordered finite set 
$\Lambda$ and finite
sets  $M(\lambda)$ to each $\lambda \in \Lambda$. We call $K$ a {\em cellular coalgebra}
if the following properties hold:

\begin{description}
\item[(C1*)] $K$ possesses an $\gr$-basis
$\{D^{\lambda}_{S,T}|\; \lambda \in \Lambda , \; S,T \in M(\lambda) \}$.
\item[(C2*)] $K$ possesses an $\gr$-linear involution $^*$ which is an
coalgebra anti-automorphism, such that ${D^{\lambda}_{S,T}}^*=
D^{\lambda}_{T,S}$ holds for all $\lambda \in \Lambda$ and $S,T\in M(\lambda)$.
\item[(C3*)] For all $\lambda \in \Lambda$ and $S,T \in M(\lambda)$
the congruence relation

\[ \kom(D^{\lambda}_{S,T})\equiv \sum_{S'\in M(\lambda)} h(S',S)
\otimes D^{\lambda}_{S',T} \; \; \mbox{ mod } K \otimes K(>\lambda ) \]

holds, where the coalgebra elements  $h(S',S) \in K$ are independent of $T$ and
$K(>\lambda)$ is defined as the 
$\gr$-linear span of basis elements $D^\mu _{U,V}$ where $\mu > \lambda$
and $U,V \in M(\mu )$.
\end{description}

To an arbitrary $\gr$-coalgebra the dual algebra is well defined. The dual
coalgebra of an algebra is well defined if the algebra is known to be
projective as an $\gr$-module. In the case of a cellular algebra this is
obviously valid. The connection between the above two concepts is given by
the following proposition which can be proved straightforwardly using
structure constants with respect to the bases (cf. \cite[4.2.3]{doc}).

\begin{prop} \label{zell dual}
The dual algebra of a
cellular coalgebra
is a cellular algebra.
The dual coalgebra of a cellular algebra
is a cellular coalgebra. In both cases the corresponding bases and involution maps
can be constructed dual to each other, i.e. in the former case $C^{\lambda}_{S,T}(D^{\mu}_{U,V})$ 
is 1 if $\lambda = \mu,\; S=U$ and $T=V$ but $0$ otherwise and 
$C^{\lambda}_{S,T}({D^{\mu}_{U,V}}^*)={C^{\lambda}_{S,T}}^*(D^{\mu}_{U,V})$.


\end{prop}

According to the proposition our next task is to find a cellular basis
for the coalgebra $\qrsAn{\gr}{r}$ together with an appropriate involution map
such that the axioms of the cellular coalgebra hold. As soon as this is done
the representation theory of $\qssch{\gr}{r}$ is developed to the extent
indicated above. 
%The key to such a cellular basis for $\qrsAn{\gr}{r}$
%is a quantum symplectic version of the famous {\em Straightening Formula}
%for bideterminants.









\section{Results}
\label{basisdefinition}
\newcommand{\blam}{\underline{\lambda}}


We will define a basis for $\qrsAn{\gr}{r}$ consisting of quantum symplectic
bideterminants and powers of the quantum symplectic coefficient of dilation.
Since they are too large in number we have to single out an appropriate subset.
This can be done using so called
$\lambda$-tableaux which will be defined now: To each partition one
associates a {\em Young}-diagram reading row lengths out of the
components $\lambda_i$. For example

\[ \begin{tableau}{4} \cline{1-3}
\tline{ & & }{1}{3}
\tline{ & }{2}{2}
\tline{ & }{2}{2}
\tline{}{3}{1}
\end{tableau}
\]


is associated to $\lambda=(3,2,2,1) \in \prt 48$.
An $\lambda$-tableau $\itab{\lambda} i$ is constructed from the diagram of 
$\lambda$
by inserting the components of a multi-index $\ibf i \in \nmind r$ column by column into the
boxes. In the above example:

\[
\itab{\lambda} i:=
\begin{tableau}{4}\cline{1-3} 
\tline{$i_1$ & $i_5$ & $i_8$}{1}{3}
\tline{$i_2$ & $i_6$}{2}{2}
\tline{$i_3$ & $i_7$}{2}{2}
\tline{$i_4$}{3}{1}
\end{tableau} .
\]

If $\lambda$ is fixed we will sometimes identify multi-indices with
their tableaux.
We put a new order $\prec$ on the set $\mg n$, namely
\[m \prec m' \prec (m-1) \prec (m-1)' \prec \ldots \prec 1 \prec 1'.\]
The reason, why we prefer $\prec$ instead of the order $\ll$ considered
in \cite{ofrt} will become clear later on. Now,
a multi-index $\ibf i$ is called {\em $\lambda$-column standard} if the
entries in $\itab{\lambda} i$ are strictly increasing down columns according
to this order. It is 
called {\em $\lambda$-row standard} if the
entries are weakly increasing along rows and 
{\em $\lambda$-standard} if it is both at the same time. We write
$\asmind{\lambda}$ to denote the subset of $\nmind r$ consisting
of all multi-indices being $\lambda$-standard. Such a multi-index
$\ibf i\in \asmind{\lambda}$
is called {\em $\lambda$-reverse symplectic standard} if for each
index $i\in \mg m$ the occurences of $i$ as well as $i'$ in
$\itab{\lambda} i$
is limited to the first $m-i+1$ rows. The corresponding subset of
$\asmind{\lambda}$ will be denoted by $\symmind{\lambda}$. The reader
should observe that even though this set is different from the one
of $\lambda$-symplectic standard tableaux (as defined in \cite{king}
and denoted $\symind{\lambda}$  in \cite{ofrt})
it has the same number of elements. For, let $\sigma \in \symg n$ be
the permutation transforming the order $\ll$ into $\prec$, that is
$\sigma (i):=(m-i+1)'$ for $i\leq m$ and $\sigma (i):=m-i'+1$ for $i>m$. Then
there is an induced bijection on $\nmind r$ sending
$(i_1, \ldots , i_r)$ to $(\sigma (i_1), \ldots , \sigma (i_r))$ and which
carries the set of $\lambda$-symplectic standard tableaux precisely 
to the set of $\lambda$-reverse symplectic standard tableaux.\Ab

Let us first describe the set $\Lambda$ occuring in the definition of 
the cellular coalgebra:

\[ \Lambda :=\{ \blam :=(\lambda,l)|\; 0\leq l \leq 
\frac r2, \; \lambda \in \prt m{r-2l} \}.\]

According to the definition of a cellular coalgebra to each $\blam =(\lambda, l)
\in \Lambda$ a set $M(\blam)$ must be assigned. We take:

\[ M(\blam ):= \symmind{\lambda}. \]

Finally the basis elements themselves are defined by

\[ D^{\blam}_{\ibf i, \ibf j}:={\qcD}^l\qbidet{\lambda}{\ibf i}{\ibf j}. \]

Now, our principal aim is to prove the following

\begin{thm} \label{Basissatz}
The $\gr$-module  $\qrsAn{\gr} r$ has a basis given by

\[ \B_r:=\{  D^{\blam}_{\ibf i, \ibf j}|\; \blam \in \Lambda, \ibf i, \ibf j
  \in M(\blam) \}. \]
Furthermore, the unique  $\gr$-linear map $^*$ with ${D^{\blam}_{\ibf i, \ibf j}}^*=
D^{\blam}_{\ibf j, \ibf i}$ is an involutory anti-coalgebra-automorphism and the axioms
of a cellular coalgebra are satisfied. 
\end{thm}

By Proposition \ref{zell dual} we may conclude immediately:

\begin{thm}\label{qSchur cellular}
The symplectic $q$-Schur algebra $\qssch{\gr}{r}$ is a cellular algebra
with the basis dual to $\B_r$ as a cellular basis.
\end{thm}

Another direct consequence of \ref{Basissatz} can be obtained from
\cite[Theorems 3.3 and 4.2]{ofrt} together with the observation
that the algebraic span $\rfalg$ 
of the endomorphism $\beta_i$ and $\gamma_i$
equals  the image of the {\em Birman-Murakami-Wenzl}
algebra under its representation on $\ntrts$ that is $\Psi(C_r(-q^{1-n},q))$
in the notation of \cite[Lemma 5.1]{wenzl2}.

\begin{thm}\label{BMW centralizer}
The symplectic $q$-Schur algebra is stable under base change and it is
identical with the centralizer of the {\em Birman-Murakami-Wenzl} algebra
acting on $\ntrts$.
\end{thm}

At the end of this paper we will improve \ref{qSchur cellular} 
by showing that the
bilinear form $\phi_{\lambda}$ on the standard modules $W_{\lambda}$ is
nonzero for each $\lambda$. By \cite[3.10]{graham} this means

\begin{thm} \label{quasi-heriditary}
The symplectic $q$-Schur algebra $\qssch{\gr}{r}$ is integrally
quasi-hereditary.
\end{thm}


Let us first see how the involution of Theorem \ref{Basissatz} arises.
It realizes matrix transposition for our quantum monoid.
On the generators $\xes{\ibf i}{\ibf  j}$ this transposition map is defined 
as in the classical case by 
${\xes{\ibf i}{\ibf j}}^*:=\xes{\ibf j}{\ibf i}$. Indeed, this gives a 
well defined algebra map on $\qsBn{\gr}$, since the coefficient matrices
of $\beta$ and $\gamma$ are symmetric implying ${\beta \no \xes{\ibf
  i}{\ibf j}}^*=\xes{\ibf j}{\ibf i}\no\beta$ and ${\gamma \no\xes{\ibf
  i}{\ibf j}}^*=\xes{\ibf j}{\ibf i}\no\gamma$ and 
thus, keeping the relations of
that algebra fixed. Furthermore, the endomorphisms $\kappa_{\lambda}\in
\End{\gr}{\ntrts}$ must have symmetric coefficient matrices, as well.
We calculate 

\[{\pqbidet{\lambda}{\ibf i}{\ibf j}}^*= 
{\kappa_{\lambda}\no\xes{\ibf i}{\ibf j}}^*=
\xes{\ibf j}{\ibf i}\no\kappa_{\lambda}=
\pqbidet{\lambda}{\ibf j}{\ibf i} \]

and in a similar way ${\qcD }^*=\qcD$. This shows, that $^*$ factors
to an algebra map of $\qsAn{\gr}$. From the comultiplication rule
(\ref{kom_koe}) it directly follows that $^*$ is an anti-coalgebra map. This implies
axiom (C2*) of a cellular coalgebra.\Ab

The verification of axiom (C3*) is the second easiest step 
in the proof of Theorem \ref{Basissatz}, but we will give it at the end of the paper
since some additional ingredients are needed.
The first statement of this theorem, which is just axiom (C1*),
is the really hard one. It is the
$q$-analogue of \cite[Theorem 6.1]{ofrt}. To prove it
we will proceed in a similar way as there. 
The difficulty is to show that $\B_r$ is a set of generators. 
For that porpose the most important step 
is a quantum symplectic version of the famous
straightening formula. 















\section{The Quantum Symplectic Straightening Formula}
\label{st formula}



In the classical case symplectic versions of the straightening formula
have already been given in \cite[2.4]{concini} and
\cite[section 7]{ofrt}. In principle, we will
follow the lines of the latter paper. 
But there are a lot of additional difficulties,
one of which forces us to work with a reversed version of $\lambda$-symplectic
standard tableaux.
Preparing the statement we define the algebra

\[ \qshAn{\gr} :=\qsAn{\gr}/\left< \qcD \right> \]

\newcommand{\abb}{{\cal K}}

by factoring out the ideal generated by the quantum coefficient of dilation.
Since $\qcD$ is homogeneous this algebra is graded again. Let us abbreviate
its $r$-th homogeneous summand by $\abb :=\qrshAn{\gr} r$.
Since $\qcD$ is grouplike the comultiplication $\kom$ obviously factors to
$ \qshAn{\gr}$ and $ \qrshAn{\gr} r$. But $ \qshAn{\gr}$ is not a bialgebra and
$ \qrshAn{\gr} r$ are not coalgebras, because the augmentation map $\koe$
does not factor. In the classical case if $\gr=K$ is a field 
$\qshAn{K}$ equals the coordinate 
ring of the symplectic semigroup $\mmsp nK:=\gspm nK\ohne \gsp nK$ 
by \cite[remark 7.5]{ofrt}. The missing augmentation map corresponds
to the missing unit element in the semigroup.\Ab

We put an order on the set
$\rprt r$ of all 
partitions of $r$, writing $\lambda < \mu$ if the dual $\lambda'$
occurs before the dual $\mu'$ in the lexicographic order.
In this order the fundamental weight $\omega_r:=(1,1, \ldots , 1)\in
\prt rr$ is the largest element, whereas $\alpha_r:=(r)\in \prt 1r $ is the
smallest one. We define $\abb (>\lambda)$
(resp.\ $\abb (\geq \lambda)$) to be the $\gr$-linear span in $\abb $ of all
bideterminants $\qbidet{\mu}{\ibf i}{\ibf j}$ such that $\mu > \lambda$
(resp.\ $\mu \geq \lambda$). For $\lambda=\omega_r$ we set
$\abb (>\omega_r):=0$. Clearly $\abb =\abb (\geq \alpha_r)$. 

\begin{prop}[Quantum Symplectic Straightening Formula] 
\label{straightening formula}
Let $\lambda \in \rprt r$ be a partition of $r$ and $\ibf j \in \nmind r$.
Then, to each $\ibf k\in \symmind{\lambda}$ there is an element
$a_{\ibf j\ibf k} \in \gr$,
such that in $\abb $ we have for all $\ibf i\in \nmind r$:


\[
\qbidet{\lambda}{\ibf i}{\ibf j} \equiv \sum_{\ibf k \in \symmind{\lambda}}
a_{\ibf j\ibf k}
\qbidet{\lambda}{\ibf i}{\ibf k} \; \; \mbox{\rm mod } \abb (>\lambda) .
\]
\end{prop}

Before starting to prove this, let us deduce the fact that $\B_r$ generates
$\qrsAn{\gr} r$. Two additional ingredients are needed. First, we use the fact
that $\qcD$ is a central element in $\qsAn{\gr}$, 
which can be deduced
from the identity $\beta_1\beta_2\gamma_1=\gamma_2\beta_1\beta_2$ holding
in $\gfalg 3$ (for details we refer the reader to \cite[Corollary 6.3]{hay1}
or \cite[Satz 3.4.1]{doc}). From this fact we see that 
multiplication by $\qcD$ from the left leads to an exact sequence

\begin{equation}\label{sequenz}
\qrsAn{\gr}{r-2}\stackrel{\cdot\qcD}{\pfr} \qrsAn{\gr} r \pfr \qrshAn{\gr} r \pfr 0. 
\end{equation}

for $r>1$.
Therefore, using induction on $r$ we can reduce to showing that

\[ \{ \qbidet{\lambda}{\ibf i}{\ibf j}|\; \lambda \in \prt m{r},\;
\ibf i,\ibf j \in \symmind{\lambda}\} \]

is a set of generators for $\abb =\qrshAn{\gr} r$. For this claim it is
enough to show that

\[ \B_{\lambda}:=\{ \qbidet{\lambda}{\ibf i}{\ibf j}|\;\;
\ibf i,\ibf j \in \symmind{\lambda}\} \]

is a set of generators of $\abb (\geq \lambda)/\abb (>\lambda)$ for each partition
$\lambda$. To get the last claim from the
straightening formula \ref{straightening formula}, observe that
the involution $^*$ is well defined on
$\qshAn{\gr}$ since ${\qcD}^*=\qcD$.  Applying $^*$
to the congruence relation of Proposition \ref{straightening formula},
one obtains another such formula in which the roles of $\ibf i$ and
$\ibf j$ are exchanged.  This shows that $ \B_{\lambda}$ is indeed a set of
generators for  $\abb (>\lambda)/\abb (\geq \lambda)$.
Thus the fact that $\B_r$ is a set of generators for
$\qrsAn{\gr} r$ is reduced to the validation of Proposition
\ref{straightening formula}.\Ab

In order to prove the quantum symplectic straightening formula we need a corresponding
algorithm. Its classical counterpart is \cite[Proposition 7.3]{ofrt}.
The reader familiar with that proposition will remember the importance
of a certain order on the set $\nmind r$. A similar, but in some sense
reversed version of that order will be needed now.
This circumstance forces us to consider reverse symplectic
standard tableaux instead of symplectic ones. To 
introduce that order we define a map 
$f:\nmind r\pfr \N_0^m$ by $f(\ibf i)=(a_1, \ldots ,
a_m)$, where 

\[ a_l:=|\{ j\in \mg r|\; i_j=l \;\mbox{ or } \; i_j=l' \}|, \]

and order $\N_0^m$ writing $(a_1,\ldots , a_m)<(b_1, \ldots , b_m)$ 
if and only if $(b_1, b_{2}, \ldots , b_m)$ appears before $(a_1,
a_{2}, \ldots , a_m)$ in the lexicographic order (induced by the ordinary
order on $\N$). This is the step where we change directions relative
to \cite[7.3]{ofrt}. Next, we
obtain an order $\lhd$ on $\N_0^m\times \nmind r$ in a lexicographic way as
well:

 \[(a, \ibf i)\lhd (b, \ibf j) :\gdw a < b \mbox{ or }
(a = b \mbox{ and } \ibf i \prec \ibf j).\]

Here, we have denoted by $\prec$
the lexicographic order on $\nmind r$ induced by our special
order $\prec$ on $\mg n$.
Finally, we obtain a second, new order $\lhd$ on $\nmind r$ via the embedding
$\nmind r\hookrightarrow \N_0^m\times \nmind r$ given by $\ibf
i\mapsto (f(\ibf i), \ibf i)$. Now we are able to state the symplectic
straightening algorithm.



\begin{prop}[Strong Quantum Symplectic Straightening Algorithm] 
\label{Straightening Algorithmus 2}
Let \\ 
$\lambda \in \rprt r$ be a partition of $r$
and $\ibf j \in \nmind r \ohne \symmind{\lambda}$. Then to each 
$\ibf k\in \nmind r$ satisfying $\ibf k\lhd \ibf j$ there is an element
$a_{\ibf j\ibf k} \in \gr$ such that
in $\abb $ the following congruence relation holds
 for all $\ibf i\in \nmind r$:


\[
\qbidet{\lambda}{\ibf i}{\ibf j} 
\equiv \sum_{\ibf k\lhd \ibf j} a_{\ibf j\ibf k}
\qbidet{\lambda}{\ibf i}{\ibf k} \; \; \mbox{ mod } \; \; \abb (>\lambda) .
\]
\end{prop}

In principle we will proceed in a similar way as in \cite{ofrt} to prove this
algorithm, 
but, there is a remarkable difference concerning
the role of the classical straightening algorithm 
(Proposition 7.2 in that paper).
One would expect that this role has to be taken over by a $q$-analogue
related to quantum general linear groups as can be found
in \cite[section 9]{hay3}. But this is not true because the embedding of
the symplectic group into the general linear group does not extend to quantum
groups. Instead of \cite[Proposition 7.2]{ofrt} we have to establish a 
weak form of the quantum symplectic straightening algorithm in a first step.
More precisely we will first prove \ref{Straightening Algorithmus 2},
where $\symmind{\lambda}$ is substituted by $\asmind{\lambda}$.\Ab


Clearly, the straightening formula \ref{straightening formula} is
an easy consequence of the above proposition since the set $\nmind r$
is finite and therefore the elimination of multi-indices $\ibf j$, that are not
 $\lambda$-reverse symplectic standard  in an expression
$\bidet{\lambda}{\ibf i}{\ibf j}$
must terminate.\Ab

The proof of the straightening algorithm will take several sections.
We start with some technical tools needed for the first step, the above
mentioned weak form.
























\section{Arithmetic of Bideterminants}


The calculus of bideterminants is needed inside $\abb =\qrshAn{\gr} r$.
Unless otherwise stated the rules hold in $\qrsAn{\gr} r$,
too.  Recall the definition of $\kappa_{\lambda}$ from 
section \ref{bideterminanten} following (\ref{def qbidet}).

\begin{lem} \label{1-ybetai Formel} 
Let $\lambda\in \komp pr$ be a composition and $y=q^2$.
Then to each $i<r$ such that the simple transposition $s_i=(i,i+1)$ is 
contained in the standard Young-subgroup
$\symg{\lambda}$, there are
endomorphisms $\mu_{\lambda,i}, \mu'_{\lambda,i}\in \rEd$
satisfying

\[\kappa_{\lambda}=(\id{\ntrts}-y^{-1}\beta_i)\mu_{\lambda ,i}=
\mu'_{\lambda,i}(\id{\ntrts}-y^{-1}\beta_i).\]
\end{lem}

\begin{Pf} Let us first reduce to the case $\lambda =\alpha_r=(r)$. Setting
$\kappa_r:=\kappa_{\alpha_r}$, $k_s:=\lambda_1 + \ldots +  \lambda_{s-1}$
and 
\[ \kappa_{\lambda}^s:=\id{\ntts{k_s}}
\otimes \kappa_{\lambda_s}\otimes\id{\ntts{r-\lambda_s-k_s}},\]
we can extend the definition to arbitrary $\lambda$
by using the formula 
$\kappa_{\lambda}=\kappa_{\lambda}^1\kappa_{\lambda}^2\ldots 
\kappa_{\lambda}^p$ in which the factors commute.
Now, using standard reduced expressions for permutations $w \in \symg r$
one easily verifies the following recursion rules for $r>1$:

\[ \kappa_r=\kappa_{r-1}(\id{\ntrts}+
\sum_{l=1}^{r-1}(-y)^{l-r}\beta_{r-1}\beta_{r-2}
\ldots \beta_l)=\] \[ (\id{\ntrts}+
\sum_{l=1}^{r-1}(-y)^{l-r}\beta_l\beta_{l+1}
\ldots \beta_{r-1})\kappa_{r-1}. \]

We proceed by induction on $r$, the case  $r=2$ being clear.
The case $i<r-1$ can be handled immediately with the help of the
above recursion formula. If $i=r-1$ we calculate


\[ \kappa_r = \kappa_{r-1}(\id{\ntrts}-y^{-1}\beta_{r-1})+
\mu'_{r-1,r-2}(\id{\ntrts}-y^{-1}\beta_{r-2})
\sum_{l=1}^{r-2}(-y)^{l-r}\beta_{r-1}\beta_{r-2}
\ldots \beta_l.
\]

But by the braid relations we get

\[(\id{\ntrts}-y^{-1}\beta_{r-2})(-y)^{l-r}\beta_{r-1}\beta_{r-2}\ldots 
\beta_l=
(-y)^{l-r}\beta_{r-1}\beta_{r-2}\ldots \beta_l(\id{\ntrts}-y^{-1}\beta_{r-1}),
\]

yielding the right hand side factorization of $\kappa_r$. The other formula
is obtained similarly.
\end{Pf}

\begin{cor} \label{qdet Null}
Let $\ibf j\in \nmind r$ be a multi-index
possessing two identical neighbouring indices
$j_l=j_{l+1}$ and $\lambda\in \komp pr$ such that the transposition
$s_l$ is contained in $\symg{\lambda}$. Then 
$\pqbidet{\lambda}{\ibf i}{\ibf j}=\pqbidet{\lambda}{\ibf j}{\ibf i}=0$
holds for all $\ibf i\in \nmind r$.
\end{cor}

\begin{Pf} By assumption, $\bs{\ibf j}$ lies in the kernel of 
$(\id{\ntrts}-y^{-1}\beta_l)$. Consequently the assertion concerning
$\pqbidet{\lambda}{\ibf i}{\ibf j}$ follows immediately from 
Lemma \ref{1-ybetai Formel} by the definition of bideterminants.
Using the matrix transposition map $^*$ introduced in
section \ref{st formula}, the formula for exchanged multi-indices follows
as well.
\end{Pf}



\newcommand{\gfkern}[1]{{\cal G}_{#1}}
\newcommand{\rfkern}{\gfkern r}

Next, we will demonstrate how the transition from $\qsAn{\gr}$
to its epimorphic image $\qshAn{\gr}$
corresponds to the transition from the \BMW  
algebra to the Iwahori-Hecke algebra of type $A$ via the
epimorphism sending the generator $e_1$ to zero. For this purpose
denote by $\rfkern$ the ideal generated by $G:=\gamma_1=\gamma\otimes
\id{\ntts{r-2}}$
in the image $\rfalg$ of the \BMW algebra in
$\rEd$.
Note that it contains the other endomorphisms $\gamma_i$ for $i=2, \ldots ,
r-1$, as well (use the relations
$\beta_i\beta_{i+1}\gamma_i=\gamma_{i+1}\beta_i\beta_{i+1}$).
The quotient $\rfalg /\rfkern$ is an epimorphic image
of the Iwahori-Hecke algebra sending the generator $T_{s_l}$ to
$\beta_l +\rfkern$ (notation as in \cite{dd}).


\begin{lem} \label{Reduktion Hecke}
Let $A, B\in \rfalg$ be endomorphisms of $\ntrts$ such that
$A\equiv B$ modulo $\rfkern$. Then, the equation
$\xes{\ibf i}{\ibf j}\no A=\xes{\ibf i}{\ibf j}\no B$ 
(to be read in the sense of
(\ref{Notation End}) ) holds in $\abb$. 
\end{lem}

\begin{Pf}
We have to show that $\xes{\ibf i}{\ibf j}\no A=0$
for all $A\in \rfkern$. Let $F,H\in \rfalg$ be such
that $A=FGH$. From the defining equation of the quantum coefficient
of dilation $\qcD$ from section \ref{dilatationskoeffizient} we have


\[ \xes{\ibf i}{\ibf j}\no G=
\left\{ \begin{array}{ll}
0 & j_1'\neq j_2 \mbox{ or } i_1'\neq i_2 \\
-q^{\rho_{i_1}+\rho_{j_1}}\epsilon_{i_1}\epsilon_{j_1}\; \qcD \;
\xes{i_3}{j_3}\ldots \xes{i_r}{j_r} & j_1'=j_2 \mbox{ and } i_1'=i_2\\
\end{array} \right..
\]

This means $\xes{\ibf i}{\ibf j}\no G=0$ in $\abb$ for all
$\ibf i,\ibf j \in \nmind r$. By (\ref{Rel MA}) we have  
$\xes{\ibf i}{\ibf j}\no F=F\no \xes{\ibf i}{\ibf j}$ and therefore,

\[ \xes{\ibf i}{\ibf j}\no FGH=\sum_{\ibf k,\ibf l,\ibf s \in \nmind r}
\xes{\ibf i}{\ibf k}f_{\ibf k\ibf l}g_{\ibf l\ibf s}h_{\ibf s\ibf j}=
\sum_{\ibf k,\ibf l,\ibf s \in \nmind r}
f_{\ibf i\ibf k}\xes{\ibf k}{\ibf l}g_{\ibf l\ibf s}h_{\ibf s\ibf j}=
\sum_{\ibf k,\ibf s \in \nmind r}
f_{\ibf i\ibf k}\xes{\ibf k}{\ibf s}\no Gh_{\ibf s\ibf j}= 0.\]

where $(f_{\ibf i\ibf j})_{\ibf i, \ibf j \in \nmind r}$,
$(g_{\ibf i\ibf j})_{\ibf i, \ibf j \in \nmind r}$ and
$(h_{\ibf i\ibf j})_{\ibf i, \ibf j \in \nmind r}$
are the coefficient matrices of $F, \; G$ and $H$.
\end{Pf}

We introduce another notation similar to the one in
(\ref{Notation End}). Let $\mu \in \rEd$ be an endomorphism of $\ntrts$. Set

\[
\pqbidet{\lambda}{\ibf i}{\ibf j}\no\mu=\sum_{\ibf k \in \nmind r}
\pqbidet{\lambda}{\ibf i}{\ibf k}\mu_{\ibf k\ibf j} \;\;\mbox{ and } \;\;
\mu\no\pqbidet{\lambda}{\ibf i}{\ibf j}=\sum_{\ibf k \in \nmind r}
\mu_{\ibf i\ibf k} \pqbidet{\lambda}{\ibf k}{\ibf j}.
\]

Similar expressions are used with respect to the capital $T$
notation for bideterminants.

\begin{lem} \label{inverse Formel}
\label{Formel in mSp}
For all $\ibf i, \ibf j\in \nmind r$ and $w \in \symg{\lambda}$ the
following equations hold in $\abb$.
\[\beta (w)\no\pqbidet{\lambda}{\ibf i}{\ibf j}=
\beta (w)^{-1}\no\pqbidet{\lambda}{\ibf i}{\ibf j}=
(-1)^{l(w)}\pqbidet{\lambda}{\ibf i}{\ibf j}
=\pqbidet{\lambda}{\ibf i}{\ibf j}\no\beta (w)^{-1}
=\pqbidet{\lambda}{\ibf i}{\ibf j}\no\beta (w) \]
\end{lem}

\begin{Pf}
Modulo $\rfkern$ we have

\[ \beta (w)\kappa_{\lambda}=\beta (w)^{-1}\kappa_{\lambda}=
(-1)^{l(w)}\kappa_{\lambda} =
\kappa_{\lambda} \beta (w)^{-1}=\kappa_{\lambda} \beta (w)
\]

since the corresponding equations (where $\beta (w)$ is replaced by
$T_w$)
hold in the Iwahori-Hecke algebra of
type $A$. Thus the assertion follows from Lemma \ref{Reduktion Hecke}.
\end{Pf}


\newcommand{\cJ}{{\cal J}}
\newcommand{\gcJ}[1]{{\cJ}_{#1}}
\newcommand{\rcJ}{\gcJ r}

Let $\cJ$ denote the ideal in the tensor algebra $\tens{\nt}=
\bigoplus_{r \in \N_0} \ntrts$ generated by the twofold invariant
tensor $\dual{J}=\sum_{i=1}^n\epsilon_iq^{\rho_i}\bs i\otimes\bs{i'} 
\in \nt\otimes \nt$ and let $\rcJ:=\cJ \cap \ntrts$ be 
its $r$-th homogeneous summand.


\begin{lem}\label{I_{cqT}^r}
Let $a_{\ibf j} \in \gr$ be such that
$\sum_{\ibf j\in \nmind r}a_{\ibf j}\bs{\ibf j} \in \rcJ$. Then, for 
all $\ibf i \in\nmind r$ and compositions $\lambda$ of $r$ we have

\[ \sum_{\ibf j\in \nmind r} a_{\ibf j}\xes{\ibf i}{\ibf j}=0, \;\;
\mbox{ and }\;\;
 \sum_{\ibf j\in \nmind r} a_{\ibf j}\pqbidet{\lambda}{\ibf i}{\ibf j}=0. \]
\end{lem}

\begin{Pf}
First, note that the second equation follows from the first one by definition
of bideterminants. Again, we use the equation
$\dual{J}=\gamma (-q^{-\rho_k}\epsilon_k\bs k\bs{k'})$ holding for each
$k \in \mg n$. It implies that $\rcJ$ is contained in the $\gr$-linear
span of all elements of the form $\gamma_l(\bs{\ibf k})$
for $\ibf k\in \nmind r$ and $1 \leq l < r$. On the other hand, it is easily
seen that allways $\gamma_l(\bs{\ibf k})\in \rcJ$. Thus, we have reduced
to the case
$\sum_{\ibf j\in \nmind r}a_{\ibf j}\bs{\ibf j} =\gamma_l(\bs{\ibf k})$.
This means
\[ \sum_{\ibf j\in \nmind r} a_{\ibf j}\xes{\ibf i}{\ibf j}=
\xes{\ibf i}{\ibf k}\no\gamma_l=0. \]
Since $\gamma_l$ is contained in $\rfkern$ the proof is finished by
Lemma \ref{Reduktion Hecke}.
\end{Pf}


Next, we give a quantum symplectic version of {\em Laplace duality}. The
corresponding classical result can be found in \cite[2.5.1]{martin}, for
instance.

\begin{prop}[Laplace Duality]\label{Laplace 1}
Let $\lambda,\mu  \in\komp pr$ be compositions, 
$Y$ a set of left coset representatives of $\symg{\lambda}\cap \symg\mu $
in $\symg{\lambda}$ and $X$ a set of right coset representatives of
$\symg{\lambda}\cap \symg\mu $
in $\symg\mu $, such that $l(vw)=l(v)+l(w)$ and $l(wu)=l(u)+l(w)$ holds
for all $v \in Y, u \in X$ and $w \in \symg{\lambda}\cap \symg\mu $.
Then for all $\ibf i, \ibf j \in \nmind r$ the following equation holds:

\[
\sum_{u \in X}(-y)^{-l(u)}\pqbidet{\lambda}{\ibf i}{\ibf j}\no\beta (u)=
\sum_{v \in Y}(-y)^{-l(v)}\beta (v)\no\pqbidet\mu {\ibf i}{\ibf j}.
\]

\end{prop}

\begin{Pf}
Using the disjoint union
\[Z:=\symg{\lambda}\symg\mu =
\bigcup_{u\in X}\symg{\lambda}u=\bigcup_{v\in Y}v\symg\mu , \]

we calculate

\[ 
\sum_{u \in X}(-y)^{-l(u)}\pqbidet{\lambda}{\ibf i}{\ibf j}\no\beta (u)=
\sum_{u\in X} \sum_{w \in \symg{\lambda}}
(-y)^{-l(u)-l(w)}\xes{\ibf i}{\ibf j}\no\beta (w)\beta (u)=\]
\[
\sum_{z \in Z}(-y)^{-l(z)}\xes{\ibf i}{\ibf j}\no \beta (z)= 
\sum_{z \in Z}(-y)^{-l(z)}\beta (z) \no\xes{\ibf i}{\ibf j}=\]
\[
\sum_{v\in Y} \sum_{w \in \symg\mu }
(-y)^{-l(v)-l(w)}\beta (v)\beta (w)\no\xes{\ibf i}{\ibf j}=
\sum_{v\in Y} (-y)^{-l(v)}\beta (v)\no\pqbidet\mu {\ibf i}{\ibf j}.
\]
\end{Pf}


The next result is needed for the transition from 
t-bideterminants of compositions
to T bideterminants of
partitions.

\begin{lem}\label{praebidet}
Let $\lambda\in \komp pr$ be a composition and $\ibf i, \ibf j \in \nmind r$. 
Then the bideterminant $\pqbidet{\lambda}{\ibf i}{\ibf j}$ can be written
as a linear combination of bideterminants
$\qbidet{\lambda '}{\ibf k}{\ibf l}$.
\end{lem}

\begin{Pf}
First, there is a permutation $\pi \in \symg p$, such that
$\bar{\lambda}=(\lambda_{\pi(1)}, \ldots , \lambda_{\pi(p)})\in \prt pr$
is a partition. This $\bar{\lambda}$ is uniquely determined by $\lambda$ 
(but $\pi$ only under the restriction to be of minimal length).
Clearly the parabolic subgroups $\symg{\lambda}$ and
$\symg{\bar{\lambda}}$ in $\symg r$ are conjugate to each other. Thus, there
is an element $v \in \symg r$ 
such that $v\symg{\lambda}=\symg{\bar{\lambda}}v$. Furthermore, it is known
from the theory of parabolic subgroups that in the left coset
$v\symg{\lambda}$ and in the right coset
$\symg{\bar{\lambda}}v$ there are unique representatives $w$ resp. $\bar w$
of minimal length. In fact we have $w=\bar w$, but this is not important for
the proof. The facts we really need are
$l(wu)=l(w)+l(u)$ for all $u \in \symg{\lambda}$ and $l(u\bar w)=l(u)+
l(\bar w)$ for all $u \in \symg{\bar{\lambda}}$. For, this implies
$\beta(w)\kappa_{\lambda}=\kappa_{\bar{\lambda}}\beta(\bar w)$.
By the definition of bideterminants and the relations (\ref{Rel MA}) 
holding inside $\qrsBn{\gr} r$ we obtain

\[ \beta(w)\no\pqbidet{\lambda}{\ibf i}{\ibf j}=
\pqbidet{\bar{\lambda}}{\ibf i}{\ibf j}\no\beta(\bar w)=
\qbidet{\bar{\lambda}'}{\ibf i}{\ibf j}\no\beta(\bar w). \]

Since $\bar{\lambda}'=\lambda '$ this results in

\[ \pqbidet{\lambda}{\ibf i}{\ibf j}=
\beta(w)^{-1}\no\qbidet{\lambda '}{\ibf i}{\ibf j}\beta(\bar w) =
\sum_{\ibf k, \ibf l \in \nmind r}\beta (w)^{-1}_{\ibf i\ibf k}
\bidet{\lambda '}{\ibf k}{\ibf l} \beta (\bar w)_{\ibf l\ibf j}. \]
\end{Pf}



Next, we introduce a calculus for our bideterminants bringing
our special order $\lhd$ on $\nmind r$ into the picture. First,
some new notation has to be explained. The sum of two
multi-indices  $\ibf i\in \nmind r$ and $\ibf j\in \nmind s$ is
defined by juxtaposition, that is

\[ \ibf i+\ibf j:=(i_1,\ldots , i_r, j_1, \ldots , j_s) \;\in \;\nmind{r+s}. \]

Note that the map $f:\nmind r \pfr \N_0^m$ occuring in the
straightening algorithm is additive in the sense $f(\ibf i+\ibf j)=
f(\ibf i)+f(\ibf j)$. This implies

\begin{equation} \label{Additivitat}
 f(\ibf i +\ibf j)< f(\ibf i+\ibf k)\;\;\mbox{ resp.\ }\;\;
 f(\ibf j +\ibf i)<f(\ibf k+\ibf i)\;\;\mbox{ if }\;\;
f(\ibf j)<f(\ibf k)
\end{equation}

with respect to the lexicographic order $<$ on $\N_0^m$. 
To a  multi-index
$\ibf i \in \nmind r$ we consider the following $\gr$-spans
in $\ntrts$

\[ W_{\ibf i}:=\left< \bs{\ibf j}|\;\ibf j \in \nmind r , \; f(\ibf j)<
f(\ibf i)\right> \; \;\mbox{ and }\;\;
 \overline W_{\ibf i}:=
\left< \bs{\ibf j}|\;\ibf j \in \nmind r , \; f(\ibf j) \leq
f(\ibf i)\right>.
\]

Furthermore, we set

\[ h_{ij}:= \left\{ \begin{array}{ll}
q^{-1} & j\neq i, i' \\
1 & j=i,  j=i'>m \\
q^{-2} & j=i'\leq m
\end{array} \right.   .
\]

and denote the simple transpositions by $s_l=(l,l+1)$, as before.
The following lemma is the key concerning calculations with bideterminants.
Again we set $y=q^2$.

\begin{lem} \label{Formel W_i}
For all $\ibf i\in \nmind r$ and $l \in \mg r$ the following formulas
hold in $\ntrts$ modulo the $\gr$-module $W_{\ibf i}$:

\[ \beta_l(\bs{\ibf i})\equiv \left\{ \begin{array}{ll}
yh_{i_{l+1}i_{l}}\bs{\ibf is_l} + (y-1)
(\id{\ntrts}-\gamma_l)(\bs{\ibf i}) & i_l > i_{l+1} \\
yh_{i_{l+1}i_{l}}\bs{\ibf is_l}  & i_l \leq i_{l+1}
\end{array} \right. \]

\[ \beta_l^{-1}(\bs{\ibf i})\equiv \left\{ \begin{array}{ll}
h_{i_{l+1}i_{l}}\bs{\ibf is_l} + (y^{-1}-1)
(\id{\ntrts}-\gamma_l)(\bs{\ibf i}) & i_l \leq i_{l+1} \\
h_{i_{l+1}i_{l}}\bs{\ibf is_l}  & i_l > i_{l+1}
\end{array} \right. .\]

\end{lem}

\begin{Pf}
The congruence relation for $\beta_l^{-1}$ follows from the one for $\beta_l$ 
because $y\beta^{-1}=\beta+(y-1)(\gamma-\id{\ntts 2})$.
Therefore, it is enough to prove the first assertion.\Ab

First, consider the case $i_l > i_{l+1}$. If $i_l\neq i_{l+1}'$, 
the asserted congruence relation is also an equation, as can be seen directly
from the definition of $\beta$. Turning to the case
$i_l=i_{l+1}'=:j\leq m$, we split $\ibf i$ into three summands

\[ \ibf i^1=(i_1, \ldots , i_{l-1}), \; \;
\ibf i^2=(j',j), \; \;
\ibf i^3=(i_{l+1}, \ldots , i_r) .\]

To $k \in \mg n$ we set $\ibf i(k):=\ibf i^1 + (k,k') + \ibf i^3$ and calculate

\[  \beta_l(\bs{\ibf i}) = \bs{\ibf is_l} + (y-1)\bs{\ibf i}
 -(y-1)\sum_{k>j}q^{\rho_k-\rho_j}\epsilon_k
\bs{\ibf i(k)} .\]

Since
$(y-1)\sum_{k=1}^nq^{\rho_k-\rho_j}\epsilon_k\epsilon_j
\bs{\ibf i(k)} =(y-1)\gamma_l(\bs{\ibf i})$ we obtain the equation

\[
\beta_l(\bs{\ibf i}) = 
\bs{\ibf is_l} + (y-1)(\id{\ntrts}-\gamma_l)(\bs{\ibf i})
 +(y-1)\sum_{k\leq j}q^{\rho_k-\rho_j}
\bs{\ibf i(k)} .\]

But  $\ibf i(j)=\ibf is_l$ and
\[ f(\ibf i(k))=f(\ibf i^1) + f((k,k')) +
f(\ibf i^3)  <  f(\ibf i^1) + f((j',j)) +
f(\ibf i^3)  =f(\ibf i(j'))=f(\ibf i)\]

for all $k < j$ by (\ref{Additivitat}), yielding the asserted congruence
modulo $W_{\ibf i}$. If $i_l<i_{l+1}$ the interesting case is
$i_{l+1}'=i_l=:j\leq m$. Here the assertion immediately follows from
the calculation

\[  \beta_l(\bs{\ibf i}) = \bs{\ibf is_l}
 -(y-1)\sum_{k>j'}q^{\rho_k+\rho_j}
\bs{\ibf i(k)} ,\]

because $f(\ibf i(k))<f(\ibf i)$ for all $k>j'$.
\end{Pf}

\begin{rem} \label{Invarianz W_i}
Since $\gamma_l(\ntrts)\subseteq \rcJ$ for $1\leq
l <r$ the lemma implies that $\overline W_{\ibf i}+\rcJ$ is invariant
under $\rfalg$.
But, $W_{\ibf i}=\sum_{f(\ibf j)<f(\ibf i)}\overline W_{\ibf j}$ and thus,
$W_{\ibf i}+\rcJ$ must be invariant, as well.
\end{rem}


\begin{cor} \label{Wirkung Nachbartausch}
Let $\ibf j \in \nmind r$ and  $l \in \mg r$.
Then to each
$\ibf k \in \nmind r$ satisfying $f(\ibf k) < f(\ibf j)$ there is
$a_{\ibf j\ibf k}(s_l)$ in $\gr$ (not neccessarily nonzero)
such that the following equations hold
in $\abb$ for all $\ibf i\in \nmind r$:

\[ \qbidet{\lambda}{\ibf i}{\ibf j}\no\beta_l^{-1}=
h_{j_{l+1}j_l}\qbidet{\lambda}{\ibf i}{\ibf js_l}
\; +\; \sum_{f(\ibf k)<f(\ibf j)} 
a_{\ibf j\ibf k}(s_l)\qbidet{\lambda}{\ibf i}{\ibf k}
\;\; \mbox{ if } j_l > j_{l+1},\]
\[ \qbidet{\lambda}{\ibf i}{\ibf j}\no\beta_l=
yh_{j_{l+1}j_l}\qbidet{\lambda}{\ibf i}{\ibf js_l}
\; +\; \sum_{f(\ibf k)<f(\ibf j)} 
a_{\ibf j\ibf k}(s_l)\qbidet{\lambda}{\ibf i}{\ibf k}
\;\; \mbox{ if } j_l \leq  j_{l+1}.\]
\end{cor}

\begin{Pf}
By definition of bideterminants, this follows immediately from the lemma.
\end{Pf}

\begin{cor} \label{Operation Spaltenstabil}
Let $\ibf j \in \nmind r$ and  $w\in \symg{\lambda'}$.
Then there is an invertible element $a_{\ibf j}(w) \in\gr$ and
to each $\ibf k \in \nmind r$ satisfying $f(\ibf k)< f(\ibf j)$ another
element $a_{\ibf j\ibf k}(w)$ in $\gr$ 
such that the following equations hold
in $\abb$ for all $\ibf i\in \nmind r$:

\[
\qbidet{\lambda}{\ibf i}{\ibf j}=
a_{\ibf j}(w)\qbidet{\lambda}{\ibf i}{\ibf jw}
\; +\; \sum_{f(\ibf k)<f(\ibf j)} 
a_{\ibf j\ibf k}(w)\qbidet{\lambda}{\ibf i}{\ibf k}.
\]

\end{cor}

\begin{Pf}
We use induction on the length of $w$. 
If this is zero there is nothing to prove.
If not, we write
$w=w's_l$ where $w', s_l \in \symg{\lambda'}$ and $l(w')=l(w)-1$.
By induction hypothesis we have
\[
\qbidet{\lambda}{\ibf i}{\ibf j}=
a_{\ibf j}(w')\qbidet{\lambda}{\ibf i}{\ibf jw'}
\; +\; \sum_{f(\ibf k)<f(\ibf j)} a_{\ibf j\ibf k}(w')
\qbidet{\lambda}{\ibf i}{\ibf k}.
\]
But by Lemma
\ref{Formel in mSp} we have $\qbidet{\lambda}{\ibf i}{\ibf jw'}=
-\qbidet{\lambda}{\ibf i}{\ibf jw'}\no\beta_l^{-1}$ as well as
$\qbidet{\lambda}{\ibf i}{\ibf jw'}=
-\qbidet{\lambda}{\ibf i}{\ibf jw'}\no\beta_l$. Thus, the assertion follows
from the preceding corollary and the fact that
$f(\ibf j)=f(\ibf jw')$.
\end{Pf}

Setting
$h_{\ibf i}(w):=\prod h_{i_{w(k)}i_{w(j)}}$,
where the product runs over all pairs $1\leq j < k \leq r$ such that
$w(j)>w(k)$, the more explicit formula
$a_{\ibf j}(w)=(-1)^{l(w)}y^th_{\ibf j}(w)$ for the invertible coefficients
is easy to verify but will not be needed any more.
Here, $t$ is the number of factors $h_{ij}$ in $h_{\ibf j}(w)$ such that 
$i\geq j$. Now we turn to a more special 
situation  for the multi-index $\ibf i$.


\begin{lem} \label{Formel ibf i geordnet}
Let $\ibf i \in \nmind r$ satisfy $i_1\leq i_2 \leq \ldots \leq i_r$
and let $w\in \symg r$ be arbitrary. Then the following
congruence relation holds in $\ntrts$ modulo the $\gr$-submodule 
$W'_{\ibf i}=W_{\ibf i}+\rcJ$:

\[ \beta(w^{-1})(\bs{\ibf i})\equiv y^{l(w)}h_{\ibf i}(w)\bs{\ibf iw} .\]

\end{lem}

\begin{Pf}
We use induction on $r$, the case $r=1$ being trivial. 
For $r>1$ we embed $\symg{r-1}$ as the parabolic subgroup of $\symg r$
generated by $s_1, \ldots ,s_{r-2}$ which fix $r$. 
If $w\in \symg{r-1}$ there is nothing to prove by the induction
hypothesis. Otherwise, we write
$w=w's$ where $w'\in \symg{r-1}$ and $s:=s_{r-1}
s_{r-2}\ldots s_{j+1}s_j$ for an appropriate $j < r$. We may assume
$l(w)=l(w')+r-j$. By the induction hypothesis and remark
\ref{Invarianz W_i} we calculate

\[ \beta(w^{-1})(\bs{\ibf i})\equiv
\beta(s^{-1})(y^{l(w')}h_{\ibf i}(w')\bs{\ibf iw'})=
y^{l(w')}h_{\ibf i}(w')\beta_j\beta_{j+1} \ldots \beta_{r-1}(\bs{\ibf iw'})
\equiv \]
\[ y^{l(w')}y^{r-j}h_{\ibf i}(w')h_{i_{w'(r)}i_{w'(r-1)}}
h_{i_{w'(r)}i_{w'(r-2)}}\ldots h_{i_{w'(r)}i_{w'(j)}} \bs{\ibf iw's}=
y^{l(w)}h_{\ibf i}(w)\bs{\ibf iw}.\]

Here, Lemma \ref{Formel W_i} has to be applied $r-j$ times and we have
used the assumption on $\ibf i$
implying $i_{w'(j)}, i_{w'(j+1)}, \ldots ,i_{w'(r-1)} \leq i_r=i_{w'(r)}$.
\end{Pf}



\begin{cor} \label{Garnir Vorbereit}
Let $\ibf j\in \nmind r$ satisfy $j_l \leq j_{l+1} 
\leq \ldots \leq j_{k-1} \leq j_k$ for some
$1\leq l < k \leq r$ and $w \in \symg r$
satisfy $w(i)=i$ for $1\leq i \leq l$ and $k< i \leq r$. Then, to each
$\ibf k \in \nmind r$ satisfying $f(\ibf k) < f(\ibf j)$ there is an element
$a'_{\ibf j\ibf k}(w)$ in $\gr$ such that the following equations hold in
$\abb$ for all
$\ibf i\in \nmind r$:
\[
\qbidet{\lambda}{\ibf i}{\ibf j}\no\beta(w)=
y^{l(w)}h_{\ibf j}(w^{-1})\qbidet{\lambda}{\ibf i}{\ibf jw^{-1}}
\; +\; \sum_{f(\ibf k)<f(\ibf j)} 
a'_{\ibf j\ibf k}(w)\qbidet{\lambda}{\ibf i}{\ibf k}.\]
\end{cor}

\begin{Pf}
As for the proof of Corollary \ref{Wirkung Nachbartausch}, this follows
easily from the preceeding lemma together with Lemma \ref{I_{cqT}^r}
by the definition of bideterminants.
\end{Pf}




















\section{The Weak Straightening Algorithm}

We are now able to give the proof of the following weak form of the straightening
algorithm. 

\begin{prop}[Weak Quantum Symplectic Straightening Algorithm]
\label{Straightening Algorithmus 1} 
Let \\
$\lambda \in \rprt r$ be a partition of $r$
and $\ibf j \in \nmind r \ohne \asmind{\lambda}$. Then to each 
$\ibf k\in \nmind r$ satisfying 
$\ibf k\lhd \ibf j$ there is an element
$a_{\ibf j\ibf k} \in \gr$ such that
in $\abb $ the following congruence relation holds
 for all $\ibf i\in \nmind r$:
\[
\qbidet{\lambda}{\ibf i}{\ibf j} 
\equiv \sum_{\ibf k\lhd\ibf j} a_{\ibf j\ibf k}
\qbidet{\lambda}{\ibf i}{\ibf k} \; \; \mbox{ mod } \; \; \abb (>\lambda) .
\]
\end{prop}


\begin{Pf}
We divide into the following two cases

\begin{enumerate}
\item $\ibf j$ is not $\lambda$-column standard.
\item $\ibf j$ is $\lambda$-column standard but not $\lambda$-row standard.
\end{enumerate}

{\em Case 1:} \Ab

By assumption there are two consecutive indices $j_l$ and $j_{l+1}$ in
$\ibf j=(j_1, \ldots , j_r)$ such that $j_l\succeq  j_{l+1}$ and 
$s_l=(l, l+1)\in
\symg{\lambda '}$. If
$j_l=j_{l+1}$, we have $\qbidet{\lambda}{\ibf i}{\ibf j}=0$
by Corollary \ref{qdet Null}, implying our assertion.
In the case
$j_{l+1} \succ j_{l}$ we apply Corollary \ref{Operation Spaltenstabil}:

\[\qbidet{\lambda}{\ibf i}{\ibf j}
=a_{\ibf j}(s_l)\qbidet{\lambda}{\ibf i}{\ibf js_l}
\; +\; \sum a_{\ibf j\ibf k}(s_l)\qbidet{\lambda}{\ibf i}{\ibf k}.\]

The multi-indices $\ibf k$ 
in the sum satisfy $f(\ibf k)<f(\ibf j)$ and consequently
$\ibf k \lhd \ibf j$. Finally, since
$f(\ibf j)=f(\ibf js_l)$ and $\ibf js_l$ occurs before $\ibf j$ in the
lexicographic order on $\nmind r$ we have $\ibf js_l\lhd \ibf j$ as well.\Ab


{\em Case 2:}\Ab

In principle we follow the lines of the proof of \cite[2.5.7]{martin},
using our tools developed in the previous section to substitute corresponding
classical results. However, two additional difficulties lead to some
complications. Firstly, the lack of commutativity inside $\qsAn{\gr}$ forces
us to work with a fixed basic tableau. But, the change of 
basic tableaux in the above mentioned proof can be compensated for by 
Lemma \ref{praebidet}. Secondly, our tools from the previous section
rather prefer the usual order $<$ on $\mg n$ instead of $\prec$, which is the
right one for the combinatoricis of tableaux.\Ab 


To start, let $l\in \mg r$ be the
smallest index such that $j_l$ is larger than its right hand neighbour
$j_{l'}$ in the  $\lambda$-tableaux of $\ibf j$.
Assume that the entry $j_l$ lies in the $s$-th column $\libf js{\lambda}$
and that $j_{l'}$ lies in the $s+1$-th column $\libf j{s+1}{\lambda}$,
where $1 \leq s < \lambda_1$. Clearly, $l'=l+\lambda'_s$.
Let $t$ be the index of the row containing both entries. We picture this
by

\[ \itab{\lambda} j = \ldots \begin{array}{|c|c|c}
\multicolumn{1}{c}{\libf js{\lambda}}  & 
\multicolumn{1}{c}{\libf j{s+1}{\lambda}} & 
\multicolumn{1}{c}{}\\
\vdots   & \vdots &\\
j_{l-1} &  j_{l'-1} & t-1\\\cline{1-1}
\multicolumn{1}{|c}{\rule[-3mm]{0mm}{7mm} j_{l}} &  
\multicolumn{1}{c}{j_{l'}} & 
\multicolumn{1}{|c}{t}\\ \cline{2-2}
j_{l+1} &    j_{l'+1} & t+1 \\
\vdots &  \vdots
\end{array} \ldots .\]

By assumption we have $\ldots \prec j_{l'-1}\prec j_{l'} \prec j_l \prec
j_{l+1} \prec \ldots $. Now, we refine the dual partition 
 $\lambda'$ of $\lambda$ to a 
composition $\eta \in \komp{p+2}r$, where $p:=\lambda_1$
is the number of columns of the diagram of $\lambda$. More
precisely, we split the
$s$-th and $(s+1)$-th column in front of and below the $t$-th row, explicitly


\[ \eta_i:=\left\{ \begin{array}{ll}
\lambda'_i & i < s \\
t-1 & i=s \\
\lambda'_s-t+1 & i=s+1 \\
t & i=s+2 \\
\lambda'_{s+1}-t & i=s+3 \\
\lambda'_{i-2} & i > s+3
\end{array} \right. , \; \; \; \; \;
\mu _i:=\left\{ \begin{array}{ll}
\eta_i & i \leq  s \\
\eta_{s+1}+\eta_{s+2} & i=s+1 \\
\eta_{i+1} & i > s+2
\end{array} \right. .
\]

Obviously, this $\eta$ is the coarsest refinement of the partition $\lambda'$
and the composition
$\mu \in \komp{p+1}r$ defined above. Let us split the multi-index $\ibf j$
according to $\eta$ as follows:

\[\libf js{\eta}=(j_h, \ldots , j_{l-1}), \;\;\;
\libf j{s+1}{\eta}=(j_l, \ldots , j_{h+k-1}), \]
\[
\libf j{s+2}{\eta}=(j_{h+k}, \ldots , j_{l'}), \;\;\;
\libf j{s+3}{\eta}=(j_{l'+1}, \ldots , j_{h+k+k'-1}). \]

Here, $h:=l-t+1=\lambda'_1+\ldots +\lambda'_{s-1}+1$ is the index of the
first entry of the $s$-th column  and $k:=\lambda'_s$
(resp.\ $k':=\lambda'_{s+1}$) are the lengths of both columns in question.
We have

\[ \libf js{\lambda}=\libf js{\eta} +\libf j{s+1}{\eta}, \; \; \;
 \libf j{s+1}{\lambda}=\libf j{s+2}{\eta} +\libf j{s+3}{\eta} \; \;
\mbox{ and set } \; \;
\libf j{s+1}\mu :=\libf j{s+1}{\eta} +\libf j{s+2}{\eta}. \]


In order to apply {\em Laplace Duality}  \ref{Laplace 1} to the pair
$(\lambda�, \mu)$ of compositions
we have to choose coset representatives of
$\symg{\eta}=\symg{\lambda'}\cap \symg\mu $ in
$\symg{\lambda'}$ and $\symg\mu $ carefully. From the theory of
parabolic subgroups of reflection groups it is known that each right coset
of $\symg{\eta}$ in
$\symg\mu \cong \symg{\mu _1}\times \ldots \times \symg{\mu _{p+1}}$
contains a unique element of minimal length called the distinguished
right coset representative, in fact one looks for coset representatives of
$\symg{\eta_{s+1}}\times \symg{\eta_{s+2}}$ in
$\symg{\mu _{s+1}}$. We choose these for our set $X$. The property
$l(wu)=l(w)+l(u)$ for $u\in X$ and
$w \in \symg{\eta}$ follows from
that theory as well. Similarily, one finds a set $Y$ of distinguished
left coset representatives of $\symg{\eta}$ in
$\symg{\lambda'}$ satisfying $l(vw)=l(v)+l(w)$ for $w \in
\symg{\eta}$ and $v \in Y$.\Ab

We will not apply Laplace Duality to the original index pair
$\ibf i, \ibf j$. This is the trick to handle the transition from the
order $<$  to $\prec$, mentioned above. Instead of $\ibf j$
we consider rather $\ibf{j'}:=\ibf jw$ where
$w\in \symg{\mu _{s+1}}\subseteq\symg{r}$ is chosen in such a way
that $j'_l<j'_{l+1}<\ldots < j'_{l'-1} < j'_{l'}$ and $j'_i=j_i$ for
$1\leq i <l$ and $l'<i\leq r$ (the embedding of $\symg{\mu_{s+1}}$ is
understood according to the composition $\mu$). This $w$ exists uniquely
since $\libf j{s+1}\mu =(j_l, \ldots , j_{l'})$ contains exactly
$\mu_{s+1}=\lambda_s'+1$ elements by the assumption 
$j_{h+k}\prec j_{h-k+1} \prec
\ldots \prec j_{l'}\prec j_l \prec \ldots \prec j_{h+k-1}$ on $\ibf j$.
Now, by Laplace-Duality we obtain

\begin{equation}\label{lapl}
\sum_{u \in X}
(-y)^{-l(u)}
\qbidet{\lambda}{\ibf i}{\ibf{j'}}\no\beta(u)=
\sum_{v \in Y}(-y)^{-l(v)}\beta(v)\no\pqbidet\mu {\ibf i}{\ibf{j'}}.
\end{equation}


With help of Lemma \ref{praebidet} the right hand side of this equation
can be written as a linear combination of
bideterminants $\qbidet{\mu '}{\ibf k}{\ibf l}$. Thus the right hand side
is seen to lie in $\abb (>\lambda)$ as soon we have shown that
$\mu '>\lambda$. But, this follows since the longest column
being removed from the diagram of $\lambda$ to obtain the diagram of $\mu'$
has length $\lambda_s'$, whereas a column of length $\mu_{s+1}=\lambda_s'+1$
has to be added to the diagram of $\mu '$. On the left hand side of 
(\ref{lapl}) we may apply Corollary \ref{Garnir Vorbereit} by construction
of the multi-index $\ibf{j'}$:
\[
(-y)^{-l(u)}
\qbidet{\lambda}{\ibf i}{\ibf{j'}}\no\beta(u)=\sign u
h_{\ibf{j'}}(u^{-1})\qbidet{\lambda}{\ibf i}{\ibf{j'}u^{-1}}
\; +\; \sum (-y)^{-l(u)}a'_{\ibf{j'}\ibf k}(u)\qbidet{\lambda}{\ibf i}{\ibf
  k},
\]

the sum running over all $\ibf k$ satisfying $f(\ibf k)<f(\ibf j')=f(\ibf j)$.
Now, for all $u \in X$ we have $\tilde u:=uw^{-1}\in \symg{\mu _{s+1}}$ since
$w$ is contained in $\symg{\mu _{s+1}}$. Furthermore, there is a unique
coset representative $u_0\in X$ satisfying
$\symg{\eta}u_0=\symg{\eta}w$ and this is the only one for which the corresponding
$\tilde{u}$ lies in $\symg{\eta}$. Therefore, in the case
$u\neq u_0$ there is an $e$ such that $l\leq e<h+k$ and
$h+k\leq \tilde u^{-1}(e)\leq l'$. 
Choose such an $e$ for each $u \in X$. In doing so, we are assigning a
transposition $\hat{u}:=(l,e)$ to each $u$ which is contained
in $\symg{\eta}$. In the case of $u_0$ we set
$\hat{u_0}:=\tilde{u_0}\in \symg{\eta}$.
Applying Corollary \ref{Operation Spaltenstabil} to $\hat u$ one calculates

\[
\qbidet{\lambda}{\ibf i}{\ibf{j'}u^{-1}}=
\qbidet{\lambda}{\ibf i}{\ibf j\tilde u^{-1}}=a_{\ibf j\tilde u^{-1}}(\hat{u})
\qbidet{\lambda}{\ibf i}{\ibf j\tilde u^{-1}\hat{u}}
+\sum a_{(\ibf j\tilde u^{-1})\ibf k}(\hat{u})
\qbidet{\lambda}{\ibf i}{\ibf k}, \]

where the sum is  running over all  $\ibf k$ satisfying 
$f(\ibf k)< f(\ibf j\tilde u^{-1})=
f(\ibf j)$, again. For these $\ibf k$ we set

\[ \bar a_{\ibf j\ibf k}:=
\sum_{u\in X} (-y)^{-l(u)}a'_{\ibf{j'}\ibf k}(u)+
\sign u h_{\ibf{j'}}(u^{-1})a_{(\ibf j\tilde u^{-1})\ibf k}(\hat{u}), \]

whereas in the case  $f(\ibf k)=f(\ibf j)$ we write

\[ \bar a_{\ibf j\ibf k}:=\left\{ \begin{array}{cl}
\sign u h_{\ibf{j'}}(u^{-1})a_{\ibf j\tilde u^{-1}}(\hat{u}) & \mbox{ if there
  exist }\;  u \in X,\;
\ibf k=\ibf j\tilde u^{-1}\hat{u}\\
0 & \mbox{ otherwise }
\end{array} .
\right. \]

Observe, that $\bar a_{\ibf j\ibf j}$ occurs in this definition for $u=u_0$.
We assert that for $u\neq u_0$ the multi-index
$\ibf k:=\ibf j\tilde u^{-1}\hat{u}$ occurs before  $\ibf j$ in the 
lexicographic order with respect to $\prec$.
For by construction of $\tilde u$ and $\hat{u}$ we have

\[ k_l=j_{\tilde u^{-1}\hat{u}(l)}=j_{\tilde u^{-1}(t)} \in
\{j_{h+k}, j_{h+k+1} ,\ldots , j_{l'} \}\]

and consequently $k_l \prec j_l$. But this implies $\ibf k \prec \ibf j$,
since $k_i=j_i$ for $i<l$. Thus we obtain

\[
-\bar a_{\ibf j\ibf j}\qbidet{\lambda}{\ibf i}{\ibf j} 
\equiv \sum_{\ibf k\lhd \ibf j} \bar a_{\ibf j\ibf k}
\qbidet{\lambda}{\ibf i}{\ibf k} \; \; \mbox{ mod } \; \; \abb (>\lambda).
\]

Since the coefficient $-\bar a_{\ibf j\ibf j}$ is invertible,
the asserted congruence relation holds as well.
\end{Pf}


It should be remarked, that the proof works with any other
order on $\mg n$ instead of $\prec$, as well. The proof of the
strong part of the algorithm (Proposition \ref{Straightening Algorithmus 2})
can be given right now in the (initial)
case $r=2$ and we are going to do this not only because it is
very instructive, but also because we will need a basis of $\qrsAn{\gr} 2$
in order to proceed to the general case.\Ab

If $r=2$ there are exactly two partitions in $\prt m2$ for $m\geq 2$,
namely $2\omega_1$ and $\omega_2$. In the first case we have
$\asmind{2\omega_1}=\symmind{2\omega_1}$, that is, the weak and the strong
form of the straightening algorithm coincide. Turning to $\omega_2$ there is exactly
one element in $\asmind{\omega_2}\ohne \symmind{\omega_2}$,
namely $\ibf j=(m,m')$.
By (\ref{g und qdet}) we obtain in $\abb =\qrshAn{\gr } 2$

\[
\qbidet{\omega_2}{\ibf i}{(m,m')}=
-q^m\sum_{i=1}^{m-1}q^{-i}\qbidet{\omega_2}{\ibf i}{(i,i')}
\]

yielding Proposition \ref{Straightening Algorithmus 2} in the case $r=2$
since $(i,i')\lhd (m, m')$ for all $i <m$.

\begin{rem} If we had used the notion of symplectic standard tableau
instead of the reverse version we would have to consider
$(1',1)$ instead of $(m,m')$ in the last step above. This would force us
to work with a reverse version of the order choosen on $\N_0^m$
(as in \cite[section 7]{ofrt}). But this would cause some trouble
concerning Lemma \ref{Formel W_i}. One way out could be a manipulation of
the Yang-Baxter operator $\beta$ conjugating it by the twofold tensorproduct
of the appropriate permutation on $\mg n$. Thus, one has to decide between
working with the familiar version of $\beta$ or following the familiar
notion of tableaux. 
\end{rem}























\section{Quantum Symplectic Exterior Algebra} \label{aussere Algebra}



We are going to prepare the proof of Proposition \ref{Straightening 
Algorithmus 2} for general $r$. Since we need
a $q$-analogue of \cite[Lemma 8.1]{ofrt}, we have to investigate
the quantum symplectic exterior algebra.
We start with its definition 
which can be found in many textbooks on 
quantum groups (for instance \cite[chapter 7]{charipr}). 
It is defined as the quotient of the
tensor algebra $\tens{\nt}=\bigoplus_{r\in \N_0}\ntrts$ by a certain ideal.
We denote it by $\aA$ and write the symbol $\wedge$ for multiplication
in this algebra. Setting

\[ c_i:=q^{i}\bs{i'} \wedge \bs i,\;\;\mbox{ and } \;\;
d_i:=-q^{-i}\bs{i} \wedge \bs{i'}  \]

for $i \in \mg m$, the relations holding in  $\aA$ can be written as


\begin{eqnarray} \label{Rel 1}
\bs k \wedge \bs l & = & -q^{-1} \bs l \wedge \bs k, \\ \label{Rel 2}
\bs i \wedge \bs{i'} & = &
-q^{-2}\bs{i'}\wedge \bs i -(y-1)q^{-i}\sum_{j=i+1}^m c_j 
\\ \label{Rel 3}
\bs{i'} \wedge \bs{i} & = &
-q^{2}\bs{i}\wedge \bs{i'} +(y-1)q^{i} \sum_{j=i+1}^m d_j
\\ \label{Rel 4}
\bs k\wedge \bs k & = & 0
\end{eqnarray}


where $k>l, \; k\neq l'$ and $i\in \mg m$ is assumed. Note that the third
relation is a consequence of the former two, and clearly the remaining
three types of relations imply that the set

\[ B:=\{ \bs I|\; I\subseteq \mg n \} \]

is a set of $\gr$ -linear generators of $\aA$ where we have set

\[ \bs I:=\bs{i_1} \wedge \bs{i_2} \wedge \ldots \wedge \bs{i_r} ,\;\;
\mbox{ if } \;\; I:=\{i_1, \ldots, i_r\} \;\;\mbox{ and }\;\;
i_1 < i_2 < \ldots <i_r.\] 

In contrast to \cite[section 7]{ofrt} we take the usual
order $<$ on $\mg n$ here for technical reasons. A subset $I\subseteq \mg n$ ordered
in that way will be called an {\em ordered subset} in the sequel.
Using the diamond lemma of ring theory one shows that $B$ in fact is
an $\gr$-basis of $\aA$ (cf. \cite[p. 157]{hay1}, 
\cite[Satz 3.8.3, B.1.1]{doc}). 
Furthermore, $\aA$ is a graded algebra since the relations are
homogeneous of degree two. A basis for the $r$-th homogeneous summand
$\raA r$ is given by the subset $B_r$ of $B$ corresponding to the
set $\tmg nr$ of subsets $I\subseteq \mg n$ having cardinality $|I|=r$.\Ab

\newcommand{\cgf}{Z}

Before proceeding, let us briefly argue why $\aA$ possesses the
structure of an $\qsBn{\gr}$-comodule algebra (for details we refer to
\cite[Satz 3.8.2]{doc}). First, the reader should convince himself that
we may reduce to the case $\gr=\cgr:=\Z[q,q^{-1}]$, the ring of integral
Laurent polynomials in the indeterminate $q$ (use \cite[Theorem 4.1]{ofrt}).
In this case it is easy to
see that the defining equations for $\aA$ precisely span the kernel
of the endomorphism $\beta -q^2 \id{\ntts 2}$ (consider the eigenspace
decomposition of $\beta$ in the case of the field of fractions $\K:=\Q(q)$ of 
$\cgr$). Therefore, it is enough to show that this kernel is a
$\qsBn{\gr}$-subcomodule
of $\ntts 2$, since this property is transmitted to the spanned ideal.
Now, if $C:=\End{\gfalg 2}{\ntts 2}$ is the centralizer algebra 
of the algebra $\gfalg 2$ spanned by $\beta$ and $\gamma$,
this kernel clearly is a $C$-submodule of $\ntts 2$. Since $C$ is the
dual algebra to the coalgebra $\qrsBn{\gr} 2$ 
(\cite[Theorem 3.3]{ofrt}), the reader
might think that this completes the argument. The following counterexample shows
that some caution is needed at this point:
Consider the $\Z$-module on abstract generators $\xes ij$:

\[ M:=\Z \xes 11 \oplus \Z \xes 22 \oplus (\Z/2\Z) \xes 21\oplus 
(\Z/2\Z) \xes 12. \]

It can be considered as a $\Z$-coalgebra with comultiplication and augmentation
beeing defined in the usual way on the generators. In fact it is the
centralizer coalgebra (as defined in \cite[section 2]{ofrt})
of the endomorphism $\mu$ on
$\Z^2$ given by the matrix $\left(\begin{array}{cc} 1 &0\\ 0 &3
\end{array}\right)$. The centralizer algebra of $\mu$  clearly is
given by

\[ C:=\End{\mu}{\Z^2}=\left\{\left(\begin{array}{cc} k & 0\\ 0 & l\end{array}\right)|\;
k,l \in \Z\right\}.\]

Thus, the span of the element $\bs 1=(1,0)$ is a $C$-submodule of $\Z^2$.
On the other hand $\cmd{V}(\bs 1)=\bs 1 \otimes \xes 11+\bs 2\otimes \xes 21$
is not contained in $\bs 1\otimes M$, since $\xes 21 \neq 0$ and therefore
this span is not an $M$-subcomodule of $\Z^2$.\Ab

However, it is easy to see that such things cannot happen if the
coalgebra is free as an $\gr$-module (\cite[Lemma 1.5.4]{doc}),
which was not the case in the counterexample. To finish our argument
that $\aA$ is an $\qsBn{\gr}$-comodule algebra, we have to show
that the coalgebra $\qrsBn{\cgr} 2$ is a free $\cgr$-module.
But we already have seen at the end of the previous section that the
straightening formula is valid in the case $r=2$ and consequently
$\B_2$ is a set of generators for $\qrsBn{\gr} 2$. Now, over the field
of fractions $\K$ one calculates $|\B_2|=({ n \choose 2} +n)^2 +
({ n\choose 2}-1)^2 + 1^2=\Dim{\K}
C=\Dim{\K}{\qrsAn{\K} 2}$ as the square sum of eigenspace dimensions
of the endomorphism $\beta$. Since $\qrsBn{\K} 2\cong \K \otimes_{\cgr}
\qrsBn{\cgr} 2$ by \cite[Theorem 4.1]{ofrt}, the set $\B_2$ must be linearly
independent as well. Thus our argument that $\aA$ is an $\qsBn{\gr}$-comodule
algebra is finished.\Ab

One might guess that the coefficient functions of $\raA r$ corresponding to the
basis $B_r$ consist of bideterminants as in the classical case.
In fact this is true and we are going to prove this below.
Denoting the structure map by
$\cmd{\wedge}:\aA\pfr \aA\otimes \qsBn{\gr}$ (we will use the same symbol
in the case of $\qshAn{\gr}$ later on) the explicit formula is

\begin{equation}\label{aA comodul}
 \cmd{\wedge}(\bs J)=\sum_{I \in \tmg nr}\bs I\otimes\qbidet{\omega_r}{\ibf
i}{\ibf j}
\end{equation}

where $\ibf i=(i_1, \ldots , i_r)$ and $\ibf j=(j_1, \ldots , j_r)$ 
are the multi-indices corresponding to the
ordered subsets $I:=\{i_1, \ldots , i_r\}$ and $J=\{j_1, \ldots , j_r\}$, 
respectively. Let us first treat the ingredients needed in the proof
of that formula.


\begin{lem} \label{kappa faktorisiert}
Let $\pi_r:\ntrts \pfr \raA r$ be the natural projection. Then the endomorphism
$\kappa_{\lambda}:=\sum_{w \in \symg{\lambda}} (-y)^{-l(w)}\beta(w)$
factors through $\pi_r$, i.e.\ there is a homomorphism of 
$\gr$-modules $\nu_r:\raA r \pfr \ntrts$ such that $\kappa_r=\nu_r\circ \pi_r$.
\end{lem}

\begin{Pf}
Again, we may assume $\gr =\cgr=\Z[q,q^{-1}]$. In this case we know that the
defining ideal of $\aA$ is generated by the kernel of 
$(\id{\ntts 2}-y^{-1}\beta)$. Thus, the assertion immediately follows
from Lemma \ref{1-ybetai Formel}.
\end{Pf}

Let
$\smind{\omega_r}:=\{\ibf i \in \nmind r |\; i_1<i_2< \ldots < i_r \}$ be the
set of multi-indices corresponding to the ordered subsets $I\in \tmg nr$.

\begin{lem} \label{Initialindizes}
Let $F_r$ be the $\gr$-linear span of
$\{\bs{\ibf j}|\; \ibf j \in \nmind r \ohne \smind{\omega_r}\}$
in $\ntrts$. Then for
all $w \in \symg r\ohne \{\id{}\}$ and $\ibf i \in \smind{\omega_r}$ it follows
that $\beta (w)(\bs{\ibf i}) \in F_r$.

\end{lem}

\begin{Pf}
We use induction on $r$. The case $r=2$ directly follows
from the formulas

\begin{eqnarray}\label{beta auf f^< kl}
\beta (\bs{(k,l)}) & = & q\bs{(l,k)} \\ \label{beta auf f^< i}
\beta (\bs{(i,i')}) & = & \bs{(i',i)} +
(y-1)\sum_{j=1}^{i-1}q^{j-i}\bs{(j',j)}
\end{eqnarray}

which are valid for $k<l,\; k\neq l'$ and $i \leq m$.
If $r>2$, we embed $\symg{r-1}$ as the subgroup of $\symg r$ that
fixes the letter $r$ (cf. proof of Lemma \ref{Formel ibf i geordnet}).
If $w \in \symg{r-1}$, there is nothing to prove since
$F_{r-1}\otimes \nt \subseteq F_r$. Otherwise, we may write
$\beta (w)=\beta (w')\beta_{r-1}\beta_{r-2} \ldots \beta_l$ where $w' \in
\symg{r-1}$ and $l \leq r-1$.\Ab

First consider the case where
$i_l'$ is not contained in $\{i_{l+1}, \ldots , i_r\}$. 
Applying $\beta_{r-1}\beta_{r-2} \ldots \beta_l$
to $\bs{\ibf i}$ we only have to use (\ref{beta auf f^< kl}) but not
(\ref{beta auf f^< i}). Consequently, we have $\beta_{r-1}\beta_{r-2} \ldots \beta_l(\bs{\ibf i})=
q^{r-l+1}\bs{i_1}\ldots \hat{\bs{i_l}}\ldots \bs{i_r}\bs{i_l}$. 
Here, 
$\hat{\bs{i_l}}$ denotes the omission of $\bs{i_l}$. This element obviously
lies in $F_r$, proving the assertion in the case $w'=\id{}$.
If $w'$ is not the identity map we have
$\beta (w')(q^{r-l+1}\bs{i_1}\ldots \hat{\bs{i_l}}\ldots \bs{i_r}\bs{i_l})\in
F_{r-1}\otimes \bs{i_l}\subseteq F_r$ by the induction hypothesis since
$(i_1,\ldots , \hat{i_l},\ldots ,i_r) \in 
\smind{\omega_{r-1}}$.\Ab

We next consider the case $i_l' \in \{i_{l+1}, \ldots , i_r\}$. 
This forces $i_l\leq m$ because $i_l <i_l'$. Let $i_l'=i_k$. 
As above, we have
$\beta_{k-2}\beta_{k-3} \ldots \beta_l(\bs{\ibf i})=q^{k-l-2}
\bs{i_1}\ldots \hat{\bs{i_l}}\ldots \bs{i_{k-1}}
\bs{i_l}\bs{i_k}\bs{i_{k+1}}\ldots
\bs{i_r}$. Applying $\beta_{k-1}$ to this expression, we have to use
(\ref{beta auf f^< i}) for the first time. But  for each 
basis element $\bs{\ibf j}$ occurring as a summand in the resulting
expression we have $j_k\leq i_l\leq m$.
Similar things happen concerning the remaining 
$\beta_k, \ldots , \beta_{r-1}$. Thus, for each $\bs{\ibf j}$ occurring
as a summand in
$\beta_{r-1}\beta_{r-2} \ldots \beta_l(\bs{\ibf i})$,
it follows $j_r\leq i_l \leq m$. On the other hand, for each such
summand there must exist an  $h < r$ where $j_h >m$. This is because $\ibf j$
must contain a pair $\{i,i'\}$ for some $i \in \mg m$, since this was the
case for the multi-index $\ibf i$ we started with. Consequently,
we obtain $\ibf j \in F_r$ in this case too.
\end{Pf}

Let the coefficient matrices of the $\gr$-module homomorphisms 
$\pi_r,\;\kappa_r$ and $\nu_r$ (from Lemma \ref{kappa faktorisiert})
be given by

\[ \pi_r(\bs{\ibf j})=\sum_{I \in \tmg nr}\pi_{I\ibf j}\bs I,\;\;
\kappa_r(\bs{\ibf j})=\sum_{\ibf i \in \nmind r}
\kappa_{\ibf i\ibf j}\bs{\ibf i}\;\; \mbox{ and }\;\;
\nu_r(\bs{J})=\sum_{\ibf i \in \nmind r}
\nu_{\ibf i J}\bs{\ibf i}.\]

Now, if $\ibf j\in \smind{\omega_r}$ corresponds to the 
ordered set $J \in \tmg nr$
we have $\pi_r(\bs{\ibf j})=\bs J$ yielding $\nu_{\ibf iJ}=\kappa_{\ibf i
\ibf j}$ by Lemma \ref{kappa faktorisiert}. From Lemma 
\ref{Initialindizes} it follows $\kappa_r(\bs{\ibf j})
\equiv \bs{\ibf j}$ modulo $F_r$. Thus, for a pair $\ibf i,\ibf j\in 
\smind{\omega_r}$ of multi-indices corresponding to ordered sets 
$I,J\in \tmg nr$, we obtain $\nu_{\ibf iJ}=\kappa_{\ibf i\ibf j}=
\delta_{IJ}$ (Kronecker symbol). Finally, from  $\kappa_r=\nu_r\circ \pi_r$
we see for all $\ibf i\in \smind{\omega_r}$ and $\ibf j \in \nmind r$

\[ \kappa_{\ibf i\ibf j}=\sum_{K\in \tmg nr}\nu_{\ibf iK}\pi_{K\ibf j}=
\pi_{I\ibf j}. \]

We are now ready to give the proof of equation (\ref{aA comodul}).
We calculate

\[\cmd{\wedge}(\bs J)=\sum_{\ibf k \in \nmind r}
\bs{k_1}\wedge \ldots \wedge
\bs{k_r} \otimes \xes{\ibf k}{\ibf j}=\]
\[\sum_{\ibf k \in \nmind r} \sum_{I \in \tmg nr}\pi_{I\ibf k}\bs I
\otimes \xes{\ibf k}{\ibf j}=
\sum_{I \in \tmg nr} \bs I \otimes
\sum_{\ibf k \in \nmind r}\kappa_{\ibf i\ibf k}\xes{\ibf k}{\ibf j}. \]

But, this is exactly what we wanted
by 
the definition $\qbidet{\omega_r}{\ibf i}{\ibf j}=
\kappa_r\no\xes{\ibf i}{\ibf j}$ of bideterminants (using notation 
(\ref{Notation End})).\Ab

The formula we just have proved has some useful consequences
concerning the comultiplication and augmentation of $A:=\qsBn{\gr}$. These
are valid for any pair $\ibf i,\ibf j\in \smind{\omega_{r}}$
of multi-indices corresponding to ordered sets $I,J \in \tmg nr$ and 
follow directly with the help of the comodule axioms $(\cmd{\wedge}\otimes
\id A)\circ \cmd{\wedge}=(\id{\wedge} \otimes \kom)\circ \cmd{\wedge}$
and $(\id{\wedge}\otimes\koe)\circ \cmd{\wedge} =\id{\wedge}$:

\begin{equation}\label{bidet kom}
 \kom (\qbidet{\omega_r}{\ibf i}{\ibf j})=\sum_{\ibf k\in \smind{\omega_r}}
\qbidet{\omega_r}{\ibf i}{\ibf k}\otimes\qbidet{\omega_r}{\ibf k}{\ibf j},
\end{equation}

\begin{equation}\label{bidet koe}
\koe (\qbidet{\omega_r}{\ibf i}{\ibf j})=
\delta_{\ibf i\ibf j}.
\end{equation}

Another useful consequence is the following corollary:

\begin{cor}\label{wedge-Regel}
Let $a_{\ibf j} \in \gr$ be such that $\sum_{\ibf j\in \mind nr}a_{\ibf j} 
v_{j_1} \wedge v_{j_2}\wedge \ldots \wedge v_{j_r} = 0 $. Then for all 
$\ibf i \in \mind nr$ we have
\[ \sum_{\ibf j \in \mind nr} a_{\ibf j}\qbidet{\omega_r}{\ibf i}{\ibf j}=
 \sum_{\ibf j \in \mind nr} a_{\ibf j}\qbidet{\omega_r}{\ibf j}{\ibf i}=0.\]
\end{cor}
\begin{Pf}
By Lemma \ref{kappa faktorisiert} and the assumption  we have
\[\kappa_r(\sum_{\ibf j\in \nmind r}a_{\ibf j}\bs{\ibf j})=
\sum_{\ibf j\in \nmind r}a_{\ibf j}\kappa_r(\bs{\ibf j})=0.\]
Consequently, for all $\ibf k \in \nmind r$ we obtain
$\sum_{\ibf j\in \mind nr}a_{\ibf j}\kappa_{\ibf k\ibf j}=0$
and therefore
\[ \sum_{\ibf j \in \mind nr} a_{\ibf j}\qbidet{\omega_r}{\ibf i}{\ibf j}=
\sum_{\ibf k,\ibf j \in \mind nr} a_{\ibf j}\xes{\ibf i}{\ibf k}
\kappa_{\ibf k\ibf j} =0 .\]
The equation with exchanged indices is deduced by an application of
the involution $^*$.
\end{Pf}













\section{Proof of Proposition \ref{Straightening Algorithmus 2}}


First, we have to state the $q$-analogue of \cite[Lemma 8.1]{ofrt},
one of the principal ingredients in the
proof of the symplectic straightening algorithm in the classical
case.
In order to define the quantum analogue to the ideal $N$ considered
there
we have to look in more detail at the elements $c_i$ and $d_i$ defined
in the previous section. Setting $y=q^2$ as before,
let us  write down the fundamental relations between them,

\begin{equation}\label{d_i und c_i}
y^{i}d_i=yc_i +(y-1)\sum_{j=i+1}^m c_j\;\;\; \mbox{ and } \;\;\;
y^{-i}c_i=y^{-1}d_i +(y^{-1}-1)\sum_{j=i+1}^m d_j.
\end{equation}

These imply $c_i\wedge d_i=d_i\wedge c_i=0$ 
according to (\ref{Rel 4}) and furthermore,

\begin{equation} \label{quadrat c_i}
c_i^2:=c_i \wedge c_i =(y^{-1}-1)\sum_{j=i+1}^m c_i\wedge c_j\;\;\;
\mbox{ and }\;\;\; d_i^2=(y-1)\sum_{j=i+1}^m d_i\wedge d_j.
\end{equation}

This stands in remarkable contrast to the classical and even
quantum linear case where such expressions vanish. On the other hand,
the elements $c_i$ and $d_i$ commute with each other, exactly as in the
classical case. Consequently, the elements
$d_K:=d_{k_1}\wedge d_{k_2} \wedge \ldots \wedge d_{k_a}$ are defined
independently on the order of the elements of the subset 
$K:=\{k_1, \ldots, k_a\}\subseteq\mg m$.
Again, we write $\tmg ma$ for the collection of all subsets $K$
of $\mg m$ whose cardinality is $a$. Set

\[ D_{a}:=\sum_{K\in \tmg ma} d_K \]


and let $N$ be the ideal in $\aA$ generated by the elements
$D_1, D_2, \ldots , D_m$. 
We call an ordered subset $I\in \tmg nr$ {\em
reverse  symplectic} if the multi-index $\ibf iw$ obtained from $I$
by ordering its elements according to $\prec$ (obtained from
$\ibf i$ by a suitable permutation $w\in \symg r$ such that $i_{(w(1)}\prec
i_{w(2)}\prec \ldots \prec i_{w(r)}$) is
$\omega_r$-reverse symplectic standard.

\begin{prop} \label{Rel nicht spiegelsymplektisch}
Let $I\in \tmg nr$ be non reverse symplectic. Then, to each $J\in \tmg nr$
such that the inequality $f(\ibf j)<f(\ibf i)$ holds
for the corresponding multi-indices $\ibf i$ and $\ibf j$, there exists 
$a_{IJ}\in \gr$ such that in $\aA$ the
following congruence relation holds:

\[ \bs I\equiv
\sum_{J\in \tmg nr, \; f(\ibf j)<f(\ibf i)}a_{IJ}\bs J  \;\; \mbox{
  mod }
\;\; N.\]

\end{prop}
 

\begin{prop} \label{Operation D_a trivial}
The semi bialgebra  $\qshAn{\gr}$ acts trivially on 
the elements $D_a$, that is $\cmd{\wedge}(D_a)=0$.
\end{prop}


We postpone the very technical proofs of 
both propositions to  separate sections below.\Ab

Let us prove Proposition \ref{Straightening Algorithmus 2} in
the case $\lambda=\omega_r$, first. Take 
$\ibf j\in \nmind r\ohne\symmind{\omega_r}$. Using the weak part of the
straightening algorithm \ref{Straightening Algorithmus 1}, we may
assume $\ibf j\in \asmind{\omega_r}\ohne\symmind{\omega_r}$.
This means $j_1\prec j_2\prec \ldots \prec j_r$. In order to apply
our lemmas
we have to change orders from $\prec$ to $<$. Let $w\in \symg r$
be such that $j_{w(1)}<j_{w(2)}<\ldots <j_{w(r)}$, that is, $\ibf jw$
is a multi-index corresponding to a non reverse 
symplectic ordered set
$J \in\tmg nr$ in the sense of Proposition \ref{Rel nicht
  spiegelsymplektisch}. Application of this proposition to $\bs{J}$ yields

\[ X:=\bs{J}-
\sum_{K\in \tmg nr, \; f(\ibf k)<f(\ibf j)}a_{\ibf j\ibf k}\bs K  \;\; \in
 N \]

since $f(\ibf j)=f(\ibf jw)$.
According to Proposition \ref{Operation D_a trivial}, the element 
$\cmd{\wedge}(X)$ must be zero.  Applying (\ref{aA comodul}),
we obtain the following equation holding in $\raA r\otimes \abb$:

\[ \sum_{I\in \tmg nr}\bs I\otimes
\left(\bidet{\omega_r}{\ibf i}{\ibf jw}-\sum_{K\in \tmg nr, \; \ibf k 
\lhd \ibf j}a_{\ibf j\ibf k}
\bidet{\omega_r}{\ibf i}{\ibf k}\right)=0.\]

Since $\{\bs I|\; I \in \tmg nr\}$ is a basis of $\raA r$, each
individual summand in the summation over $\tmg nr$ must be zero.
Together with Corollary  \ref{Operation Spaltenstabil},
this gives the desired result in the case of multi-indices $\ibf i$
corresponding to ordered subsets $I\in \tmg nr$, that is 
$\ibf i\in \smind{\omega_r}$. The case for general $\ibf i$ can be
deduced from this using 

\[ \qbidet{\omega_r}{\ibf i}{\ibf j} =\sum_{K\in \tmg nr}
\nu_{\ibf iK}\qbidet{\omega_r}{\ibf k}{\ibf j}, \]

which follows from the formula $\kappa_r=\nu_r\circ \pi_r$
we met in the previous section.\Ab

Next we conssider the general case of $\lambda$. Here, we can proceed
exactly as in the classical case.
Again, we may assume
$\ibf j\in \asmind{\lambda}\ohne\symmind{\lambda}$ by the weak part of the
straightening algorithm. Let $\lambda'=(\mu_1, \ldots ,
\mu_p)$  be the dual partition ($p=\lambda_1$). We spilt $\ibf j$
into $p$ multi-indices $\ibf j^l\in\nmind{\mu_l}$. where for
each $l\in \mg p$ the entries of $\ibf j^l$ are taken from the $l$-th
column of $\itab{\lambda} j$. The same thing can be done with
$\ibf i$. Since $\ibf j$ is not $\lambda$-reverse 
symplectic standard but standard
there must be a column $s$ such that $\ibf j^s$ is not
$\omega_{\mu_s}$-reverse symplectic standard. Applying the result to the
known case of $\qbidet{\omega_{\mu_s}}{\ibf i^s}{\ibf j^s}$, we obtain

\[ \qbidet{\lambda}{\ibf i}{\ibf j}=
\qbidet{\omega_{\mu_1}}{\ibf i^1}{\ibf j^1}
\qbidet{\omega_{\mu_2}}{\ibf i^2}{\ibf j^2} \cdots
\qbidet{\omega_{\mu_s}}{\ibf i^s}{\ibf j^s} \cdots
\qbidet{\omega_{\mu_p}}{\ibf i^p}{\ibf j^p}\]

\[\equiv \sum a_{\ibf j^s\ibf k^s}
\qbidet{\omega_{\mu_1}}{\ibf i^1}{\ibf j^1} \cdots
\qbidet{\omega_{\mu_s}}{\ibf i^s}{\ibf k^s} \cdots
\qbidet{\omega_{\mu_p}}{\ibf i^p}{\ibf j^p}\;\;
= \;\; \sum a_{\ibf j\ibf k}
 \qbidet{\lambda}{\ibf i}{\ibf k}.\]

Therein, $\ibf k^s\in \nmind{\mu_s}$ satisfies $\ibf k^s\lhd\ibf j^s$,
$\ibf k\in \nmind r$ is constructed from $\ibf j$ by replacing the
entries of $\ibf j^s$ by that of $\ibf k^s$ and $a_{\ibf j\ibf k}$ is
the same as $a_{\ibf j^s\ibf k^s}$ for the corresponding $\ibf k^s$.
The product formula for bideterminants applied above is valid by our choice
of basic $\lambda$-tableaux inserting the numbers $1, \ldots , r$
column by column top down (otherwise the non-commutativity of
$\qsAn{\gr}$ would cause some trouble). From (\ref{Additivitat}) we see
$\ibf k\lhd\ibf j$ and the proof of \ref{Straightening Algorithmus 2}
is completed.














\section{Proof of Proposition \ref{Rel nicht spiegelsymplektisch}}

For convenience, in in the following sections we abbreviate $y=q^2$ and
denote multiplication in $\aA$
by juxtaposition instead of $\wedge$.
Further, to sets $K, L \subseteq \mg m$ we define integers

\[v(K,L):=|\{(k,l)\in  K\times L |\; k>l\}|.\] 


\begin{lem} \label{Formeln c,d 1}
Let $a\in \mg m$. If $\mg{m}=L\cup M$ is a partition of $\mg{m}$ into disjoint
subsets $L$ and $M$ then to each $K \in \tmg ma$ there is an integer
$s(K,L)$ such that
\[ D_a=\sum_{K \in \tmg ma}y^{s(K,L)}c_{K\cap L}d_{K\cap M}. \]
If $K\subseteq M$, the integer $s(K, L)$ equals $v(K, L)$.
\end{lem}

\begin{PfS}
In order to prove this, one has to use a more general statement
in which the set $\mg m$ is substituted by $\{l, l+1, \ldots ,
m\}$ for some $l\in \mg m$. After this, the results can be proved
straightforwardly using induction on $m-l$ and the formulas
(\ref{d_i und c_i}) and (\ref{quadrat c_i}). For details we refer to
\cite[Lemma 3.9.1 and B.2]{doc}.
\end{PfS}

To a set $I \in \tmg nr$ we associate the followng subsets of $\mg m$:

\[I^-:=I\cap \mg m, \;\; I^+:=\{ i\in \mg m|\; i' \in I\}\;\;\mbox{ and }
\;\; I^0:=I^-\cap I^+. \]

\begin{lem}\label{Operation D_r}
Let $I \in \tmg nr$ be such that $I^0=\emptyset$ and $J \subseteq \mg m$. Set
$s.=|J|$ and
$T:=\mg m \ohne (I^-\cup I^+)$. If $J\subseteq T$ and $a\in \mg m$ is
such that $s<a$ then we have

\[ \bs Id_JD_{a-s}=\bs I
\sum_{S \in \tmg ma, J \subseteq S \subseteq T}y^{v(S\ohne J,I^+\cup J)}d_S.\]

\end{lem}

\begin{Pf}
We apply Lemma \ref{Formeln c,d 1} to $L:=I^+\cup J$ and
$M:=\mg m\ohne L$. Since $J \subseteq T$ there is an invertible element
$b \in \gr$ such that $\bs Id_J=b d_J\bs I$. Now, we see
that $\bs Id_Jc_{K\cap L}d_{K\cap M}$
vanishes for each $K\in \tmg m{a-s}$ that
is not contained in $T\ohne J$.
For, if $K$ contains an element of $I^+$, the basis element $\bs I$
will be annihilated multiplying by $c_{K\cap L}$ whereas,
in the case $K\cap I^+\neq \emptyset$
multiplication by $d_{K \cap M}$ results in zero. Here, we have used the
commutativity between $c_{K\cap L}$ and $d_{K\cap M}$.
Finally, if $K\cap J\neq \emptyset$, we have $d_Jc_{K\cap L}=0$.
A set $K \subseteq T\ohne J$ is obviously contained in $M$. 
By the second part of the lemma we have
$s(K,L)=v(K, L)=v(K,I^+\cup J)$. Thus, setting $S=J\cup K$ the assertion
follows.
\end{Pf}


\begin{lem} \label{summenformel v(K,L)}
Let $M\subseteq K \subseteq \mg m$ be fixed. Then
\[ \sum_{L\subseteq K\ohne M} (-1)^{|L|}y^{v(K,L)}=
\left\{\begin{array}{ll}
1 & K=M \\
0 & K \neq M\end{array} \right. \]
\end{lem}

\begin{Pf}
Clearly the sum is $1$ if $K=M$. For $K\neq M$ 
we show by induction on $n:=|K|> 0$ 
that the sum is zero.
Starting with the case $n=1$ we have $M=\emptyset$ and
$\sum_{L\subseteq K} (-1)^{|L|}y^{v(K,L)}= 
y^{v(K, \emptyset)}-y^{v(K,K)}=1-1=0$. For the induction step, 
let $x \in K$ be minimal and set $\widehat K:=K\ohne x$.
If $x \not\in M$ we calculate
\[ \sum_{L\subseteq K\ohne M} (-1)^{|L|}y^{v(K,L)}=
 \sum_{L\subseteq \widehat K\ohne M} (-1)^{|L|}y^{v(K,L)}
+  \sum_{L\subseteq \widehat K\ohne M} (-1)^{|L|+1}y^{v(K,L\cup \{x\})}. \]
Since $x$ is minimal, $v(K,L)=v(\widehat K, L)$ and $v(K, L\cup\{x\})=
v(\widehat K,L)+n-1$ if $L\subseteq \widehat K\ohne M$.
Now, we may apply the induction hypothesis to $\widehat K$ which results in
\[ \sum_{L\subseteq K\ohne M} (-1)^{|L|}y^{v(K,L)}=(1-y^{n-1})
   \sum_{L\subseteq \widehat K\ohne M} (-1)^{|L|}y^{v(\widehat K, L)}=0.\]
In the case $x \in M$, we write $\widehat M:=M\ohne\{x\}$. Similarly, we
calculate
\[ \sum_{L\subseteq K\ohne M} (-1)^{|L|}y^{v(K,L)}=
 \sum_{L\subseteq \widehat K\ohne \widehat M} (-1)^{|L|}y^{v(\widehat K,L)}, \]
where the right hand side is zero by the induction hypothesis.
\end{Pf}


\begin{lem} \label{N_r fuer r>m}
Let $I \in \tmg nr$ with $r>m$ and set $a=|I^0|$, 
$\widehat I:=I\ohne \{i, i'|\; i \in I^0\}$ and
$T:=\mg m \ohne (I^+ \cup I^-)$. 
Then there is an invertible $a_I\in \gr$ such that the following 
equation holds:

\[ \bs{\widehat I}\sum_{J\subseteq T} (-1)^{|J|}y^{v(J,\widehat I^+\cup J )} 
d_JD_{a-|J|} = a_I\bs I. \]
\end{lem}

\begin{Pf}
Let $\widehat T=T \cup I^0$ and observe that ${\widehat I}^+=I^+\ohne I^0$.
Because $r>m$, we must have $a\geq 1$. On the other hand 
$2a+|\widehat I|=r>m$ implies $a>m-|\widehat I|-|I^0|=|T|\geq |J|$. Using Lemma 
\ref{Operation D_r} we calculate 

\[ \bs{\widehat I}\sum_{J\subseteq T} (-1)^{|J|}y^{v(J,\widehat I^+\cup J)}
d_JD_{a-|J|} =
\bs{\widehat I}\sum_{J\subseteq T} (-1)^{|J|}y^{v(J,\widehat I^+\cup J)}
\sum_{S \in \tmg ma, J \subset S \subseteq \widehat T}
y^{v(S\ohne J,\widehat I^+\cup J)}d_S = \]
\[ \bs{\widehat I}\sum_{J\subseteq S \subseteq \widehat T, J\cap I^0=\emptyset}
(-1)^{|J|}y^{v(S,J)+ v(S,\widehat I^+)}d_S=
 \bs{\widehat I}\sum_{S \subseteq \widehat T}
y^{v(S,\widehat I^+)}d_S\sum_{J\subseteq S\ohne I^0}
(-1)^{|J|}y^{v(S,J)}.  \]
But by Lemma \ref{summenformel v(K,L)}, the last term equals
$y^{v(I^0,\widehat I^+)}\bs{\widehat I}d_{I^0}$. Using relation (\ref{Rel 1})
of the exterior algebra, this can be transformed into
$\bs I$ up to some invertible mutiple $a_I$.
\end{Pf}

\newcommand{\iaA}{{\bigwedge}_{\gr, q}(n-2)}
\newcommand{\ZZ}{{\cal Z}}
\newcommand{\ZaA}{{\bigwedge}_{\ZZ, Q}(n)}

We are now able to prove Proposition \ref{Rel nicht spiegelsymplektisch} by induction
on the Lie rank $m$. In  the case where $m=1$ both sets of $\tmg
21=\{\{1\},\{2\}\}$ are reverse symplectic. In $\tmg 22$ there is just one set.
namely $I=\{1,2\}$, for which we have
 $\bs I=-qd_1=-qD_1\in N_2$. Thus there is nothing to prove here.\Ab

For the induction step we embed $\iaA$ into $\aA$ sending $\bs i$ to
$\bs{i+1}$. It is easy to check that this indeed leads to an embedding of
algebras. Using the induction hypothesis we may treat the
case where $I \subseteq \mg n\ohne\{1,n\}$ without much effort.
Some caution is needed only concerning the difference 
between the two ideals $N$ of $\iaA$ and
$\aA$ respectively. For a single element, this 
difference can be
expressed as a sum of basis elements $\bs L$ with $1, n \in L$,
which all are smaller than $\bs I$ in our order.
If $I\cap \{1,n\}\neq \emptyset$ we may apply the induction hypothesis
to the
set $\widehat I:=I\ohne\{1,n\}$ in case $\widehat I$ 
is non reverse symplectic too:

\[ \bs{\widehat  I}\equiv
\sum_{\widehat J\subseteq \mg n\ohne\{1,n\}, \; |\widehat I|=|\widehat J|, 
\; f(\ibf{\widehat j})<f(\ibf{\widehat  i})}
    a_{\widehat I\widehat J}\bs{\widehat J} +   
\sum_{\{1,n\} \subseteq L\subseteq \mg n}a_{\widehat IL}\bs L \;\; 
\mbox{  mod }\;\; N. \]

Again, the terms $a_{\widehat IL}\bs L$ compensate for 
the difference between the
two ideals $N$.
Multiplying this equation by $\bs 1$ from the left (respectively by $\bs n$
from the right,
respectively by both from both sides) yields the assertion because
the elements $\bs L$ vanish, and $f(\ibf{\widehat
  j})<f(\ibf{\widehat  i})$ implies $ f(\ibf{j})<f(\ibf{i})$,
where $\ibf j$ is the multi-index attached to the set $J=\widehat J\cup
(I\ohne \widehat I)$.\Ab

It remains to prove the assertion in the 
case where $\widehat I$ is reverse symplectic.
Here we need the preparations of this section.                      
By the reverse symplectic condition we have  $\sum_{i=j}^m\lambda_i\leq
m-j+1$ for all $j>1$, where
$(\lambda_1, \ldots , \lambda_m)=f(\ibf i)$.
Because $I$ itself is non reverse symplectic we must have
$r=|I|=\sum_{i=1}^m\lambda_i>m$. According to Lemma \ref{N_r fuer r>m} 
we conclude
$\bs I \in N$. But this implies the assertion of Proposition
\ref{Rel nicht spiegelsymplektisch} in the remaining case, too.












\section{Proof of Proposition \ref{Operation D_a trivial}}

In order to prove the proposition
we have to consider generalizations of the 
elements $D_1, \ldots , D_m$, which are defined for any $l\leq m$ by

\[ D_{a,l}:=\sum_{K\in \tmg la} d_K. \]

For a positive integer $k$, define the $y^{-1}$-integer
$\iyan{k}:=1+y^{-1}+y^{-2}+\ldots +y^{-k+1}\in \gr$. 

\begin{lem} \label{Formel D_1D_a}
Let $a\in \mg m$. Then we have
\[D_{1,l}D_{a,l}\equiv\iyan{a+1}D_{a+1,l} \] 
modulo the ideal spanned by $D_1$.
\end{lem}

\begin{Pf}
Using the above introduced notations we may write the right hand side
of (\ref{quadrat c_i}) as $d_l^2=(y-1)\sum_{i=l+1}^m d_i$. Since
$\sum_{i=l+1}^m d_i +D_{1,l}=D_1$, we deduce $d_l^2\equiv (y^{-1}-1)d_lD_{1,l}$
modulo $D_1$. \Ab

We proceed by induction on $l$. If $l=1$, both sides are zero if $a>1$.
In the case $a=1$ we have to show that $d_1^2 \equiv 0$. 
But by (\ref{quadrat c_i}) $yd_1^2=(y-1)d_1D_1 \equiv 0$, implying 
$d_1^2\equiv 0$ since $y$ is invertible.\Ab

For the induction step we write
$D_{a,l}=d_lD_{a-1,l-1}+D_{a,l-1}$ and obtain

\[ D_{1,l}D_{a,l}= d_l^2D_{a-1,l-1}+d_lD_{a,l-1}+D_{1,l-1}(d_lD_{a-1,l-1}
+D_{a,l-1})\equiv\]
\[ ((y^{-1}-1)\iyan a +1+\iyan a)d_lD_{a,l-1} + \yan{a+1}D_{a+1,l-1}. \]

But $(y^{-1}-1)\iyan a +1 +\iyan a=\iyan{a+1}$. Thus the lemma follows.
\end{Pf}

It should be remarked that a similar formula holds even with $=$ instead
of $\equiv$ if in the definition of $D_{a,l}$ the sets $K$ are taken
as subsets of $\{l, \ldots, m\}$ instead of $\{1, \ldots ,l\}$. In that
formula $y^{-1}$  has to be replaced by $y$
(\cite[Lemma 3.9.1]{doc}).\Ab

We introduce some new conventions.  To an ordered 
subset $J=\{j_1, j_2, \ldots , j_a\}
\subseteq \mg m$ we define corresponding multi-indices by 
\[\ibf{j'j}:= (j_1,j_1', j_2, j_2' , \ldots , j_a, j_a' ), \;\;\;
  \ibf{jj'}:= (j_1, j_2, \ldots , j_a, j_a', j_{a-1}', \ldots , j_1' )
.\]
Furthemore, we write
\[ u_J:=-\sum_{j \in J}j.\]

From the definition of $d_J$, we have $d_J=q^{u_J}\bs{\ibf{j'j}}$.
By the relations of the exterior algebra there is another integer
$a_J$ such that $\bs{\ibf{j'j}}=q^{a_J}\bs{\ibf{jj'}}$.
By (\ref{aA comodul}) and Corollary \ref{wedge-Regel} we calculate
\[ \cmd{\wedge}(d_J)=
\sum_{I \in \tmg n{2a}}q^{u_J+a_J}\bs I \otimes
\qbidet{\omega_{2a}}{\ibf i}{\ibf{jj'}} =
\sum_{I \in \tmg n{2a}}q^{u_J}\bs I \otimes
\qbidet{\omega_{2a}}{\ibf i}{\ibf{j'j}}. \]

Setting

\[ G_{\ibf i, l, a}=\sum_{J\in \tmg la } 
q^{u_J}\qbidet{\omega_{2a}}{\ibf i}{\ibf{j'j}}, \]

we may write $\cmd{\wedge}(D_a)=
\sum_{I \in \tmg n{2a}}\bs I \otimes G_{\ibf i, m, a}$.
Therefore the proposition which we have to prove in this section holds
if and only if
\[ G_{\ibf i, m,a} =0, \;\;\;\mbox{ for all } \ibf i \in \mind nr
\mbox{ and } a \in \mg m, \]

since the $\bs I$ form a free basis of the comodule.
We will prove this equation with the help of the {\em Laplace-Extension} which
is a special case of {\em Laplace-Duality} (Proposition
\ref{Laplace 1}) applied to the  partitions

\[ \lambda_t:=(2a-t+1,\underbrace{ 1,1, \ldots , 1}_{\mbox{$(t-1)$-times}})
\in \prt{t}{2a}. \]

{\bf Caution}: this symbol should not confused with the $t$-th component
of a partition $\lambda$.
A bideterminant $\qbidet{\lambda_t}{\ibf i}{\ibf j}$ is the product of
a $t \times t$ minor determinant with a monomial
\[\qbidet{\lambda_t}{\ibf i}{\ibf j}=
\qbidet{\omega_t}{(i_1, \ldots , i_t)}{
(j_1, \ldots , j_t)}
\xes{i_{t+1}}{j_{t+1}}\xes{i_{t+2}}{j_{t+2}}\ldots \xes{i_{2a}}{j_{2a}}.\]

Let $L_t$ denote the set of distinguished  left coset representatives
of $S_{\lambda_{t-1}}$ in $S_{\lambda_t}$. Using basic transpositions 
$s_i$ this set can be written down explicitly
 
\[ L_t=\{ \id, s_{t-1}, s_{t-2}s_{t-1}, \ldots , s_1s_2\ldots s_{t-1} \}. \] 

Setting

\[ \mu_t:=\sum_{w\in L_t}(-y)^{-l(w)}\beta (w) \] 

the quantum symplectic (left) {\em Laplace-Extention} deduced from Proposition
\ref{Laplace 1} reads

\[ \mu_t\no\qbidet{\lambda_{t-1}}{\ibf i}{\ibf j}=
\qbidet{\lambda_t}{\ibf i}{\ibf j}. \]

There is a very useful recursive calculation rule for the endomorphisms
$\mu_t$:
\begin{equation}\label{recursion mu}
 -y^{-1}\mu_t\beta_t = \mu_{t+1} -\id . 
\end{equation}


Before we state the fundamental lemma of this section we remind the
reader of the addition of multi-indices, for example
$\ibf{j'j}+(kk')=(j_1, j_1', \ldots , j_a, j_a', k, k' )$.

\begin{lem} \label{fundament}
Let $l, a \in \mg m$ and $\ibf i\in \nmind{2a}$. Then
\[ G_{\ibf i, l, a}=\sum_{J\in \tmg{l}{a-1}, k \in \mg l}
q^{u_J -k}\qbidet{\lambda_{2a-1}}{\ibf i}{\ibf{j'j}+(kk')}
\no (\id{} -y^{-a}\beta_{2a-1}). \]
\end{lem}


\begin{Pf}
First we treat the case where $a=1$.
Here we have $G_{\ibf i, l,1}=\sum_{j=1}^lq^{-1}
\qbidet{\omega_2}{\ibf i}{(jj')}$
by definition. Since $\tmg l0=\emptyset$ the summation on the
right hand side is over $k\in \mg l$ too. Furthermore,
\[ \qbidet{\lambda_1}{\ibf i}{(kk')}\no (\id{} -y^{-1}\beta_1) =
\xes{i_1}{k}\xes{i_2}{k'}\no (\id{} -y^{-1}\beta_1) =
\qbidet{\omega_2}{\ibf i}{(kk')}.\]
Thus both sides of the equation are identical. 
For the general case we use induction on $l$. In the case $l=1$ we
necessarily have $a=1$ and there is nothing to prove.
In order to prove the induction
step we may assume $a>1$ and $l>1$. We
 divide the summation on the right hand side
into three subsums: \\

\mbox{ }\hfill (A) $l \in J$\hfill (B) $l \not \in J,\; 
k=l$ \hfill (C) $l\not\in J,\; k<l$.\hfill\mbox{ }\\

Using (\ref{summe bis l}) we calculate for the subsum (A)
\[\sum_{J\in \tmg{l-1}{a-2}, k \in \mg l}
q^{u_J -k-l}\qbidet{\lambda_{2a-1}}{\ibf i}{\ibf{j'j}+(ll'kk')}
\no (\id{} -y^{-a}\beta_{2a-1}) = \]
\[
 \sum_{J\in \tmg{l-1}{a-2}, k \in \mg {l-1}}
q^{u_J -k-l}\qbidet{\lambda_{2a-1}}{\ibf i}{\ibf{j'j}+(ll'kk')}
-y^{-a}q^{u_J +k-3l}\qbidet{\lambda_{2a-1}}{\ibf i}{\ibf{j'j}+(ll'k'k)}\]
\[
+q^{-2l}\sum_{J\in \tmg{l-1}{a-2}}
q^{u_J}\qbidet{\lambda_{2a-1}}{\ibf i}{\ibf{j'j}+(ll'll')}
-y^{-a}q^{u_J}\qbidet{\lambda_{2a-1}}{\ibf i}{\ibf{j'j}+(ll'l'l)}.\]

The bideterminant $\qbidet{\lambda_{2a-1}}{\ibf i}{\ibf{j'j}+(ll'l'l)}$
vanishes by Corollary \ref{qdet Null}. Unfortunately the other
seprated bideterminant is not zero in general. By Corollary
\ref{wedge-Regel} we deduce from $\bs{ll'}=-\bs{l'l}+(y^{-1} -1)
\sum_{k=1}^{l-1}q^{l-k}\bs{kk'}$ the equation
\[ \qbidet{\lambda_{2a-1}}{\ibf i}{\ibf{j'j}+(ll'll')}
=(y^{-1} -1)q^{l}\sum_{k=1}^{l-1}q^{-k}
\qbidet{\lambda_{2a-1}}{\ibf i}{\ibf{j'j}+(kk'l'l)}.\]

Since $\beta_{2a-1}\beta_{2a-2}(\bs{\ibf{j'j}+(k'kll')})=
q^{2}\bs{\ibf{j'j}+(k'll�k)}$
and   $\beta_{2a-1}^{-1}\beta_{2a-2}^{-1}(\bs{\ibf{j'j}+(kk'll')})
=q^{-2}\bs{\ibf{j'j}+(kll�k')}$ we may again deduce from Corollary \ref{wedge-Regel}
that

\[\qbidet{\lambda_{2a-1}}{\ibf i}{\ibf{j'j}+(ll'kk')}=
\qbidet{\lambda_{2a-1}}{\ibf i}{\ibf{j'j}+(kk'll')}
\no \beta_{2a-1}^{-1}\beta_{2a-2}^{-1}\] \[
\qbidet{\lambda_{2a-1}}{\ibf i}{\ibf{j'j}+(ll'k'k)}=
\qbidet{\lambda_{2a-1}}{\ibf i}{\ibf{j'j}+(k'kll')}
\no \beta_{2a-1}\beta_{2a-2} \]

Here, in addition,  we have used the equations 
$\bs{(ll'k')}=q^{2}\bs{(k'll')}$ and
$\bs{(ll'k)}=q^{-2}\bs{(kll')}$ which are valid inside the exterior algebra.
Modulo $\gamma$ the congruence relation $\beta^{-1}\equiv (y^{-1}\beta+ (y^{-1} -1) \id{})$
holds. Therefore, modulo the ideal ${\cal G}_{2a}$ of ${\cal A}_{2a}$
the congruence 
\[
\beta_{2a-1}^{-1}\beta_{2a-2}^{-1}
\equiv y^{-2}\beta_{2a-1}\beta_{2a-2}+y^{-1}(y^{-1}-1)(\beta_{2a-1}+
\beta_{2a-2})+ (y^{-1}-1)^2\id{} \]
is valid which implies
\[\qbidet{\lambda_{2a-1}}{\ibf i}{\ibf{j'j}+(kk'll')}
\no \beta_{2a-1}^{-1}\beta_{2a-2}^{-1}=y^{-2}
\qbidet{\lambda_{2a-1}}{\ibf i}{\ibf{j'j}+(kk'll')}
\no \beta_{2a-1}\beta_{2a-2} \]
\[+y^{-1}(y^{-1}-1)\qbidet{\lambda_{2a-1}}{\ibf i}{\ibf{j'j}+(kk'll')}
\no \beta_{2a-1} -
(y^{-1}-1)\qbidet{\lambda_{2a-1}}{\ibf i}{\ibf{j'j}+(kk'll')}. \]
by Lemma \ref{Reduktion Hecke}. Here we have also used th fact that 
$\qbidet{\lambda_{2a-1}}{\ibf i}{\ibf{j'j}+(kk'll')} \no \beta_{2a-2}= 
-\qbidet{\lambda_{2a-1}}{\ibf i}{\ibf{j'j}+(kk'll')} $
by Lemma \ref{Formel in mSp}. So far, our calculations lead to the
following expression for the subsum (A):
\[
 \sum_{J\in \tmg{l-1}{a-2}, k \in \mg {l-1}}
q^{u_J -k-l-4}(\qbidet{\lambda_{2a-1}}{\ibf i}{\ibf{j'j}+(kk'll')}\no
\beta_{2a-1}\beta_{2a-2} \] \[
-y^{-(a-1)+k-(l-1)}\qbidet{\lambda_{2a-1}}{\ibf i}{\ibf{j'j}+(k'kll')}
\no\beta_{2a-1}\beta_{2a-2} \] \[
+(1-y)\qbidet{\lambda_{2a-1}}{\ibf i}{\ibf{j'j}+(kk'll')}
\no\beta_{2a-1} ). \]
Now we apply Laplace-Extension twice to the first and second term
and once to the third term. In case of the first term this gives
\[\qbidet{\lambda_{2a-1}}{\ibf i}{\ibf{j'j}+(kk'll')}=\mu_{2a-1}\mu_{2a-2}
\no\qbidet{\lambda_{2a-3}}{\ibf i}{\ibf{j'j}+(kk'll')}. \]
Note that we may commute the $\beta_{2a_1}\beta_{2a-2}$ from the right of
$\qbidet{\lambda_{2a-3}}{\ibf i}{\ibf{j'j}+(kk'll')}$ to the left by 
(\ref{Rel MA}), since a bideterminant corresponding 
to the partition $\lambda_{2a-3}$
is a monomial on the part where $\beta_{2a-1}$ and $\beta_{2a-2}$ are operating
on. A similar fact is true concerning the third term of the sum with respect
to $\lambda_{2a-2}$ and $\beta_{2a-1}$. Furthermore, note that
for fixed $J$ and arbitrary $\ibf k\in \nmind r$ we have
\[
\sum_{k=1}^{l-1}
-y^{-(a-1)+k-(l-1)}
\qbidet{\lambda_{2a-1}}{\ibf k}{\ibf{j'j}+(k'kll')}
=
\sum_{k=1}^{l-1}
-y^{-(a-1)}
\qbidet{\lambda_{2a-1}}{\ibf k}{\ibf{j'j}+(kk'll')}\no \beta_{2a-3}
\]
by (\ref{summe bis l}). Substituting these equations into 
the above expression for the subsum (A)
gives
\[
 \sum_{J\in \tmg{l-1}{a-2}, k \in \mg {l-1}}
q^{u_J -k-l-4}(\mu_{2a-1}\mu_{2a-2}\beta_{2a-1}\beta_{2a-2} \no
\qbidet{\lambda_{2a-3}}{\ibf i}{\ibf{j'j}+(kk'll')}\no (\id{} -y^{-(a-1)}
\beta_{2a-3}) \] \[
+(1-y)\mu_{2a-1}\beta_{2a-1}\no
\qbidet{\lambda_{2a-2}}{\ibf i}{\ibf{j'j}+(kk'll')} ).
 \]

To the first term of that sum we can apply the induction hypothesis. To this claim
note that the symbol $\no$ on the right of the bideterminant stands for a sum 
over the bideterminant�s left multi-index. In order to apply the induction hypothesis
this summation has to be commuted with the summation under the $\sum$-symbol.
In a similar way we can apply Lemma \ref{Formel D_1D_a}
together with Corollary \ref{wedge-Regel} to the second term of the
sum above. This results in
\[
 \sum_{J\in \tmg{l-1}{a-1}}
q^{u_J-l}(y^{-2}\mu_{2a-1}\mu_{2a-2}\beta_{2a-1}\beta_{2a-2} \no
\qbidet{\lambda_{2a-2}}{\ibf i}{\ibf{j'j}+(ll')}\] \[
+y^{-1}(y^{-1}-1)\iyan {a-1}\mu_{2a-1}\beta_{2a-1}\no
\qbidet{\lambda_{2a-2}}{\ibf i}{\ibf{j'j}+(ll')} ).
 \]

Since $\mu_{2a-2}$ and $\beta_{2a-1}$ commute, (\ref{recursion mu}) yields
\[y^{-2}\mu_{2a-1}\mu_{2a-2}\beta_{2a-1}\beta_{2a-2}=
(-y^{-1}\mu_{2a-1}\beta_{2a-1})(-y^{-1}\mu_{2a-2}\beta_{2a-2})=(\mu_{2a} -\id{})(\mu_{2a-1} -\id{}).\]
Similarily one calculates
\[y^{-1}(y^{-1}-1)\iyan {a-1}\mu_{2a-1}\beta_{2a-1}=
(1-y^{-(a-1)})(\mu_{2a} -\id{}). \]
Finally we obtain the following expression for the subsum (A):
\[
 \sum_{J\in \tmg{l}{a}, l\in J}
q^{u_J}(\mu_{2a}\mu_{2a-1}-\mu_{2a-1}-y^{-(a-1)}(\mu_{2a} - \id{}))\no
\qbidet{\lambda_{2a-2}}{\ibf i}{\ibf{j'j}}.
\]
The calculation of subsum (B) only needs one application of Laplace-Extension together with the
commutation of $\beta_{2a-1}$ from the right of the bideterminant to the left:
\[
 \sum_{J\in \tmg{l}{a}, l\in J}
q^{u_J} \qbidet{\lambda_{2a-1}}{\ibf i}{\ibf{j'j}}\no (\id{} -y^{-a}\beta_{2a-1} ) \] \[
= \sum_{J\in \tmg{l}{a}, l\in J}
q^{u_J}(\mu_{2a-1}+y^{-(a-1)}(\mu_{2a} - \id{}))\no
\qbidet{\lambda_{2a-2}}{\ibf i}{\ibf{j'j}}.
\]
Thus subsum (A) and (B) together equal
\[
 \sum_{J\in \tmg{l}{a}, l\in J}
q^{u_J}\mu_{2a}\mu_{2a-1}\no \qbidet{\lambda_{2a-2}}{\ibf i}{\ibf{j'j}}=
 \sum_{J\in \tmg{l}{a}, l\in J} q^{u_J} \qbidet{\lambda_{2a}}{\ibf i}{\ibf{j'j}}
\]

To the subsum (C) the induction hypothesis can be applied directly 
and we see that it equals
\[ G_{\ibf i,l-1,a}=
 \sum_{J\in \tmg{l-1}{a}} q^{u_J} \qbidet{\lambda_{2a}}{\ibf i}{\ibf{j'j}}.
\]
Thus, it follows that all three subsums add up to $G_{\ibf i, l, a}$.
\end{Pf}

Now we are able to prove Proposition \ref{Operation D_a trivial}. As we
have seen above, we have to show $G_{\ibf i,m,a}=0$ for $a=1, \ldots , m$
and $\ibf i \in \nmind{2a}$. From (\ref{g und qdet})
we already know $G_{\ibf i, m,1}=0$ for all $\ibf i \in \nmind 2$.
We will deduce the general case with help of 
Lemma \ref{fundament}. Let $a>1$ and
$\ibf i \in \nmind{2a}$ be arbitrary. We apply Laplace-Extension to the
formula of the lemma:
\[ G_{\ibf i, m, a}=\sum_{J\in \tmg{m}{a-1}, k \in \mg l}
q^{u_J -k}\mu_{2a-1}\no\qbidet{\lambda_{2a-2}}{\ibf i}{\ibf{j'j}+(kk')}
\no (\id{} -y^{-a}\beta_{2a-1}). \]

As in the proof of the lemma we may commute $(\id{} -y ^{-a}\beta_{2a-1})$
to the other side of the bideterminant. Let 
$\left(\mu_{\ibf i\ibf h}\right)_{\ibf i, \ibf h \in \nmind{2a}}$ be the
coefficient matrix of the endomorphism 
$\mu_{2a-1}(\id{} -y^{-a}\beta_{2a-1})$
with respect to the canonical basis. We denote the multi-index consisting
of the first $2a-2$ indices of $\ibf h$ by 
$\ibf{\bar h}:=(h_1, \ldots , h_{2a-2})$ and obtain
\[ 
G_{\ibf i, m, a}=\sum_{\ibf h\in \nmind{2a}}\mu_{\ibf i\ibf h}
\sum_{J\in \tmg{m}{a-1}}
q^{u_J}\qbidet{\omega{2a-2}}{\ibf{\bar h}}{\ibf{j'j}}\sum_{k=1}^m
q^{-k}\xes{h_{2a-1}}{k}\xes{h_{2a}}{k'}
\]
\[ =\sum_{\ibf h\in \nmind{2a}}\mu_{\ibf i\ibf h}
G_{\ibf{\bar h}, m,a-1} \sum_{k=1}^m
q^{-k}\xes{h_{2a-1}}{k}\xes{h_{2a}}{k'} = 0,
\]

since $G_{\ibf{\bar h}, m, a-1}=0$ for all $\ibf h\in \nmind{2a}$ by the 
induction hypothesis.

















\section{Finishing the proof of Theorem \ref{Basissatz}}

Let us briefly recall what we have done so far. With respect to the
proof that $\B_r$ is a basis, we have reduced the fact that it is a set
of generators
in section \ref{st formula} to the verification of Proposition 
\ref{straightening formula}, which we just have completed.
Further we already know from section \ref{basisdefinition}
that the axiom $(C2^*)$ of a cellular coalgebra
is valid. It remains to show axiom
$(C3^*)$ and the fact that $\B_r$ is linearly independent.\Ab

Let us start with the latter task. It is clearly enough to
consider the case where $\gr =\ZZ=\Z[q,q^{-1}]$ Any relation among elements of
$\B_r$ with coefficients from $\ZZ$ is a relation
with coefficients from the field of fractions $\K$ on $\ZZ$, too.
Thus, we only have to show $|\B_r|=\dim_{\K} \rsAn{\K, Q}{r}$.
Now, $\rsAn{\K,Q}{r}$ is the centralizer
coalgebra of the \BMW algebra acting on $\ntrts$.
Consequently, by the comparison theorems from \cite{ofrt}
the dimension in question is the same as the dimension of the
centralizer algebra of that action on $\ntrts$. But this dimension
can be deduced from well-known results from the theory of quantum groups,
for which we refer to \cite{charipr}. \Ab

With regard to this claim note that, although our $\beta$ differs
from the corresponding Quantum Yang Baxter operator in 
\cite[chapter 7 (p. 237)]{charipr},
by a multiple of $q^{-1}$, the centralizer algebra must be
the same. By \cite[Theorem 10.2.5]{charipr}, this algebra is just the
image of the {\em quantized universal enveloping algebra} (QUE) corresponding to
the Dynkin diagram $C_m$ under its action on $\ntrts$.
Now,  by \cite[Proposition 10.1.13 and Theorem 10.1.14]{charipr}, 
the tensor space $\ntrts$ 
decomposes into irreducibles as a QUE-module because $q \in  \K$ is
transcendental over $\Q$.
These irreducibles are indexed
by the highest weights of the symplectic group and have corresponding
dimensions. The weights occurring are the same as in the case of the
symplectic group and correspond precisely to the elements of the
set $\Lambda$ from the definition of $\B_r$ (cf. \cite[7.1]{ofrt}).
It follows from work of R.C.\ King  that these dimensions are 
just $|M(\lambda)|$ (\cite{king},cf. \cite{donk1}).
Consequently, we obtain the required identity:

\[ \dim_{\K}{\rsAn{\K, Q}{r}} =\sum_{\lambda \in \Lambda} |M(\lambda )|^2 = |\B_r
|. \]


We now verify axiom $(C3^*)$.
We abbreviate  $\abb :=\qrsAn{\gr} r$. Let
$D^{\blam}_{\ibf i, \ibf j}\in \B_r$ where $\blam =(\lambda, l)\in \Lambda$ and 
$\ibf i, \ibf j\in M(\lambda)$. As $\qcD^l$ is grouplike and 
$\kom$ a homomorphism of algebras we calculate using 
(\ref{bidet kom}) that

\[\kom(D^{\blam}_{\ibf i, \ibf j})=(\qcD^l\otimes \qcD^l)
\kom(\qbidet{\lambda}{\ibf i}{ \ibf j})=
\sum_{\ibf h\in \smind{\lambda}} D^{\blam}_{\ibf i, \ibf h}\otimes
D^{\blam}_{\ibf h, \ibf j}. \]


Here, as in section \ref{aussere Algebra}, 
 $\smind{\lambda}$ is the set of multi-indices that are
$\lambda$-column-standard with respect to the usual order $<$
on $\mg n$ (see section \ref{st formula}). Now, according to the straightening
formula \ref{straightening formula}
(after application of $^*$)
to each $\ibf h\in \smind{\lambda} $ and $\ibf k\in M(\lambda)$
there is an element $a_{\ibf h\ibf k}\in \gr$ (unique by the
linear independence of $\B_r$) such that

\[D^{\blam}_{\ibf h, \ibf j}\equiv\sum_{\ibf k\in M(\lambda)}
a_{\ibf h\ibf k}D^{\blam}_{\ibf k, \ibf j}\;\;
\mbox{ mod } \;\; \abb(>\blam).
 \]

We set

\[ h(\ibf k,\ibf i):=\sum_{\ibf h\in \smind{\lambda}  } 
D^{\blam}_{\ibf i, \ibf h} a_{\ibf h\ibf k}\;\; \in \abb(\geq\blam)
\]

and obtain

\[ \kom(D^{\blam}_{\ibf i,\ibf j})\equiv\sum_{\ibf k\in M(\lambda)}
h(\ibf k, \ibf i)\otimes D^{\blam}_{\ibf k, \ibf j}
\mbox{ mod } \;\; \abb(\geq\blam)\otimes \abb(>\blam).
\]

This completes the verification of axiom $(C3^*)$ and hence the proof of
Theorem \ref{Basissatz}.










\section{Quasi-Heredity of the Symplectic q-Schur Algebra}


In  \cite{graham} Graham and Lehrer have presented 
a nice criterion for quasi-heridity of a cellular algebra 
which we will now verify in our case.
This will prove Theorem
\ref{quasi-heriditary}.\Ab

To this aim we have to investigate the 
bilinear form $\phi_{\blam}$ on the
standard modules $W(\blam)$. We must show that they are not zero
(\cite[3.10]{graham}. Let us first calculate 
the Gram-matrix of $\phi_{\blam}$ with respect to 
the basis $\{C^{\blam}_{\ibf i}|\; \ibf i\in M(\blam)\}$ of
$W(\blam)$. We abbreviate its entries by $\phi_{\ibf i\ibf j}:=
\phi_{\blam}(C^{\blam}_{\ibf i}, C^{\blam}_{\ibf j})\in \gr$. According to the
definition in \cite[2.3]{graham}, these numbers satisfy

\[ C^{\blam}_{\ibf i, \ibf k}C^{\blam}_{\ibf l, \ibf j}\equiv
\phi_{\ibf k\ibf l}C^{\blam}_{\ibf i, \ibf j} \;\;\;
\mbox{ mod } \qssch{\gr} r(<\blam). \]

Such a congruence relation is valid with $\phi_{\ibf k\ibf l}$ being
independent of $\ibf i$ and $\ibf j$ by the axioms of a cellular algebra
(see \cite[1.7]{graham}). Dualizing this congruence we obtain the following
counterpart in the cellular coalgebra  $\abb =\qrsAn{\gr} r$:

\[\kom(D^{\blam}_{\ibf i, \ibf j})\equiv
\sum_{\ibf k,\ibf l \in M(\blam)} \phi_{\ibf k\ibf l}
D^{\blam}_{\ibf i, \ibf k}\otimes
D^{\blam}_{\ibf l, \ibf j} \]

modulo $\abb(\geq \blam)\otimes \abb(>\blam)+
\abb(> \blam)\otimes \abb(\geq\blam)$. According to the calculations
for the verification of axiom $(C3^*)$ in the previous section
we see using the notations from there that

\begin{equation}\label{Formel phi_ij}
\phi_{\ibf k\ibf l}=\sum_{\ibf h\in \smind{\lambda} }a_{\ibf h\ibf k}
a_{\ibf h\ibf l}. 
\end{equation}

The bilinear form $\phi_{\blam}$ is different from zero if this is the case
for a single entry $\phi_{\ibf k\ibf l}$. We calculate $\phi_{\ibf k\ibf k}$ where
$\ibf k$ is the $\lambda $-tableau $\itab\lambda {\ibf k}=T$ with constant rows
$T(i,j):=m+i$ for all $1\leq i\leq m$ and $1\leq j \leq \lambda _j$.
Obviously $T$ is a standard tableau with respect to both orders on \mg n, namely
$<$ as well as $\prec$. Furthermore the reverse symplectic condition holds
because $T(i,j) = (m-i+1)'$. Consequently we have
$\ibf k\in M(\blam)\cap \smind{\lambda} $. Note that 
$\ibf k$ does not contain any pair of associated indices
$(i, i')$. This implies that each multi-index $\ibf h$ for which
$a_{\ibf h\ibf k} \neq 0$ must have the same content as $\ibf k$,
as can be seen from the proof of the straightening algorithm.
The content $\eta:=|\ibf k|$ of $\ibf k$ is given by

\[ \eta_i=\left\{\begin{array}{cl}
0 & i \leq m\\
\lambda _{i-m} & i>m.
\end{array} \right.
\]

Note, that $\ibf k$ is the only element in $\smind{\lambda}$ having this
content. This means $a_{\ibf h\ibf k}=0$ if $\ibf h \neq \ibf k$ and
we conclude

\[ \phi_{\ibf k\ibf k}=\sum_{\ibf h\in \smind{\lambda} }{a_{\ibf h\ibf k}}^2=
{a_{\ibf k\ibf k}}^2=1. \]

By \cite[Remark 3.10]{graham}, this finishes the proof of Theorem
\ref{quasi-heriditary}.







\section{Outlook}

Dualizing the coalgebra map $\qrsAn{\gr}{r-2}\stackrel{\cdot\qcD}{\pfr}
\qrsAn{\gr} r$
from the sequence (\ref{sequenz}), one obtains an
epimorphism of algebras from $\ssch{q} r$ to $\ssch{q}{r-2}$. On a
basis element $C^{\blam}_{\ibf i, \ibf j}$ it is given by subtracting $1$ from $l$
in $\blam=(\lambda, l)$ and keeping $\ibf i, \ibf j$  fixed. Its
kernel is the
linear span of those basis elements which occur in the case $l=0$.
This forces a recursive structure on the representation theory of these
algebras in a similar way as is known for the Birman-Murakami-Wenzl
algebras (see \cite{birwenz}). In addition these epimorphisms
can be used to define an inverse limit of the symplectic $q$-Schur algebras
in a similar way as has been worked out for the type $A$ $q$-Schur algebra
in \cite[section 6.4]{greenr}. It seems to be plausible that accordingly
the quantized universal enveloping algebra embeds into this inverse
limit.\Ab

Concerning analogues to the orthogonal case, note that
Lemma \ref{kappa faktorisiert} will not
work here. Maybe, a way out is to consider coefficient functions of the
symmetric algebra, i.e. the elements
\[
\sum_{w \in \symg{\lambda}}
y^{-l(w)}\beta (w)\no\xes{\ibf i}{\ibf j}=
\sum_{w \in \symg{\lambda}} y^{-l(w)}\xes{\ibf i}{\ibf j}\no
\beta (w)
\]
instead of bideterminants, which are coefficient functions of the
exterior algebra.






\addcontentsline{toc}{chapter}{References}
\begin{thebibliography}{abcd}
\bibitem[BW]{birwenz} Birman,\ J., Wenzl,\ H,: Braids, Link Polynomials and a 
        new Algebra. Tansactions of the Amer. Math. Soc., 
        Vol. 313, No. 1 (1989), 249-273.
\bibitem[CP]{charipr} Chari,\ V., Pressley,\ A.: A Guide to Quantum Groups.
        Cambridge University Press. 1994.
\bibitem[Co]{concini} De Concini C.: Symplectic Standard Tableaux. Advances
        in Mathematics 34 (1979), 1-27.
\bibitem[DD]{dd} Dipper,\ R., Donkin,\ S.: Quantum $GL_n$. Proc. London Math.
       Soc. 63 (1991), 165-211.
\bibitem[DJ]{dj} Dipper,\ R., James,\ G.: The $q$-Schur Algebra. Proc.
        London Math. Soc. (3) 59 (1989), 23-50.
\bibitem[DJ2]{dj2} Dipper,\ R., James,\ G.: $q$-tensor space and $q$-Weyl 
        modules, Trans. A.M.S. 327 (1991), 251-282.
\bibitem[DJM]{dja} Dipper,\ R., James,\ G., Mathas,\ A.: 
        Cyclotomic $q$-Schur Algebras. Math. Zeitschrift 229 (1998),
        385-416.
\bibitem[Do1]{donk3} Donkin,\ S.: Good Filtrations of Rational Modules for
        Reductive Groups. Arcata Conf. on Repr. of Finite Groups. Proceedings
        of Symp. in Pure Math., Vol. 47 (1987), 69-80.
\bibitem[Do2]{donk1} Donkin,\ S.: Representations of symplectic groups and the
       symplectic tableaux of R.C. King. Linear and Multilinear Algebra, Vol.
       29  (1991), 113-124.
\bibitem[Dt]{doty} Doty,\ S.: Polynomial Representations, Algebraic Monoids,
       and Schur Algebras of Classical Type. J. of 
       Pure and Applied Algebra, 123 (1998), 165-199.
\bibitem[GL]{graham} Graham,\ J.J., Lehrer,\ G.I.: Cellular Algebras. 
        Invent. Math. 123 (1996), 1-34.
\bibitem[Gr]{green2} Green,\ J.A.: Combinatorics and the Schur algebra. J. of 
       Pure and Appl. Alg. 88 (1993), 89-106.
\bibitem[GR]{greenr} Green,\ R.M.: $q$-Schur algebras and quantized enveloping
       algebras. Thesis. University of Warwick, 1995.
\bibitem[HH]{hay3} Hashimoto,\ M., Hayashi,\ T.: Quantum Multilinear Algebra.
       Tohoku Math. J., 44 (1992), 471-521.
%\bibitem[Ha1]{hay4} Hayashi,\ T.: Quantum Deformation of Classical Groups. 
%       Publ.\ RIMS, Kyoto Univ. 28 (1992), 57-81.
\bibitem[Ha]{hay1} Hayashi,\ T.: Quantum Groups and Quantum Determinants. 
       J. of Algebra 152 (1992), 146-165.
\bibitem[Ki]{king} King,\ R.C.: Weight multiplicity for classical groups., 
       Group Theoretical Methods in Physics (fourth International Colloquium,
       Nijmegen 1975), Lecture Notes in Physics 50, Springer 1975.
\bibitem[KX]{steffen} K{\"o}nig,\ S., Xi,\ C.: On the structure of
       cellular algebras. Algebras and Modules II, Proceedings of ICRA VIII
       (Geiranger), CMS Conference Proceedings.
\bibitem[Ma]{martin} Martin,\ S.: Schur Algebras and Representation Theory.
       Cambridge University Press, 1993.
\bibitem[O1]{doc} Oehms,\ S.: Symplektische $q$-Schur-Algebren, Thesis, 
        University of Stuttgart. Shaker Verlag Aachen, 1997.
\bibitem[O2]{ofrt} Oehms,\ S.: Centralizer Coalgebras,
        FRT-Construction and Symplectic Monoids. 
        J. of Algebra 244 (2001), 19-44.
\bibitem[RTF]{frt} Reshetikhin,\ N. Y., Takhtajan,\ L. A., Faddeev,\ L. D.:
       Quantization of Lie groups and Lie algebras, Leningrad Math. J. 1
       (1990), 193-225.
\bibitem[Tk]{tak} Takhtajan,\ L.A.:  Lectures on Quantum Groups. 
       In: Introduction to Quantum Group and Integrable Massive Models
       of Quantum Field Theory (Hrsg.: M.-L. Ge, B.-H. Zhao. World 
       Scientific, 1990.
\bibitem[We]{wenzl2} Wenzl,\ H.: Quantum Groups and Subfactors of Type B, 
        C and D. Commun. Math. Phys. 133 (1990), 383-432.
\end{thebibliography}


\end{document}




