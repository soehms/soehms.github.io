\newtheorem{prop}{Proposition}[section]
\newtheorem{thm}[prop]{Theorem}
\newtheorem{cor}[prop]{Corollary}
\newtheorem{lem}[prop]{Lemma}
\newtheorem{Def}[prop]{Definition}
\newtheorem{rem}[prop]{Remark}

\newenvironment{Pf}{{\sc Proof: }}{$\Box$\\}
\newenvironment{PfS}{{\sc Sketch of Proof: }}{$\Box$\\}



\newcommand{\N}{{\mathbb N}\,}
\newcommand{\F}{{\mathbb F}\,}
\newcommand{\Z}{{\mathbb Z}\,}
\newcommand{\C}{{\mathbb C}\,}
\newcommand{\K}{{\mathbb K}\,}
\newcommand{\Q}{{\mathbb Q}\,}
\newcommand{\X}{{\mathbb X}\,}
\newcommand{\Y}{{\mathbb Y}\,}
\newcommand{\I}{{\mathbb I}\,}




\newcommand{\Ab}{\\[-2mm]}
\newcommand{\pfr}{\rightarrow}
\newcommand{\pfl}{\leftarrow}
\newcommand{\gdw}{\Longleftrightarrow}
\newcommand{\ohne}{\backslash}

\newcommand{\Dim}[2]{{\rm dim}_{#1}{(#2)}}      
\newcommand{\Min}[1]{{\rm min}(#1)}      
\newcommand{\Char}[1]{{\rm char}(#1)}   
\newcommand{\Ker}[1]{{\rm ker}(#1)}   
\newcommand{\Ima}[1]{{\rm im}(#1)}   
\newcommand{\mg}[1]{\underline{#1}}           %Menge 1,... , #1 
\newcommand{\tmg}[2]{P({#1},{#2})}     %#2-elm Teilmengen von 1,... ,#1
\newcommand{\tlmg}[3]{P({#1},{#2},{#3})}  %#2-elm Teilmengen von #3,... ,#1



\newenvironment{tableau}[1]{\begin{array}{*{#1}{|p{2mm}}|}
}{\end{array}}
\newcommand{\tline}[3]{#1  &\multicolumn{#2}{c}{} \\ \cline{1-#3}}








% Kapitel 1



\newcommand{\kom}{\Delta}                     % Komultiplikation
\newcommand{\koe}{\epsilon}                   % Koeins
\newcommand{\cmd}[1]{\tau_{#1}}               % Komodulstrukturabb.
\newcommand{\mult}{\bigtriangledown}          % Multiplikation
\newcommand{\eins}{\mbox{i}}                  % Eins
\newcommand{\md}[1]{\eta_{#1}}                % Modulstrukturabb.
\newcommand{\ap}{\gamma}                      % Antipode


\newcommand{\id}[1]{{\rm id}_{#1}}            % Identit"at
\newcommand{\gr}{R}                           % Grundring
\newcommand{\er}{S}                           % Erweiterungsring
\newcommand{\nt}{V}                           % Natuerlicher Modul
\newcommand{\Ed}{{\cal E}}                    % Abbk. f"ur \grntend
\newcommand{\rEd}{{\cal E}_r}                 % Abbk. f"ur \grntrtsend
\newcommand{\gEd}[1]{{\cal E}_{#1}}           % wie \rEd f"ur bel. r
\newcommand{\dual}[1]{{#1}^*}                 % dualer Modul
\newcommand{\dEd}{\dual{\Ed}}                 % dualer Modul zu \grntend
\newcommand{\rdEd}{\dual{\rEd}}               % dualer Modul zu \grntrtsend
\newcommand{\gdEd}[1]{\dual{\gEd{#1}}}        % wie \rdEd f"ur bel. r


\newcommand{\bs}[1]{v_{#1}}                   % Basis von \nt
\newcommand{\es}[2]{e_{#1#2}}                 % Basis von \Ed

\newcommand{\ts}[2]{{#1}^{\otimes #2}}        % mehrfache Tensorprodukte
\newcommand{\ntts}[1]{\nt^{\otimes #1}}       % "         " mit nat. Modul
\newcommand{\ntrts}{\nt^{\otimes r}}          % r-fache   " "   nat Mod.

\newcommand{\End}[2]{{\rm End}_{#1}{(#2)}}          % Endomrphismenring
\newcommand{\Hom}[3]{{\rm Hom}_{#1}{(#2,#3)}}       % Hom-Menge
\newcommand{\tr}{tr}                          % Spurabb.
\newcommand{\tens}[1]{{\cal T}(#1)}           % Tensoralgebra


\newcommand{\mind}[2]{I(#1,#2)}               %  Multiindizes
\newcommand{\nmind}[1]{\mind{n}{#1}}          % Menge der n Multiindizes
\newcommand{\mmind}[1]{\mind{m}{#1}}          % Menge der m Multiindizes
\newcommand{\ibf}[1]{{\bf #1}}                % Multiindex
\newcommand{\komp}[2]{\Lambda(#1, #2)}        % Komposition
\newcommand{\skomp}[2]{\Lambda^{\rm s}(#1, #2)}     % sympl. Komposition
\newcommand{\nkomp}[1]{\komp{n}{#1}}          % Komposition in n Teile
\newcommand{\nrkomp}{\komp{n}{r}}             % Komposition r in n Teile
\newcommand{\prt}[2]{\Lambda^+(#1, #2)}       % Partition in Teile
\newcommand{\sprt}[2]{\Lambda^{{\rm s}+}(#1, #2)}   % sympl. Partition in Teile
\newcommand{\nprt}[1]{\prt{n}{#1}}            % Partition in n Teile
\newcommand{\nrprt}{\prt{n}{r}}               % Partition r in n Teile
\newcommand{\rprt}[1]{\Lambda^+(#1)}          % Partition


\newcommand{\frt}[1]{{\cal M}(#1)}            % FRT-Konstruktion zu
\newcommand{\gfrt}[2]{{\cal M}(#1)_{#2}}      % FRT-Konstr. homog. Summ.
\newcommand{\rfrt}[1]{{\cal M}(#1)_r}         % FRT-Konstr. r-homog. Summ.

\newcommand{\greg}[2]{{#2}\left[#1\right]}    % Regul"arer Funktionenring
\newcommand{\reg}[1]{\greg{#1}{\gr}}          % Reg. Funktionenring "uber  \gr

\newcommand{\gmhg}[2]{{\rm M}_{#1}(#2)}       % Matrixhalbgruppe Typ A
\newcommand{\gspm}[2]{{\rm SpM}_{#1}(#2)}     % Matrixmonoid Typ C
\newcommand{\mmsp}[2]{{\rm SpH}_{#1}(#2)}     % nichtinvertierbare sympl. Matr.
\newcommand{\gom}[2]{{\rm OM}_{#1}(#2)}       % Matrixmonoid Typ B, D
\newcommand{\qmhg}[2]{{\rm M}_{#1,q}(#2)}     % Quanten mhg
\newcommand{\qspm}[2]{\gspm{#1,q}{#2}}        % Quanten spm
\newcommand{\qom}[2]{\gom{#1,q}{#2}}          % Quanten om
\newcommand{\qmo}[2]{\gmo{#1,q}{#2}}          % Quanten om
\newcommand{\mhg}[1]{\gmhg{#1}{\gr}}          % Matrixhalbgruppe Typ A
\newcommand{\spm}[1]{\gspm{#1}{\gr}}          % Matrixmonoid Typ C
\newcommand{\om}[1]{\gom{#1}{\gr}}            % Matrixmonoid Typ B, D
\newcommand{\gln}[2]{{\rm GL}_{#1}(#2)}       % Generelle lineare Gruppe
\newcommand{\gsp}[2]{{\rm GSp}_{#1}(#2)}      % Generelle symplektische Gr
\newcommand{\go}[2]{{\rm GO}_{#1}(#2)}        % Generelle orthogonale Gr
\newcommand{\zgsp}[2]{\overline{\gsp{#1}{#2}}} % Matrixmon. C Zariski-Abs.
\newcommand{\zgo}[2]{\overline{\go{#1}{#2}}}   % Matrixmon. B,D Zariski-Abs.
\newcommand{\sln}[2]{{\rm SL}_{#1}(#2)}       % Spezielle lineare Gruppe
\newcommand{\spn}[2]{{\rm Sp}_{#1}(#2)}       % Symplektische Gruppe
\newcommand{\on}[2]{{\rm O}_{#1}(#2)}         % Orthogonale Gruppe

\newcommand{\symg}[1]{{\cal S}_{#1}}          % Symetrische Gruppe
\newcommand{\asymg}[1]{\gr symg{#1}}          % Symetrische Gruppen-Algebra

\newcommand{\zopf}[1]{{\cal Z}_{#1}}          % Zopf Gruppe
\newcommand{\azopf}[1]{\gr \zopf{#1}}         % Zopf Gruppen-Algebra
\newcommand{\zerz}[1]{\sigma_{#1}}            % Zopf Gruppen Erzeuger

\newcommand{\frei}[1]{{\cal F}_{\gr,#1}}      % Freie-Algebra
\newcommand{\heck}[2]{{\cal H}_{#1,#2}}       % Hecke-Algebra
\newcommand{\bmw}[2]{{\cal C}_{#1,#2}}        % BMW-Algebra
\newcommand{\br}[2]{{\cal B}_{#1,#2}}         % Brauer-Algebra
\newcommand{\falg}{{\cal A}}                  % Algebren-Familie
\newcommand{\gfalg}[1]{{\cal A}_{#1}}         % Algebren-Familie homogene Teile
\newcommand{\rfalg}{{\cal A}_r}               % Algebren-Fam. r-homogene Teile
\newcommand{\dar}{\rho}                       % Darstellung der Familie
\newcommand{\rdar}{\rho_r}                    % r-te Darstellung der Familie
\newcommand{\gdar}[1]{\rho_{#1}}              % #1-te Darstellung der Familie
\newcommand{\il}{s}                           % linke Inlusion
\newcommand{\ir}{t}                           % rechte Inlusion
\newcommand{\gil}[1]{s_{#1}}                  % #1-te linke Inlusion
\newcommand{\gir}[1]{t_{#1}}                  % #1-te rechte Inlusion
\newcommand{\kerz}[2]{\isod([#1,#2])}         % Erzeuger Komm. Koideal
\newcommand{\ma}{\frt{\falg}}
\newcommand{\rma}{\rfrt{\falg}}

\newcommand{\ZK}[1]{M(#1)}                    % Zentralisatorkoalgebra
\newcommand{\ZA}[1]{C(#1)}                    % Zentralisatoralgebra
\newcommand{\KK}[1]{K(#1)}                    % Kommutatorkoideal
\newcommand{\MA}{\ZK A}                       % Zentralisatorkoalgebra von A
\newcommand{\tfq}[1]{\overline{#1}}           % Torsionsfreier Quotient
\newcommand{\TMA}{\tfq{\MA}}                  % Torsionsfreier Quotient von \MA
\newcommand{\Tma}{\tfq{\ma}}                  % Torsionsfreier Quotient von \ma
\newcommand{\Trma}{\tfq{\rma}}                % Torsionsfreier Quotient \rma
\newcommand{\CA}{\ZA A}                       % Zentralisatoralgebra von A
\newcommand{\CCA}{\ZA{\CA}}                   % Doppelzentralisator von A
\newcommand{\KA}{\KK A}                       % Kommutatorkoideal von A
\newcommand{\comp}[1]{{#1}^{\bot}}            % duales Komplement
\newcommand{\ausw}[1]{{\rm Ev}_{#1}}          % Auswerteabbildung
\newcommand{\merw}[2]{{#1}^{#2}}              % Erweiterter Modul
\newcommand{\smerw}[1]{\merw{#1}{\er}}        % Erweiterter Modul mit \er
\newcommand{\zerw}[2]{{#1}_{#2}}              % Zentralisator nach Erweiterung
\newcommand{\szerw}[1]{\zerw{#1}{\er}}        % Zentr. nach Erweiterung mit \er
\newcommand{\ierw}[2]{\merw{#1}{{#2}^{\succ}}}
                                              % Eingebetteter erw. Modul
\newcommand{\iserw}[1]{\ierw{#1}{\er}}        % Eingebett. erw. Modul \er
\newcommand{\einb}{\iota}                     % Einbettungsabbildungen
\newcommand{\geinb}[1]{\einb_{#1}}            % Einbettungsabb. konkret
\newcommand{\seinb}[1]{\smerw{(\geinb{#1})}}  % induzierte konkr. Einbettung
\newcommand{\duerw}[1]{\psi_{#1}}             % Isom. zwischen W*S und WS*
\newcommand{\auerw}[1]{\psi'_{#1}}            % Isom. zwischen W**S und WS**

\newcommand{\gloc}[2]{{#1}_{#2}}              % Lokalisation an Primideal
\newcommand{\rloc}[1]{\gr_{#1}}               % Lokalis. an Primideal in \gr
\newcommand{\Spec}[1]{{\rm Spec}(#1)}         % Spektrum eines Rings.

\newcommand{\qan}[1]{[#1]_q}                  % q-Analog
\newcommand{\ean}[1]{[#1]_{\epsilon}}         % q-Analog f"ur Zahl epsilon
\newcommand{\yan}[1]{\{#1\}_{y}}              % y-Analog 
\newcommand{\Xan}[1]{\{#1\}_{X}}              % y-Analog 
\newcommand{\ybin}[2]{\{{#1\atop #2}\}_{y}}   % y-analoger Binomialkoeff 
\newcommand{\iyan}[1]{\{#1\}_{y^{-1}}}        % y-Analog f"ur y^{-1}




% Kapitel 2


\newcommand{\An}[1]{A_{#1}(n)}
\newcommand{\rAn}[2]{A_{#1}(n,#2)}
\newcommand{\sAn}[1]{A^{{\rm s}}_{#1}(n)}
\newcommand{\rsAn}[2]{A^{{\rm s}}_{#1}(n,#2)}
\newcommand{\qAn}[1]{A_{q,#1}(n)}
\newcommand{\qrAn}[2]{A_{#1,q}(n,#2)}
\newcommand{\qsAn}[1]{A^{{\rm s}}_{#1,q}(n)}
\newcommand{\qrsAn}[2]{A^{{\rm s}}_{#1,q}(n,#2)}
\newcommand{\qsBn}[1]{A^{{\rm s}}_{#1,q}(n)}
\newcommand{\qrsBn}[2]{ A^{{\rm s}}_{#1,q}(n,#2)}
\newcommand{\rsBn}[2]{A^{{\rm s}}_{#1}(n,#2)}
\newcommand{\qshAn}[1]{A^{{\rm sh}}_{#1,q}(n)}
\newcommand{\qrshAn}[2]{A^{{\rm sh}}_{#1,q}(n,#2)}
\newcommand{\sch}[2]{S_{#1}(n,#2)}            % Schur-Algebra
\newcommand{\ssch}[2]{S^s_{#1}(n,#2)}         % sympl. Schur-Algebra
\newcommand{\qsch}[2]{S_{#1,q}(n,#2)}        % q-Schur-Algebra
\newcommand{\qssch}[2]{S^{\rm s}_{#1,q}(n,#2)}     % sympl. q-Schur-Algebra


\newcommand{\spT}{J}
\newcommand{\spq}{d}

\newcommand{\BMW}{Bir\-man-Mu\-ra\-ka\-mi-Wenzl }
\newcommand{\ggr}{\Z[BMW]}
\newcommand{\pgr}{\Z[B]}                      % Grundring Brauer-Algebra plus
\newcommand{\pars}[1]{P_n^{\rm s}(#1)}        % sympl. BMW-Tripel
\newcommand{\paro}[1]{P_n^{\rm o}(#1)}        % orth.. BMW-Tripel
\newcommand{\agr}{{\cal Z}'}
\newcommand{\zgr}{\Z[H]}
\newcommand{\cgr}{{\cal Z}}
\newcommand{\ocgr}{\cgr'}
\newcommand{\gk}[2]{g_{#1|#2}}
\newcommand{\ek}[2]{e_{#1|#2}}


\newcommand{\spg}[1]{s_{#1}}          % Elementarspiegelung in \symg r
\newcommand{\Mla}{M^{\lambda}}        % Permutationsmoduln
\newcommand{\gewr}{({\ntrts})^{\lambda}}   % Gewichtsraum
\newcommand{\ysym}{\symg{\lambda}}    % Yang Untergruppe
\newcommand{\xsym}{\symg\mu }            % Yang Untergruppe
\newcommand{\dsym}[1]{{\cal D}_{#1}}  % Nebenklassen vertr. Yang Untergruppe
\newcommand{\dysym}{\dsym{\lambda}}   
\newcommand{\dxsym}{\dsym\mu }           
\newcommand{\ddysym}{\dsym{\lambda ,\mu }} % Doppelnebenklassen vertr.



\newcommand{\aqdar}{\rho_r}
\newcommand{\bqdar}{\rho^{\rm o}_r}
\newcommand{\cqdar}{\rho^{\rm s}_r}
\newcommand{\rcqdar}[1]{\rho^{\rm s}_{#1}}


\newcommand{\raqfalg}{\falg_r}           % Bild \aqdar
\newcommand{\rbqfalg}{\falg^{\rm o}_r}   % Bild \cqdar
\newcommand{\rcqfalg}{\falg^{\rm s}_r}   % Bild \bqdar 
\newcommand{\aqfalg}{\falg}              % zugehoerige Familien.
\newcommand{\bqfalg}{\falg^{\rm o}}
\newcommand{\cqfalg}{\falg^{\rm s}}










% Kapitel 3


%\newcommand{\naA}[2]{{\bigwedge}_{#1}(#2)}
%\newcommand{\nraA}[3]{{\bigwedge}_{#1}(#2,#3)}
%\newcommand{\aA}{\naA{\gr}{n}}
%\newcommand{\raA}[1]{\nraA{\gr}{n}{#1}}
\newcommand{\snaA}[2]{{\bigwedge}^{\rm s}_{#1}(#2)}
\newcommand{\snraA}[3]{{\bigwedge}^{\rm s}_{#1}(#2,#3)}
\newcommand{\saA}{\snaA{\gr}{n}}
\newcommand{\sraA}[1]{\snraA{\gr}{n}{#1}}
\newcommand{\qnaA}[2]{{\bigwedge}_{#1,q}(#2)}
\newcommand{\qnraA}[3]{{\bigwedge}_{#1,q}(#2,#3)}
\newcommand{\aA}{\qnaA{\gr}{n}}
\newcommand{\raA}[1]{\qnraA{\gr}{n}{#1}}
\newcommand{\qsnaA}[2]{{\bigwedge}^{\rm s}_{#1,q}(#2)}
\newcommand{\qsnraA}[3]{{\bigwedge}^{\rm s}_{#1,q}(#2,#3)}

\newcommand{\qsaA}{\qsnaA{\gr}{n}}

\newcommand{\qsraA}[1]{\qsnraA{\gr}{n}{#1}}




\newcommand{\vaA}[1]{\qnaA{#1}{n}}
\newcommand{\vgaA}[2]{\qnraA{#1}{n}{#2}}
\newcommand{\vnaA}[1]{{\qsnaA{#1}{n}}}
\newcommand{\vngaA}[2]{\qsnraA{#1}{n}{#2}}
\newcommand{\vtaA}[2]{\vaA{#1}[#2]}
%\newcommand{\aA}{\vaA{}}
\newcommand{\gaA}[1]{\vgaA{\gr}{#1}}
\newcommand{\taA}[1]{\vtaA{\gr}{#1}}
%\newcommand{\raA}{\gaA{r}}
\newcommand{\naA}{\vnaA{}}
\newcommand{\nraA}{\vngaA{}{r}}
\newcommand{\vsA}[1]{{\sf \large S}_{#1}}
\newcommand{\vgsA}[2]{\vsA{#1}(#2)}
\newcommand{\sA}{\vsA{}}
\newcommand{\rsA}{\vgsA{r}}
\newcommand{\frtsym}{A^{\rm s}(n)}
\newcommand{\kfrtsym}{\qsAn{\gr}{1}}
\newcommand{\gfrtsym}{A^{\rm sh}(n)}
\newcommand{\rgfrtsym}[1]{A^{\rm sh}(n,#1)}
\newcommand{\rfrtsym}[1]{A^{\rm s}(n,#1)}
\newcommand{\rkfrtsym}[1]{\qrsAn{\gr}{1}{#1}}
\newcommand{\shape}[1]{[#1]}                 %Diagramm von Partition
\newcommand{\lshape}{\shape{\lambda}}        %Diagramm von #1
\newcommand{\gtab}[1]{T^{#1}}                %Grund Tableau von #1
\newcommand{\stabz}[1]{Z(\gtab{#1})}         %Zeilenstabilisator
\newcommand{\stabs}[1]{S(\gtab{#1})}         %Spaltenstabilisator

% Tableau mengen

\newcommand{\gsmind}[2]{I_{#1}^{#2}}          %Standardtableau
\newcommand{\asmind}[1]{\gsmind{#1}{}} 
\newcommand{\smind}[1]{\gsmind{#1}{<}} 
\newcommand{\symind}[1]{\gsmind{#1}{\rm sym}}
\newcommand{\symmind}[1]{\gsmind{#1}{\rm mys}} 
\newcommand{\ssmind}[1]{\gsmind{#1}{\rm col}} 

\newcommand{\itab}[2]{\gtab{#1}_{\ibf{#2}}}  %Tableau zu Multiindex
\newcommand{\col}[2]{C_{#1}(#2)}             %Spalte eines Tableaus
\newcommand{\row}[2]{R_{#1}(#2)}             %Zeile eines Tableaus
\newcommand{\bidet}[3]{\gtab{#1}(#2:#3)}     %Bideterminante klassisch
\newcommand{\qbidet}[3]{\gtab{#1}_q(#2:#3)}  %Bideterminante quantum
\newcommand{\pbidet}[3]{t^{#1}(#2:#3)}       %PraeBideterminante klassisch
\newcommand{\pqbidet}[3]{t_q^{#1}(#2:#3)}    %PraeBideterminante quantum
\newcommand{\qdet}[2]{{\rm det}_q(#1, #2)}
\newcommand{\Det}[2]{{\rm det}(#1, #2)}
\newcommand{\libf}[3]{\ibf{#1}^{#2}_{#3}}    %Summmand Multi-Index bez. Komp. 

\newcommand{\xes}[2]{x_{#1 #2}}              %Monome in $\frtsym$
\newcommand{\yes}[2]{X_{#1 #2}}              % Unbestimmte
\newcommand{\B}{{\bf B}}                     %Basis von $\frtsym$
\newcommand{\BB}{{\bf C}}                    %Teil von Basis von $\frtsym$
\newcommand{\stdard}[1]{L(#1)}               % Standardmodul (weyl)
\newcommand{\strich}[1]{{#1}^{+}}
\newcommand{\kstrich}[1]{{#1}^{-}}
\newcommand{\ostrich}[1]{{#1}^{\circ}}
\newcommand{\sign}[1]{{\rm sign}(#1)}        %Signum einer Perm.

\newcommand{\rg}[1]{{\rm rg}(#1)}
\newcommand{\aeq}[1]{<#1>}
\newcommand{\spi}[1]{{#1}^\times }

%\newcommand{\K}{{\cal K}}                    % Halbkoaslgebra in straighening
\newcommand{\M}{{\cal N}}                    % Ordnungsmenge
\newcommand{\fM}[1]{[]#1[]}                  % Ordnungsfunktion 
\newcommand{\ltperm}[2]{\ll _{#1}^{#2}}         % lt-Permutation operator.  
\newcommand{\Uilt}[1]{U_{\ibf i,l#1}}        % Aufspann  alle lt permut















% Kapitel 4

\newcommand{\dbidet}[3]{D^{#1}_{#2,#3}}
\newcommand{\cbidet}[3]{C^{#1}_{#2,#3}}



%%% Local Variables: 
%%% mode: latex
%%% TeX-master: "doc"
%%% End: 
