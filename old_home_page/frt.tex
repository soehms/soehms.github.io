\documentclass[twoside,12pt]{article}
%\usepackage{babel}
\usepackage{latexsym,amssymb,amsfonts}
\topmargin-1.1cm
\parindent0cm
\headheight0mm
\textwidth16.0cm
\textheight24.5cm
\tabcolsep-4mm
\oddsidemargin-2mm
\evensidemargin-2mm
\pagestyle{myheadings}
\markboth{\centerline{Sebastian Oehms}}{\centerline{Symplectic Monoids}}






\begin{document}





\newtheorem{prop}{Proposition}[section]
\newtheorem{thm}[prop]{Theorem}
\newtheorem{cor}[prop]{Corollary}
\newtheorem{lem}[prop]{Lemma}
\newtheorem{Def}[prop]{Definition}
\newtheorem{rem}[prop]{Remark}

\newenvironment{Pf}{{\sc Proof: }}{$\Box$\\}
\newenvironment{PfS}{{\sc Sketch of Proof: }}{$\Box$\\}



\newcommand{\N}{{\mathbb N}}
\newcommand{\F}{{\mathbb F}}
\newcommand{\Z}{{\mathbb Z}}
\newcommand{\C}{{\mathbb C}}
\newcommand{\Q}{{\mathbb Q}}




\newcommand{\Ab}{\\[-2mm]}
\newcommand{\pfr}{\rightarrow}
\newcommand{\pfl}{\leftarrow}
\newcommand{\gdw}{\Longleftrightarrow}
\newcommand{\ohne}{\backslash}

\newcommand{\Dim}[2]{{\rm dim}_{#1}{(#2)}}      
\newcommand{\Min}[1]{{\rm min}(#1)}      
\newcommand{\Char}[1]{{\rm char}(#1)}   
\newcommand{\Ker}[1]{{\rm ker}(#1)}   
\newcommand{\Ima}[1]{{\rm im}(#1)}   
\newcommand{\mg}[1]{\underline{#1}}           %Menge 1,... , #1 
\newcommand{\tmg}[2]{P({#1},{#2})}     %#2-elm Teilmengen von 1,... ,#1



\newenvironment{tableau}[1]{\begin{array}{*{#1}{|p{2mm}}|}
}{\end{array}}
\newcommand{\tline}[3]{#1  &\multicolumn{#2}{c}{} \\ \cline{1-#3}}




\newcommand{\kom}{\Delta}                     % Komultiplikation
\newcommand{\koe}{\epsilon}                   % Koeins
\newcommand{\cmd}[1]{\tau_{#1}}               % Komodulstrukturabb.


\newcommand{\id}[1]{{\rm id}_{#1}}            % Identit"at
\newcommand{\gr}{R}                           % Grundring
\newcommand{\er}{S}                           % Erweiterungsring
\newcommand{\nt}{V}                           % Natuerlicher Modul
\newcommand{\Ed}{{\cal E}}                    % Abbk. f"ur \grntend
\newcommand{\rEd}{{\cal E}_r}                 % Abbk. f"ur \grntrtsend
\newcommand{\gEd}[1]{{\cal E}_{#1}}           % wie \rEd f"ur bel. r
\newcommand{\dual}[1]{{#1}^*}                 % dualer Modul
\newcommand{\dEd}{\dual{\Ed}}                 % dualer Modul zu \grntend
\newcommand{\rdEd}{\dual{\rEd}}               % dualer Modul zu \grntrtsend
\newcommand{\gdEd}[1]{\dual{\gEd{#1}}}        % wie \rdEd f"ur bel. r


\newcommand{\bs}[1]{v_{#1}}                   % Basis von \nt
\newcommand{\es}[2]{e_{#1}^{ #2}}              % Basis von \Ed
\newcommand{\dbs}[1]{\dual{\bs{#1}}}          % Basis von \dnt
\newcommand{\des}[2]{{\dual{e}_{#1}}^{#2}}    % Basis von \dEd

\newcommand{\isod}{{\vartheta}_{\tr}}         % Isomorph. \Ed nach \dEd

\newcommand{\ts}[2]{{#1}^{\otimes #2}}        % mehrfache Tensorprodukte
\newcommand{\ntts}[1]{\nt^{\otimes #1}}       % "         " mit nat. Modul
\newcommand{\ntrts}{\nt^{\otimes r}}          % r-fache   " "   nat Mod.

\newcommand{\End}[2]{{\rm End}_{#1}{(#2)}}          % Endomrphismenring
\newcommand{\Hom}[3]{{\rm Hom}_{#1}{(#2,#3)}}       % Hom-Menge
\newcommand{\tr}{tr}                          % Spurabb.
\newcommand{\tens}[1]{{\cal T}(#1)}           % Tensoralgebra


\newcommand{\mind}[2]{I(#1,#2)}               %  Multiindizes
\newcommand{\nmind}[1]{\mind{n}{#1}}          % Menge der n Multiindizes
\newcommand{\mmind}[1]{\mind{m}{#1}}          % Menge der m Multiindizes
\newcommand{\ibf}[1]{{\bf #1}}                % Multiindex
\newcommand{\komp}[2]{\Lambda(#1, #2)}        % Komposition
\newcommand{\skomp}[2]{\Lambda^{\rm s}(#1, #2)}     % sympl. Komposition
\newcommand{\nkomp}[1]{\komp{n}{#1}}          % Komposition in n Teile
\newcommand{\nrkomp}{\komp{n}{r}}             % Komposition r in n Teile
\newcommand{\prt}[2]{\Lambda^+(#1, #2)}       % Partition in Teile
\newcommand{\sprt}[2]{\Lambda^{{\rm s}+}(#1, #2)}   % sympl. Partition in Teile
\newcommand{\nprt}[1]{\prt{n}{#1}}            % Partition in n Teile
\newcommand{\nrprt}{\prt{n}{r}}               % Partition r in n Teile
\newcommand{\rprt}[1]{\Lambda^+(#1)}          % Partition


\newcommand{\frt}[1]{{\cal M}(#1)}            % FRT-Konstruktion zu


\newcommand{\gmhg}[2]{{\rm M}_{#2}(#1)}       % Matrixhalbgruppe Typ A
\newcommand{\gspm}[2]{{\rm SpM}_{#2}(#1)}     % Matrixmonoid Typ C
\newcommand{\mmsp}[2]{{\rm SpH}_{#2}(#1)}     % nichtinvertierbare sympl. Matr.
\newcommand{\gln}[2]{{\rm GL}_{#2}(#1)}       % Generelle lineare Gruppe
\newcommand{\gsp}[2]{{\rm GSp}_{#2}(#1)}      % Generelle symplektische Gr
\newcommand{\zgsp}[2]{\overline{\gsp{#1}{#2}}} % Matrixmon. C Zariski-Abs.
\newcommand{\sln}[2]{{\rm SL}_{#2}(#1)}       % Spezielle lineare Gruppe
\newcommand{\spn}[2]{{\rm Sp}_{#2}(#1)}       % Symplektische Gruppe

\newcommand{\symg}[1]{{\cal S}_{#1}}          % Symetrische Gruppe

\newcommand{\zopf}[1]{{\cal Z}_{#1}}          % Zopf Gruppe

\newcommand{\br}[2]{D_{#1}({#2})}         % Brauer-Algebra
\newcommand{\falg}{{\cal A}}                  % Algebren-Familie
\newcommand{\gfalg}[1]{{\cal A}_{#1}}         % Algebren-Familie homogene Teile
\newcommand{\rfalg}{{\cal A}_r}               % Algebren-Fam. r-homogene Teile
\newcommand{\il}{s}                           % linke Inlusion
\newcommand{\ir}{t}                           % rechte Inlusion
\newcommand{\gil}[1]{s_{#1}}                  % #1-te linke Inlusion
\newcommand{\gir}[1]{t_{#1}}                  % #1-te rechte Inlusion
\newcommand{\kerz}[2]{\isod([#1,#2])}         % Erzeuger Komm. Koideal

\newcommand{\ZK}[1]{M(#1)}                    % Zentralisatorkoalgebra
\newcommand{\ZA}[1]{C(#1)}                    % Zentralisatoralgebra
\newcommand{\KK}[1]{K(#1)}                    % Kommutatorkoideal
\newcommand{\KL}[1]{L(#1)}                    % Vor-Kommutatorkoideal
\newcommand{\MA}{\ZK A}                       % Zentralisatorkoalgebra von A
\newcommand{\CA}{\ZA A}                       % Zentralisatoralgebra von A
\newcommand{\KA}{\KK A}                       % Kommutatorkoideal von A
\newcommand{\comp}[1]{{#1}^{\bot}}            % duales Komplement
\newcommand{\ausw}[1]{{\rm Ev}_{#1}}          % Auswerteabbildung
\newcommand{\merw}[2]{{#1}^{#2}}              % Erweiterter Modul
\newcommand{\smerw}[1]{\merw{#1}{\er}}        % Erweiterter Modul mit \er
\newcommand{\zerw}[2]{{#1}_{#2}}              % Zentralisator nach Erweiterung
\newcommand{\szerw}[1]{\zerw{#1}{\er}}        % Zentr. nach Erweiterung mit \er

                                              % Eingebetteter erw. Modul
\newcommand{\iserw}[1]{\ierw{#1}{\er}}        % Eingebett. erw. Modul \er
\newcommand{\einb}{J}                     % Einbettungsabbildungen
\newcommand{\geinb}[1]{\einb_{#1}}            % Einbettungsabb. konkret
\newcommand{\seinb}[1]{\smerw{\geinb{#1}}}  % induzierte konkr. Einbettung
\newcommand{\duerw}[1]{\psi_{#1}}             % Isom. zwischen W*S und WS*


\newcommand{\An}[1]{A_{#1}(n)}
\newcommand{\rAn}[2]{A_{#1}(n,#2)}
\newcommand{\sAn}[1]{A^{{\rm s}}_{#1}(n)}
\newcommand{\rsAn}[2]{A^{{\rm s}}_{#1}(n,#2)}
\newcommand{\shAn}[1]{A^{{\rm sh}}_{#1}(n)}
\newcommand{\rshAn}[2]{A^{{\rm sh}}_{#1}(n,#2)}



\newcommand{\spT}{J}
\newcommand{\spq}{d}


\newcommand{\naA}[2]{{\bigwedge}_{#1}(#2)}
\newcommand{\nraA}[3]{{\bigwedge}_{#1}(#2,#3)}
\newcommand{\aA}{\naA{\gr}{n}}
\newcommand{\raA}[1]{\nraA{\gr}{n}{#1}}
\newcommand{\snaA}[2]{{\bigwedge}^{\rm s}_{#1}(#2)}
\newcommand{\snraA}[3]{{\bigwedge}^{\rm s}_{#1}(#2,#3)}
\newcommand{\saA}{\snaA{\gr}{n}}
\newcommand{\sraA}[1]{\snraA{\gr}{n}{#1}}
\newcommand{\gtab}[1]{T^{#1}}                %Grund Tableau von #1

\newcommand{\gsmind}[2]{I_{#1}^{#2}}          %Standardtableau
\newcommand{\asmind}[1]{\gsmind{#1}{}} 
\newcommand{\symind}[1]{\gsmind{#1}{\rm sym}}

\newcommand{\itab}[2]{\gtab{#1}_{\ibf{#2}}}  %Tableau zu Multiindex
\newcommand{\bidet}[3]{\gtab{#1}(#2:#3)}     %Bideterminante klassisch

\newcommand{\xes}[2]{x_{#1 #2}}              %Monome in $\frtsym$
\newcommand{\B}{{\bf B}}                     %Basis von $\frtsym$
\newcommand{\sign}[1]{{\rm sign}(#1)}        %Signum einer Perm.







\thispagestyle{empty}

\begin{center} 
{\bf \huge Centralizer Coalgebras, FRT-Construction and Symplectic Monoids}
 \\[3mm]
{\bf  Sebastian Oehms\footnote{E-Mail: seba@mathematik.uni-stuttgart.de}, 
Mathematisches Institut B,\\
Universit{\"a}t Stuttgart, D70550 Stuttgart, Germany
}
\end{center}


\section{Introduction}

L.\ Faddeev, N.\ Reshetikhin and L.\ Takhtadjian \cite{frt}
introduced a construction to obtain quantum deformations of
coordinate rings of classical groups.
General considerations about this so called FRT-construction 
can be found for instance in
\cite{manin},\cite{tak2}, \cite{hay4}, \cite{Sudbery} 
and also in many textbooks on quantum groups. 
Our approach differs from former ones in the following three aspects:\Ab

First, we focus attention to the {\em graded matric bialgebra} which
arises in the first step of the construction. This means that we
rather look
at quantizations of appropriate closed monoids instead of classical groups.
Especially we look at the {\em homogenous summands} of these graded bialgebras.
These are coalgebras which can be defined in a dual way to
{\em centralizer algebras} of subsets in an endomorphism ring.
We therefore call them {\em centralizer coalgebras} and investigate their
relationship to the corresponding {\em centralizer algebras}.\Ab

Further, we work over arbitrary noetherian integral domains as base rings. This makes
sense since all known examples are already well defined
over rings of integral {\em Laurent} polynomials, i.e.\
$\Z[X, X^{-1}]$ (in the  indeterminate $X$). We will see that there are
tremendous differences to the theory over fields, especially
concerning the comparison of centralizer algebras and coalgebras.
For instance the centralizer coalgebra may have $\gr$-torsion. We present the
following criterion for $\gr$-projectivity: This property holds if and only if
the centralizer algebra is stable under base changes. Furthermore, we will see
that the latter property is always valid for centralizer coalgebras.
\Ab

Finally, the FRT-construction in the ordinary form depends on exactly
one endomorphism 
which usually is a {\em quantum Yang-Baxter operator} in the applications.
Here we give another version of the FRT-construction which can be applied to
sets of endomorphisms. This generalization is necessary to describe
the coordinate rings of 
classical symplectic and orthogonal monoids by use of an
FRT-type construction. We demonstrate this in the symplectic case, giving
some improvements of
results by S.\ Doty \cite{doty}. 
Furthermore, applying our results on centralizer coalgebras
we obtain an integral form for the {\em symplectic Schur algebra} 
defined by S.\ Donkin in \cite{donk3}
without any use of the hyperalgebra or Kostant $\Z$-form. As an additional incredient
we need a symplectic version of the straightening formula for bideterminants
the proof of which covers all of the last two sections.














\section{Centralizer Coalgebras}


Let $\gr$ be a noetherian integral domain, $\nt$ a free
$\gr$-module with a fixed basis $\{\bs 1, \ldots , \bs n\}$. Let
$\es ij$ denote the corresponding basis of matrix units for
$\Ed:=\End{\gr}{\nt}$. The algebra structure on $\Ed$ induces a
coalgebra structure on the dual $\gr$-module
$\dEd:=\Hom{\gr}{\Ed}{\gr}$. On the dual basis elements the comultiplication
$\kom$ and the counit $\koe$ are given by

\[ \kom (\des ij)=\sum_{k=1}^n \des ik  \otimes \des kj ,\; \; \koe (\des ij)
=\delta_{ij}.\]

We will be engaged with epimorphic coalgebra images of $\dEd$ since
they correspond to subalgebras of $\Ed$. Consider the isomorphism
$\isod :\Ed \pfr \dEd$ of $\gr$-modules induced by 
the nondegenerate bilinear form corresponding to 
the matrix trace map $\tr :\Ed \pfr \gr$, i.e.\ $\isod(\mu)(\nu):= \tr(\mu\nu)$.
Note that $\isod$ maps $\es ij$ onto $\des ji$.
Let $A\subseteq \Ed$ be an arbitrary
subset and $\KL A$ the $\gr$-linear span in $\Ed$ of all commutators
$[\nu, \mu ]=\nu\mu-\mu\nu$ where $\nu$ runs through $A$ and $\mu$
runs through $\Ed$. We set $\KK A:=\isod (\KL A)\subseteq \dEd$. If $A$ consists of
just one element $\nu$ we use abbreviations $\KL{\nu}$ and $\KK{\nu}$
for $\KL A$ and $\KK A$.

\begin{lem} \label{Koideal}
$\KK A$ is a coideal in $\dEd$ for each subset $A\subseteq \Ed$.
\end{lem}

\begin{PfS}
Since the sum of coideals is a coideal again we only need to consider
the case $A=\{\nu\}$. Using an explicit expression for $\nu$ 
in terms of the basiselements $\es{i}{j}$ one
calculates for arbitrary numbers $k,l \in \{1,\ldots, n\}$:

\[ \kom (\isod([\nu ,\es{k}{l}]))=
\sum_m (
\des{l}{m}\otimes\isod([\nu,\es{k}{m}]) +
\isod([\nu, \des{m}{l}]) \otimes \des{m}{k},
\]

which just means

\[ \kom (\KK{\nu}) \subseteq \KK{\nu} \otimes \dEd + \dEd
\otimes \KK{\nu}. \]

In a simialar way one shows $\koe(\KK{\nu})=0$. A more detailed proof
can be found in section 1.3 of
\cite{doc}.
\end{PfS}

If $C$ is an arbitrary coalgebra such that $\nt$ is a $C$-comodule
with structure map $\cmd{\nt}:\nt \pfr \nt \otimes C$ we denote the set of $C$-comodule
endomorphisms by

\[ \End C{\nt}:=\{
\mu \in \Ed|\; \; (\mu \otimes \id{C}) \circ \cmd{\nt}= \cmd{\nt} 
\circ \mu \}.\]

Clearly this is a subalgebra of $\Ed$. The proof of the next lemma is
similar to the preceeding one.

\begin{lem} \label{Koalgebren Endomorphismen}
Let $C$ be a coalgebra together with an epimorphism
$\pi: \dEd \pfr C$. Then
\[ \mu \in \End{C}{\nt} \gdw \KK{\mu}  \subseteq \Ker{\pi}.
\]
\end{lem}

\begin{cor} \label{Verknupfungstreue}
Let $M$ be a coideal in $\dEd$ and $\mu , \nu \in \Ed$. Then
$\KK\mu  \subseteq M$ and $\KK{\nu} \subseteq M$ implies
$\KK{\mu \nu} \subseteq M$ and $\KK{\nu\mu } \subseteq M$.
\end{cor}

We now define the {\em centralizer coalgebra} of the subset
$A\subseteq\Ed$ as
\[\MA:=\dEd / \KK{A}.\] 
According to
lemma
\ref{Koalgebren Endomorphismen} we have
\[A \subseteq \End{\MA}{\nt}.\]
Furthermore $\MA$ is the largest epimorphic image of $\dEd$ with this
property. By the corollary $\MA$ does not change if $A$ is substituted
by its algebraic span. Before going deeper into the analysis of
relationships between $\MA$ and the {\em centralizer algebra}
\[ \ZA A:=\End A{\nt}=\{ \mu \in \Ed|\; [\mu , \nu]=0, \;\mbox{ for
all } \nu \in A\} \]
we give a presentation of $\MA$ by generators and relations, which is
convenient for practical use.\Ab

The residue classes of the
basis elements $\des ij$ with respect to any coideal in $\dEd$ will
always be denoted by $\xes ij$ where $i,j\in \mg n:=\{1, \ldots , n\}$. 
If 
$\mu =\sum_{i,j=1}^n a_{ij}\es ij\in \Ed$ is arbitrary we write 

\begin{equation}\label{Notation End}
\mu \xes{i}{j} :=\sum_{k=1}^n a_{ik}\xes{kj} \; \; \mbox{ and } \; \;
\xes{i}{j}\mu :=\sum_{k=1}^n \xes{i}{k}a_{kj}.
\end{equation}

Now, if $N$ is a subset of $\Ed$ and $A$ its algebraic span
then  $\ZK A$ is defined by the generators $\xes ij$ and the relations

\begin{equation} \label{Rel MA}
\mu \xes{i}{j} = \xes{i}{j} \mu \; \; \mbox{ for all } \mu \in N, \;
i,j \in \mg n.
\end{equation}

As consequences one has relations of the same form where $\mu$ runs
through all of $A$.








\section{Comparison Theorems}

Remember the definition of the {\em complement}

\[
\comp{U}:=\{f \in \dual{W}=\Hom{\gr}{W}{\gr}| \; f(u)=0 \; \, \forall u \in U\}
\]

of a submodule $U$ in an $\gr$-module $W$ and the definition of the {\em
  evaluation map}
\[\ausw{W}:W \pfr \dual{\dual W}, \;\mbox{ given by }
\ausw W(x)(y):=y(x)\; x \in W, \; y \in \dual W\] 
In $\dual{\dual{\dual{W}}}$ we have the following commutativity rule

\begin{equation} \label{Ev comp}
\ausw{\dual{W}}(\comp U)=\comp{\ausw W(U)}.
\end{equation}

Turning to our special situation we first note that 
$\tr(b[x,a])=\tr(x[a,b])$ for all $x, a, b \in \Ed$. Since the bilinear form
induced by $\tr$ is nondegenerate, it follows:
\begin{equation} \label{Lieklammer}
\isod([x,a])(b)=\tr(b[x,a])=0 \mbox{ for all } x \in\Ed \gdw [a,b]=0.
\end{equation}

The following fundamental lemma of this section is easy to prove now.

\begin{lem} \label{duales Komplement}
We have $\comp{\KA} =\ausw{\Ed}(\CA)$ and
$\comp{\comp{\KA}} =\ausw{\dEd}(\comp{\CA})$
\end{lem}

\begin{Pf}
According to the definition $\ausw{\Ed}(b)\in \comp{\KA}$ if and only if
$\ausw{\Ed}(b)(\isod([x,a]))=\isod([x,a])(b)=0$ for all $x \in \Ed$
and $a\in A$. Applying (\ref{Lieklammer}) this is the case if and only if
$[a,b]=0$ for all $a \in A$, thus if and only if
$b \in \CA$. The second equation follows from the first by use of
equation (\ref{Ev comp}).
\end{Pf}

Remember that the dual module
$\dual{C}=\Hom{\gr}{C}{\gr}$ of a coalgebra $C$ always posseses the structure
of an algebra by use of the {\em convolution product} 

\[ \mu \nu :=(\mu \otimes \nu) \circ\kom \;\;\mbox{ where }
\mu ,\nu \in \dual C.
 \]

Here, we have identified $\mu \otimes \nu$ with its image under the
natural homomorphism
$\dual{C} \otimes \dual{C} \pfr \dual{(C \otimes C)}$. Note that this 
construction is functorial. In the special case
$C=\dEd$ we obtain an algebra structure on $\dual{\dEd}$. Furthermore
it is easy to show that the evaluation map
$\ausw{\Ed}:\Ed\pfr\dual{\dEd}$ is an isomorphism of algebras.


\begin{lem} \label{duale Algebra allgemein}
Let $C$ be a coalgebra together with an epimorphism 
$\pi:\dEd\pfr C$. Set $K:=\Ker{\pi}$.
Then $\rho:=\ausw{\Ed}^{-1}\circ \dual{\pi}:\dual{C}\pfr \Ed$ is a
monomorphism
of $\gr$-algebras and $\Ima{\rho}=\ausw{\Ed}^{-1}(\comp K)$.
\end{lem}

\begin{Pf}
By functoriality the dual map
$\dual{\pi}:\dual{C} \pfr \dual{\dEd}$ is an
algebra homomorphism.
Since $\Hom{\gr}{-}{\gr}$ is exact on the right, $\dual{\pi}$ is injective.
One easily shows $\Ima{\dual{\pi}}=\comp{ K}$. This completes the
proof, since $\ausw{\Ed}$ is an algebra isomorphism as mentioned above.
\end{Pf}

\begin{thm}[First Comparison Theorem] \label{duale Algebra}
The centralizer algebras $\CA$ and the dual of the centralizer coalgebra
$\dual{\MA}$ are isomorphic to each other. An isomorphism
is given by $\rho$.
\end{thm}

\begin{Pf}
This follows immediately from lemmas
\ref{duales Komplement} and \ref{duale Algebra allgemein}.
\end{Pf}


Now, let us compare the dual of $\ZA A$ with $\MA$. To this aim we
consider the dual map $\dual{\einb}:\dEd \pfr \dual{\CA}$ of the inclusion
$\einb:\CA \hookrightarrow \Ed$.
Because of $\ausw{\dEd}(U)\subseteq \comp{\comp U}$ for arbitrary
submodules $U$ and according to Lemma
\ref{duales Komplement} we have

\[\KA \subseteq\ausw{\dEd}^{-1}(\comp{\comp{\KA}})=
\ausw{\dEd}^{-1}(\ausw{\dEd}(\comp{\CA}))= \comp{\CA}
=\Ker{\dual{\einb}}. \]

Therefore $\dual{\einb}$ factors to an $\gr$-module homomorphism

\[ \theta :\MA \pfr \dual{\CA}. \]

\begin{lem}\label{Kern theta}
The kernel of $\theta$ is precisely the torsion submodule of $\MA$
\end{lem}

\begin{Pf}
From general results of commutative algebra (cf. \cite{doc} Anhang A 1.2) it follows
that the torsion submodule of $\MA$ coincides with
$\ausw{\dEd}^{-1}(\comp{\comp{\KA}})/\KA$.
By the above calculations this is just the kernel
$\comp{\CA}/\KA$ of $\theta$.
\end{Pf}

This immediately implies

\begin{cor}[Criterion of torsion freeness] \label{schwaches Kriterium}
The following statements are equivalent:
\begin{itemize}
\item[(a)] $\MA$ is torsion free
\item[(b)] $\KA=\comp{\CA}$
\item[(c)] $\theta$ is injective.
\end{itemize}
\end{cor}


\begin{rem} \label{Ext}
The map $\theta$ is
surjective if and only if the  extension group
${\rm Ext}_{\gr}^1(\Ed/\CA, \gr)$ is trivial, in particular if
$\CA$ is a direct summand in $\Ed$.
\end{rem}

At the beginning of this section we have constructed an algebra
structure on the dual module of a coalgebra. Conversely we should obtain
a coalgebra structure on the dual of an algebra $A$. Whereas this is not
possible in general, it can be done under certain restrictions to
the $\gr$-module structure of the algebra $A$. To be more precise,
the natural $\gr$-homomorphism from $\dual A\otimes \dual A$ into
$\dual{(A\otimes A)}$ must be an isomorphism. This is the case
if $A$ is projective and finitely generated. Therefore, under these
circumstances the construction is always possible and functorial,
that is: duals of algebra maps become coalgebra maps. Thus we obtain


\begin{thm}[Second Comparison Theorem] \label{duale Koalgebra}
Suppose the centralizer algebra $\CA$  of $A$ is a direct summand in
$\Ed$ as an $\gr$-module. Then
$\dual{\CA}$ can be turned into a coalgebra. The map 
$\theta:\MA\pfr \dual{\CA}$  is an epimorphism of coalgebras, whose
kernel is just the torsion submodule of $\MA$.
\end{thm}












\section{Change of Base Rings}

Let $\er$ be another noetherian integral domain together with a
homomorphism $\gr \pfr \er$ of rings. We will consider $\er$
as an $\gr$-algebra via this homomorphism. In this situation there
is a functor from the category of $\gr$-modules into the category of 
$\er$-modules given by

\begin{equation} \label{S-Funktor}
\merw{W}{\er}:=\er \otimes_{\gr} W \;\;\mbox{ and } \;\;
 \merw{\alpha}{\er}:=\id{\er} \otimes \alpha:\merw{U}{\er} \pfr
\merw{W}{\er}
\end{equation}

to each pair of $\gr$-modules $W$ and $U$ and an  $\gr$-homomorphism
$\alpha:U \pfr W$. We will study the behaviour of the construction of
centralizer coalgebras under these functors. It will turn out that the
centralizer coalgebra behaves better than the centralizer algebra.
To start with let

\[\szerw{\Ed} := \End{\er}{\smerw{\nt}},\;\;
\szerw{\zeta}:\smerw{\Ed}\pfr\szerw{\Ed}\]

where $\szerw{\zeta}$ is given by
$\szerw{\zeta}(s \otimes e)(t \otimes v):=
st \otimes e(v)$ for all $s,t \in \er, e \in \End{\gr}{\nt}, v \in
\nt$. Let $\geinb A:A\hookrightarrow \Ed$ be the natural embedding 
of the subalgebra $A\subseteq \Ed$ and let

\[ \szerw A:=\Ima{\szerw{\zeta}\circ\seinb A}\]

denote the image of $\smerw A$ in $\szerw{\Ed}$. Further there are
natural homomorphisms

\[\szerw{\eta}: \smerw{\CA} \pfr \ZA{\szerw A}=\End{\szerw
A}{\smerw{\nt}}\subseteq \szerw{\Ed}\]

on generators given in a similar way to $\szerw{\zeta}$. Note, that
both, $\szerw{\zeta}$ and $\szerw{\eta}$ are homomorphisms of algebras
connected by the equation

\[ \geinb{\ZA{\szerw A}}\circ \szerw{\eta}=\szerw{\zeta} \circ
\seinb{\CA}. \]

Thus $\szerw{\eta}$ is injective
if and only if $\seinb{\CA}$ is injective ($\einb$ stands for the 
corresponding embeddings).  This may fail if $\ZA A$
is not a pure $\gr$-submodule in $\Ed$. Further $\szerw{\eta}$ may fail to
be surjective (see the example following theorem \ref{starkes Kriterium}).
$\ZA A$ is called {\em stable under base change} if $\szerw{\eta}$ is an
isomorphism of for all choices for $\er$. Now, in 
analogy to $\szerw{\eta}$ we are going to consider natural homomorphisms

%\[ \szerw{\rho}:\smerw{\KK A} \pfr \KK{\szerw A}\;\mbox{ and }\;
\[ \szerw\mu : \smerw{\ZK A} \pfr \ZK{\szerw
  A}=\dual{(\szerw{\Ed})}/\KK{\szerw{A}}\]

We will show that they are isomorphisms independent of the choices for $A,\; \gr$ and $\er$.
To this claim we consider

\[\szerw{\chi}:={\dual{\szerw{\zeta}}}^{-1}\circ\duerw{\Ed}:
\smerw{\dEd}\pfr \dual{(\szerw{\Ed})}
\]

where  $\duerw{\Ed}:\smerw{\dEd}\pfr \dual{\smerw{\Ed}}$ is the natural
homomorphism given by
$\duerw{\Ed}(s \otimes f)(t \otimes e):=stf(e)$
on genorators with $s,t \in \er, \; f \in \dual{\Ed}
\; e \in W$. It is easy to check 
that $\szerw{\chi}$ is a homomorphism of coalgebras
if the coalgebra structure on $\smerw{\dEd}$ is defined in a canonical
way (for details see \cite{doc} section 1.5). Further, the verification
of the commutativity rule

\begin{equation} \label{komm. isod}
\szerw{\chi}\circ \smerw{\isod}=\szerw{\isod}\circ \szerw{\zeta}
\end{equation}

is straightforward, as well. Here $\szerw{\isod}:\szerw{\Ed}\pfr
\dual{(\szerw{\Ed})}$ is the isomorphism induced by the
matrix trace map $\szerw{\tr}:\szerw{\Ed}\pfr \er$. Setting

\[\KL{\szerw A}:=<[\nu,\mu ]| \nu \in \szerw A, \mu \in \szerw{\Ed}
>_{\er
-\mbox{mod}},\]

we obtain

\[ \Ima{\szerw{\zeta}\circ \seinb{\KL A}}=\KL{\szerw A},\]

since $\szerw{\zeta}$ is an isomorphism of $\er$-algebras. By
definition we have $\szerw{\KA}=\szerw{\isod}(\szerw{\KL A})$
and $\Ima{\seinb{\KA}}=\Ima{\smerw{\isod}\circ \seinb{\KL  A}}$.
Using (\ref{komm. isod}) this yields
\begin{equation} \label{Bild KK}
\Ima{\szerw{\chi}\circ \seinb{\KA}}=\KK{\szerw A}.
\end{equation}

Here again, we have used the symbol $\einb$ to indicate embeddings
of $\gr$-submodules. Note that in particular $K:= \Ima{\seinb{\KA}}$ is
a coideal in $\smerw{\dEd}$ since $\szerw{\chi}$ is an isomorphism of
coalgebras and therefore
$\smerw{\MA}\cong \smerw{\dEd}/ K$ is a coalgebra.
Finally, we are able to define the natural homomorphism $\szerw\mu$
%\[ \szerw{\rho}:=\szerw{\chi}\circ\seinb{\KA}:\smerw{\KK A} \pfr
%\KK{\szerw A}\]
%and 
% : \smerw{\ZK A} \pfr \ZK{\szerw A}=\dual{(\szerw{\Ed})}/
%\KK{\szerw A}\]
%where the latter one is defined 
as the factorization of $\szerw{\chi}$ 
which exists by (\ref{Bild KK}).
We immediately obtain

\begin{thm}[Change of Base Rings]   \label{base extension}
For any noetherian integral domain $\er$ which is an $\gr$-algebra and any
$\gr$-subalgebra $A$ of $\Ed$ there is a  natural homomorphism
\[ \szerw\mu:\er \otimes_{\gr}\MA \pfr \ZK{\er \otimes_{\gr} A} \]
which is an isomorphism of $\er$-coalgebras. 
\end{thm}

This means that $\MA$ is stable under base changes for all choices of $A$ and $\gr$.
If the $\gr$-algebra $\er$ is a field, it follows from theorems
\ref{duale Algebra} and \ref{base extension} that
\begin{equation} \label{Dim CA MA}
\Dim{\er}{\ZA{\szerw A}}=
\Dim{\er}{\dual{\ZK{\szerw A}}}=
\Dim{\er}{\dual{(\smerw{\MA})}}=\Dim{\er}{\smerw{\MA}}
\end{equation}

Now, for a noetherian integral domain $\gr$ 
it is known from commutative algebra that an $\gr$-module $W$ is
projective if and only if the dimension of $\smerw W$ is independent
of the field $\er$. Thus we obtain


\begin{cor} \label{MA lokal frei} 
Let $F$ be the field of fractions of $\gr$. Then
$\MA$ is projective if and only if
$\Dim{\er}{\ZA{\szerw A}}=\Dim{F}{\ZA{\zerw AF}}$ holds for each field
$\er$.
\end{cor}



\begin{thm}[Criterion of Projectivity] \label{starkes Kriterium}
The following statements are equivalent:
\begin{itemize}
\item[(a)] The centralizer coalgebra $\MA$ is projectiv.
\item[(b)] The centralizer algebra $\ZA A=\End{A}{\nt}$ is stable under base change.
\item[(c)] $\MA$ is torsion free and $\CA$ a direct summand in $\Ed$. 
\end{itemize}
\end{thm}

\begin{Pf}
First assume (a). Then the sequence
\[ 0\pfr \KA\pfr \dEd \pfr \MA\pfr 0 \]
is split and consequently the same is true for
\[ 0\pfr \dual{\MA}\pfr \dual{\dEd} \pfr \dual{\dEd}/\comp{\KA}\pfr 0. \]

Since $\ausw{\Ed}$ induces an isomorphism between
$\dual{\dEd}/\comp{\KA}$ and $\Ed /\CA$
according to lemma \ref{duales Komplement} it follows
that $\Ed /\CA$ is projective, as well. Thus
$\CA$ is a direct summand in $\Ed$
proving (c).\Ab

Part (a) follows from (c) by theorem \ref{duale Koalgebra}, since the
dual of a projective module is projective again.
To verify (b) we therefore may assume both (a) and (c). Since $\ZA A$
is a direct summand $\seinb{\ZA A}$ is injective for all $\gr$-algebras
$\er$. Consequentely all
$\szerw{\eta}$ are injective (see above). To show surjectivity note that the image
$\Ima{\szerw{\zeta} \circ\seinb{\CA}}=
\Ima{\geinb{\ZA{\szerw A}}\circ \szerw{\eta}}$ of $\smerw{\CA}$ in $\szerw{\Ed}$
must be a direct summand therein, since $\szerw{\zeta}$
is an isomorphism and $\Ima{\seinb{\CA}}$ a direct summand in $\smerw{\Ed}$.
Therefore, to show that this submodule of $\szerw{\Ed}$ 
coincides with $\ZA{\szerw A}$, it is enough to verify that both
have the same rank (the dimension of the $G$-tensored module
over the field $G$ of fractions on $\er$). But these ranks must indeed be
the same as can be seen from the following calculations
\[ \Dim{G}{\merw{\CA}{G}}=\Dim{F}{\merw{\CA}{F}}=\Dim{F}{\ZA{\zerw AF}}=
\Dim{G}{\ZA{\zerw AG}}.\]

where the left-hand-side equation holds by projectivity of $\CA$, the right-hand-side one
by corollary \ref{MA lokal frei} and the one in the middle
since $\zerw{\eta}{F}$
is an isomorphism by flatness of the field $F$ of fractions on $\gr$. 
This establishes (b).\Ab

Now assume (b). This implies that the map $\seinb{\ZA A}$ induced by
the embedding $\geinb{\ZA A}$ is injective for all $\er$. By
commutative algebra arguments one concludes that $\ZA A$ is a direct
summand in the $\gr$-free module $\Ed$, in particular it is
projective. Now, let $\er$ be a field.
Since $\szerw{\eta}$ is an isomorphism we have

\[ \Dim{\er}{\smerw{\CA}}= \Dim{\er}{\ZA{\szerw A}}.\]

The left-hand-side is independent of $\er$ by projectivity of $\ZA
A$. Thus by corollary \ref{MA lokal frei} $\MA$ is projective yielding
(a).
\end{Pf}

{\bf Example:} Let $\gr=\Z$ and $\nt=\Z^4$. Further let
\[a:=\left( \begin{array}{cccc}
0 & 2 & 0 & 0 \\
0 & 2 & 0 & 0 \\
0 & 0 & 0 & 1 \\
0 & 0 & 0 & 2 \\
\end{array} \right)\in \Ed=\End{\Z}{\Z^4}\]
and $A:=<a>$ be the subalgebra in $\Ed$ generated by $a$. 
Each field can be considered as a $\Z$-algebra.
For a field of characteristic different from $2$ the minimum
polynomial of $\smerw a$ is
$t^2-2t$, but in characteristic $2$ it is $t^2$. This means that
the algebra $\szerw A$ is two dimensional for each field $\er$.
It follows that $A$ is a direct summand in $\Ed$.
But nevertheless $\MA$ is not projective (free), because in the case
of a field of characteristic different from $2$
$\smerw a$ is diagonalisable and one calculates
$\Dim{\er}{\ZA{\szerw A}}=2^2+2^2=8$, while in the case of characteristic
$2$ we have $\Dim{\er}{\ZA{\szerw A}}=9$. Therefore $\MA$
can't be projective in view of corollary \ref{MA lokal frei}.





















\section{FRT-Construction}



The $r$ fold tensor product of $\nt$ is denoted by $\ntrts$. Since
it is a free $\gr$-module, as well, we may apply all results of
former sections to a situation where $\nt$ is substituted by $\ntrts$,
$\Ed$ by $\rEd:=\End{\gr}{\ntrts}$ and $\dEd$ by
$\rdEd:=\Hom{\gr}{\rEd}{\gr}$. There are
natural isomorphisms between $\rEd$ and
$\ts{\Ed}{r}$ and between $\rdEd$ and $\ts{\dEd}{r}$. We will omit
them in our notation and consider them as identity maps. Set
$\ntts 0:=\gEd 0:=\gr$.\Ab

Suppose we are given a family 
$\falg=(\rfalg)_{r \in \N_0}$ of $\gr$-algebras
such that $\rfalg \subseteq \rEd$. According to lemma \ref{Koideal}
there is an associated family of coideals $K_r$ in $\rdEd$. 

\begin{eqnarray} \label{homogene Koideale}
K_r:=\KK{\rfalg}.
\end{eqnarray}

By the above identification we consider $\tens{\dEd}:=\bigoplus_{r \in
\N_0}\rdEd$ as the tensor algebra on the $\gr$-module $\dEd$.
There is a unique coalgebra structure on this tensor algebra, extending
the one from $\dEd$ and turning $\tens{\dEd}$ into a bialgebra.
Furthermore the $\rdEd$ are subcoalgebras dual to the algebras
$\rEd$, i.e.\ the coalgebra structure on $\rdEd$ is the one considered
above.
Epimorphic images of this bialgebra are called {\em matric bialgebras}
(cf. \cite{tak2}).
Let us investigate under what circumstances the coideal
\begin{eqnarray} \label{Def. I}
I:=\oplus_{r \in \N_0 } K_r
\end{eqnarray}
becomes an ideal in $\tens{\dEd}$ and thus consequently is a
biideal. To this claim we consider inclusion maps
$\gil{r},\gir{r}:\rEd \pfr \gEd{r+1}$ given by
\[
\gil r(\mu )=\mu \otimes \id{\nt}, \; \gir r(\mu )=\id{\nt}\otimes \mu ,\; \mu 
\in \rEd \]
and focus attention to a special situation, which is general enough
for our applications.  We start with an arbitrary subset
$N\subseteq \gEd 2$ and define a family $\falg$ inductively beginning with
$\gfalg 0:=\gr,\; \gfalg 1:=\gr\cdot\id{\nt}$ and $\gfalg 2$ as the
algebraic span of $N$ in $\gEd 2$, than continuing by the formula
\[ \rfalg:=\left<\gil{r-1}(\gfalg{r-1}) +
\gir{r-1}(\gfalg{r-1})\right>_{\rm Alg} \;\;\mbox{ for } \;\;r>2.\] 
We call this a {\em by $N$ induced family} of subalgebras of $\rEd$.

\begin{prop} \label{Biideal}
Let $\falg$ be the family induced by a subset $N\subseteq
\gEd 2$. Then the coideal $I$ defined in (\ref{Def. I}) is a
homogenous biideal in $\tens{\dEd}$ generated by the coideal $K_2$.
\end{prop}

\begin{Pf}
First we show that $I$ is a homogenous ideal. To this claim take
$a \in K_r$ and $b \in\gdEd{u}$. We have to show 
$a \otimes b \in K_{r+u}$ and $b \otimes a \in K_{r+u}$. The choice of the element
$a$ can be reduced to a generator $a=\kerz\mu {\nu}$ of the $\gr$-module $K_r$
with $\mu 
\in\rEd$ and $\nu \in \rfalg$.
Letting 
$\hat{\nu}:=\gil{r+u-1} \circ \gil{r+u-2} \circ \ldots \circ \gil{r}(\nu) =
\nu \otimes \id{\ntts{u}} \in \gfalg{r+u}$ and
$\hat\mu :=\mu \otimes \bar b$, where $\bar b\in \gEd{u}$ is the preimage of $b$
under $\isod$ one obtains
\[ a \otimes b = \kerz\mu {\nu} \otimes b = \kerz{\hat\mu }{\hat{\nu}} \in
K_{r+u}.\]
Similary $b \otimes a \in K_{r+u}$ and the first statement is
established.\Ab

For the second part denote by $J$ the homogenous ideal in
$\tens{\dEd}$ generated by $K_2$. Since $I$ is a homogenous ideal
containing $K_2$ we get $J\subseteq I$. For the reverse inclusion we
show $K_r\subseteq J$ by induction on $r$. Clearly
$K_0=K_1=(0)$ and $K_2$ are contained in $J$. Now suppose $r>2$.
Letting $J_r:=J\cap\rdEd$ one obtains $J_r=J_{r-1}\otimes \dEd +\dEd\otimes
J_{r-1}$. By induction hypothesis we see $K_{r-1} = J_{r-1}$. Therefore
\[ J_r=K_{r-1}\otimes \dEd + \dEd \otimes K_{r-1}=
\KK{M} \]
where $M:=\gil{r-1}(\gfalg{r-1}) + \gir{r-1}(\gfalg{r-1}) \subseteq
\rEd$. But since $\rfalg$ is generated by $M$ as an algebra
we finally see from corollary \ref{Verknupfungstreue}
$K_r =\KK{\rfalg}=\KK M=J_r$.
\end{Pf}

According to the proposition we may assign a
{\em graded matric bialgebra} to each subset $N\subseteq \gEd 2$ by
\begin{eqnarray}  \label{FRT allgemein}
\frt{N}:=\tens{\dEd}/I
\end{eqnarray}
whose homogenous summand are the centralizer coalgebras $\ZK{\rfalg}$.
This bialgebra will be called the {\em FRT-construction} corresponding
to the subset $N$. The reader familiar with the ordinary
FRT-construction will recognize the latter one as the special case
where $N$ consists of just one element $\beta$ (use the description
(\ref{frt konkret}) below). In this case we write
$\frt{\beta}:=\frt{N}$. Usually in the applications this $\beta$ is a
{\em quantum Yang-Baxter operator} leading to a representation of
the {\em Artin braid groups} on the modules $\ntrts$. In this situaton
the algebras $\rfalg$ are just the images of the corresponding
group algebras (over $\gr$) under this representation.\Ab

An application where $N$ must consist of two elements will be given in
the next section. Here the $\rfalg$ are images of the {\em Brauer
  centralizer algebras} under Brauer's representations corresponding
to the symplectic groups. The bialgebra $\frt{N}$ will turn out to be
the coordinate ring of a certain {\em symplectic monoid}.
It is a remarkable fact, that the second operator is only needed
in the classical situation, whereas in the quantum case it lies in the
algebraic span of the quantum Yang-Baxter operator (see \cite{doc},
Bemerkung 2.5.1, 2.5.4). Thus, the ordinary FRT-construction behaves
singularly (in a certain sense) when specializing the deformation parameter
to 1, i.e.\ in the classical limit.\Ab

We close this section giving a more convenient description of $\frt
N$. We denote by
$\nmind r$ the set of maps from 
from $\mg r:=\{1, \ldots , r\}$ to
$\mg n:=\{1, \ldots , n\}$ and call the elements 
${\bf i} \in \nmind r$ multi-indices writing ${\bf i}=(i_1, \ldots ,
i_r)$, where $i_j\in \mg n$ for $j\in \mg r$. The residue classes
of the multiplicative
generators $\des ij +I$ of $\frt N$ where $i,j\in \mg n$ will be
denoted by $\xes ij$. For pairs of multi-indices $\ibf i,\ibf j\in
\nmind r$ we introduce the abbreviation

\[ \xes{\ibf i}{\ibf j}:=\xes{i_1}{j_1}\xes{i_2}{j_2}\ldots
\xes{i_r}{j_r}. \]

Using the notation introduced in (\ref{Notation End}) we obtain
a presentation of $\frt N$ by generators and relations given as
follows

\begin{equation} \label{frt konkret}
\frt N=\left< \xes ij, \; i,j\in \mg n|\;\; \mu \xes{\ibf i}{\ibf
    j}=\xes{\ibf i}{\ibf j}\mu,\; \ibf i, \ibf j \in \nmind 2,\;
  \mu\in N\right>. 
\end{equation}

The verification of this formula follows from the second statement of
proposition \ref{Biideal} together with (\ref{Rel MA}).
  













\section{Example: Symplectic Monoids}

Let $n=2m$ be even. We will apply the FRT-construction to two 
endomorphisms $\beta, \gamma \in \gEd
2=\Ed\otimes \Ed$. In order to define them we have to introduce some notation.
First consider the involution
$i':=n-i+1$ on $\mg n$, that is

\[ (1',2', \ldots , n')=(n,n-1, \ldots , 1). \]

Further set $\epsilon_i:=1$ if $i\leq m$ and $\epsilon_i:=-1$ if $i>m$
and define

\[ \beta :=\sum_{i,j\in \mg n}\es ij \otimes \es ji \]
\[\gamma :=\sum_{i,j\in \mg n} \epsilon_i \epsilon_j
\es i{j'} \otimes\es {i'}j.
\]

The first endomorphism is just the flip operator on $\nt
\otimes \nt$, whereas the second is an integer multiple of a projection
map whose kernel is just the kernel of the linear form on
$\nt \otimes \nt$ corresponding to the canonical skew bilinear form on
$\nt$ (see below) and whose image is the one dimensional span of a
skew bivector. Our object of interest will be the FRT-construction

\[ \sAn{\gr} :=\frt{\{\beta, \gamma\}}. \]


According to the preceeding section it is a graded matric bialgebra
whose homogenous summands

\[ \rsAn{\gr} r:=\ZK{\rfalg} \]

are the centralizer coalgebras of algebras $\rfalg$ which are generated by endomorphisms

\[ \beta_i := \id{\ntts{i-1}} \otimes \beta
\otimes\id{\ntts{r-i-1}},\in \rEd \]
\[ \gamma_i :=\id{\ntts{i-1}} \otimes \gamma  \otimes\id{\ntts{r-i-1}}
\in \rEd \]

for $i=1, \ldots , r-1$. Using the notation of \cite{wenzl1} the
{\em Brauer centralizer algebra} $\br{r}{x}$ ($x$ an element in $\gr$)
is generated by symbols $g_i$ and $e_i$ for $i=1, \ldots, r-1$ and
the assignment $g_i\mapsto \beta_i$ and $e_i\mapsto \gamma_i$ defines
a representation of $\br{r}{-n}$ on $\ntrts$. Thus $\rfalg$ is
just the image of $\br{r}{-n}$ under this representation, in
H.\ Wenzl's notation from \cite{wenzl1}: $\rfalg=B_r(\spn n{})$.
R.\ Brauer showed in \cite{brauer} that this is just the
centralizer algebra of the symplectic group $\spn n{\C}$
acting on $\ntrts$
if $\gr =\C$. One of our
aims is to generalize this to the case of an arbitrary algebraically
closed field $K$ instead of $\C$. \Ab

Another problem, connected with the former, is
to show that the centralizer algebra $\ZA{\rfalg}$ 
of $\rfalg$ which will turn out
to be the {\em symplectic Schur algebra} $S_0(n,r)$ defined by
S.\ Donkin in \cite{donk3} is stable under base changes. In view of
theorem \ref{starkes Kriterium} this is equivalent to the projectivity
of the coalgebra $\rsAn{\gr} r$ as an $\gr$-module. For this purpose
we are going to construct a basis for the latter one. 
The procedure follows \cite{doc}
where the more general quantum case is treated. But it will become more
transparent in the much simpler classical case. First note that there is an
epimorphism of graded bialgebras from

\[ \An{\gr}:=\gr [\xes 11, \xes 12, \ldots , \xes nn] \]

to $\sAn{\gr}$ leaving the symbols $\xes ij$ fixed (we use $\xes ij$
as symbols for residue classes of $\des ij$ in all cases of matric bialgebras).
For $\An{\gr}$ is just the 
FRT-construction $\frt{\beta}$ where the relations coming from the
endomorphism $\gamma$ are omitted. This is because 
$\xes{i_2}{j_1}\xes{i_1}{j_2}=\beta \xes{\ibf i}{\ibf j}=\xes{\ibf
  i}{\ibf j}\beta=\xes{i_1}{j_2}\xes{i_2}{j_1}$ just give the ordinary
commutativity relations. The kernel of this bialgebra epimorphism is the ideal
in $\An{\gr}$ which is generated by the polynomials 
$\gamma \xes{\ibf i}{\ibf j}=\xes{\ibf
  i}{\ibf j}\gamma$ where $\ibf i,\ibf j \in \nmind 2$. To write down
these polynomials explicitly let us fix some notation:

\[f_{ij}:=\sum_{k=1}^n \epsilon_k\xes ik\xes j{k'},\;\;
\bar f_{ij}:=\sum_{k=1}^n \epsilon_k\xes ki\xes{k'} j \in \rAn{\gr}{2}.\]

Setting $\ibf i=(i,j)$ and $\ibf j=(k,l)$ we obtain

\[ \gamma  \xes{\ibf i}{\ibf j} = \left\{\begin{array}{ll}
\epsilon_j\bar f_{kl} & i=j' \\
0 & i \neq j' 
\end{array} \right .
\;\;\;\;\mbox{ and } \;\;\;\;
\xes{\ibf i}{\ibf j}\gamma = \left\{\begin{array}{ll}
\epsilon_l f_{ij} & k=l' \\
0 & k \neq l' 
\end{array} \right .
\]

Therefore we have 

\begin{equation}\label{sAn epi}\sAn{\gr}=\An{\gr}/{\cal F}\end{equation} 

where 
${\cal F}$ is the ideal in $\An{\gr}$ generated by the set
\begin{equation}\label{ideal generator} 
F:=\{ f_{ij},\bar f_{ij}, f_{ll'}-\bar f_{kk'}|\; 1 \leq i <j\leq n,
i\neq j', \; 1 \leq l\leq k \leq m\}.
\end{equation}
If $\gr=K$ is an algebraically
closed field we can interpret this in terms of algebraic geometry,
i.e.\ we can look at the vanishing set of ${\cal F}$ in the 
monoid $\gmhg nK$ of $n\times n$-matrices. 
It is easy to see that this is just the closed submonoid 

\[ \gspm nK:= \{ A \in \gmhg nK|\;\exists \;\; d(A)\in K,\;
A^t\spT A=A\spT A^t=\spq (A)\spT \}\]

in $\gmhg nK$ called the {\em symplectic monoid} by S.\ Doty
\cite{doty} and which has been  
considered by D.J.\ Grigor'ev \cite{grig} first.
Here $\spT$ is the Gram-matrix of the canonical
skew bilinear form, that is $\spT=(\spT_{ij})_{i,j \in \mg n}$ where
$\spT_{ij}:=\epsilon_i\delta_{ij'}$. The function $\spq :\gspm nK \pfr K$ is
called the {\em coefficient of dilation}. It is neccesarily a
regular function on $\gspm nK$ and already 
well defined in $\sAn{K}$, explicitly:

\begin{equation}\label{formel d}
\spq = \epsilon_k f_{kk'}=
\epsilon_k\bar f_{kk'}
\in \rsAn{\gr} 2.
\end{equation}

Note that this is independent of $k\in \mg n$ by the relations in $\sAn K$.
Furthermore $\spq$ is a group-like element of this bialgebra
(cf. \cite{doc} 2.1.1). The set $\gln nK \cap \gspm nK$ 
of invertible elements in $\gspm nK$
is precisely the group $\gsp nK$
of {\em symplectic similitudes}. S. Doty showed in \cite{doty} that
$\gspm nK$  in fact coincides with the Zariski-closure of $\gsp nK$ in
$\gmhg nK$. We will obtain this as an easy consequence of the results
presented below.\Ab

To write down a basis for $\rsAn{\gr} r$ we need some combinatorics.
The set of partitions of $r$ is denoted by $\rprt r$. It contains 
subsets $\prt lr$ which consist of partitions having not more than $l$ parts.
We write partitions as $l$-tuples
$\lambda :=(\lambda_1, \lambda_2, \ldots , \lambda_l)$
of nonnegative integers $\lambda_i$ in descending order
$\lambda_1\geq \lambda_2 \geq \ldots \geq \lambda_l\geq 0$ such that
$\lambda_1 +\ldots +\lambda_l=r$. To each partition one
associates a {\em Young}-diagram reading row lengths out of the
components $\lambda_i$. For example

\[ \begin{tableau}{4} \cline{1-3}
\tline{ & & }{1}{3}
\tline{ & }{2}{2}
\tline{ & }{2}{2}
\tline{}{3}{1}
\end{tableau}
\]


is associated to $\lambda=(3,2,2,1) \in \prt 48$. The column lengths
of the diagram lead to another partition $\lambda'\in
\prt{\lambda_1}{r}$ called the dual of the partition $\lambda$, i.e.\
$\lambda'_i:=|\{j|\; \lambda_j \geq i\}|$.
Let $\symg r$ denote the symmetric group on $r$ symbols and
$\symg{\lambda}$ the {\em standard Young subgroup} of $\symg r$
corresponding to the partition $\lambda$. This is the subgroup fixing
the subsets
$\{1, \ldots , \lambda_1 \}, \; \{\lambda_1+1, \ldots , \lambda_1+\lambda_2\},
\ldots $ of $\mg r$. In the above example $\lambda=(3,2,2,1)$ the
standard Young subgroup of $\symg 8$ corresponding to the dual
partition $\lambda'$ fixes $\{1,2,3,4\},\;\{5,6,7\}$ and $\{8\}$.\Ab

To each partition $\lambda\in \rprt r$ and a pair of multi-indices
$\ibf i, \ibf j\in \nmind r$
one defines a {\em bideterminant} $\bidet{\lambda}{\ibf i}{\ibf j}\in
\rAn{\gr} r$ by

\[
\bidet{\lambda}{\ibf i}{\ibf j}:=
\sum_{w \in \symg{\lambda'}} \sign w\xes{\ibf i}{(\ibf jw)}=
\sum_{w \in \symg{\lambda'}} \sign w\xes{(\ibf iw)\ibf j}.
\]

where $\ibf iw:=(i_{w(1)}, i_{w(2)}, \ldots , i_{w(r)})$. These are
products of minor determinants, one factor for each column, the size
of which
correspond to the length of the column. By (\ref{sAn epi}) 
they can be interpreted as
elements of $\rsAn{\gr} r$, as well. We wish to
write down a basis of the latter $\gr$-module consisting of such
bideterminants. Since they are too large in number one needs a criterion
to single out the right ones. This can be done using
$\lambda$-tableaux. These are constructed from the diagram of $\lambda$
by inserting the components of a multi-index column by column into the
boxes. In the above example:

\[
\itab{\lambda} i:=
\begin{tableau}{4}\cline{1-3} 
\tline{$i_1$ & $i_5$ & $i_8$}{1}{3}
\tline{$i_2$ & $i_6$}{2}{2}
\tline{$i_3$ & $i_7$}{2}{2}
\tline{$i_4$}{3}{1}
\end{tableau} .
\]

We put a new order $\ll$ on the set $\mg n$, namely
$1'\ll 1 \ll 2' \ll 2 \ll \ldots \ll m' \ll m$.
A multi-index $\ibf i$ is called {\em $\lambda$-column standard} if the
entries in $\itab{\lambda} i$ are strictly increasing down columns according
to this order. It is 
called {\em $\lambda$-row standard} if the
entries in $\itab{\lambda} i$ are weakly increasing along rows and 
{\em $\lambda$-standard} if it is both at the same time. We write
$\asmind{\lambda}$ to denote the subset of $\nmind r$ consisting
of all multi-indices being $\lambda$-standard. Such a multi-index
$\ibf i\in \asmind{\lambda}$
is called {\em $\lambda$-symplectic standard} if for each
index $i\in \mg m$ the occurences of $i$ as well as $i'$ in
$\itab{\lambda} i$
is limited to the first $i$ rows. The corresponding subset of
$\asmind{\lambda}$ will be denoted by $\symind{\lambda}$.\Ab


The notion of {\em symplectic standard tableaux} traces back to 
R.C.\ King \cite{king} and it has appeard in a lot of work concerning
symplectic groups and their representation theory (for details see
\cite{donk1}).\Ab

It is well known from invariant theory (cf. \cite{martin},
section 2.5) that the collection of all bideterminants
$\bidet{\lambda}{\ibf i}{\ibf j}$ where $\lambda$ runs through $\prt
nr$ and $\ibf i, \ibf j$ run through $\asmind{\lambda}$ form a
basis of $\rAn{\gr} r$. Similarily we will prove in the next section

\begin{thm} \label{Basissatz}
The $\gr$-module  $\rsAn{\gr} r$ has a basis given by

\[ \B_r:=\{ {\spq}^l\bidet{\lambda}{\ibf i}{\ibf j}|\; 0\leq l \leq 
\frac r2, \; \lambda \in \prt m{r-2l},\;
\ibf i,\ibf j \in \symind{\lambda}\}. \]
\end{thm}

Before proving this let us have a look at some consequences. The first
one generalizes theorem 9.5 (a) of \cite{doty} avoiding the
restriction to characteristic zero. Furthermore, it contains corollary
5.5 (f) of that paper for algebraically closed fields.

\begin{cor} \label{Koordinatenring beliebig}
Let $K$ be an algebraically close field. Then $\sAn K$ coincides with
the coordinate ring of the Zariski-closure $\zgsp nK$ 
of $\gsp nK$ in $\gmhg nK$.
In particular $\gspm nK$ is identical to $\zgsp nK$. Therefore, a complete set
of generators of the vanishing ideal of $\zgsp nK$ in $\An K$ is given by the set
$F$ (defined in equation \ref{ideal generator}).
\end{cor}

\begin{Pf}
Let $A_0(n)$ be the  coordinate ring
of $\zgsp nK$ and $A_0(n,r)$ its $r$-th homogenous summand.
In \cite{donk3} the symplectic Schur algebra
$S_0(n,r)$ is defined as the dual algebra to the coalgebra $A_0(n,r)$.
The dimension of the latter one is given by Weyl's character formula
and therefore independent of the field $K$ (cf. \cite{donk3} p. 77).
On the other hand there is an epimorphism of graded bialgebras from
$\sAn K$ to $A_0(n)$ since $\zgsp nK$ is closed in $\gspm nK$ and
the latter one has been defined as the vanishing set of the ideal
${\cal F}$ by which $\sAn K$ is defined. But by our basis theorem
\ref{Basissatz} the dimension of $\rsAn Kr$ is independent of the
field $K$, as well. Thus, the proof can be finished looking at the
case $K=\C$ and using
Doty's theorem 9.5 (a) or alternately by a direct calculation of
$|\B_r|=\Dim{\C}{(A_0(n,r))}$ (see proposition \ref{|B_r|} below).
\end{Pf}


By theorem \ref{base extension} we have isomorphisms

\[ K\otimes_{\Z}\rsAn{\Z} r \cong \rsAn Kr,\;\; 
 K\otimes_{\Z}\sAn{\Z}  \cong \sAn K.\]

Since $\sAn K$ has been recognized to be the
coordinate ring of $\gspm nK=\zgsp nK$ we may interpret the spectrum of
the ring $\sAn{\Z}$ as an integral monoid scheme $\gspm n{\Z}$. Accordingly, an integral
form for the symplectic Schur algebra can be obtained as the
dual algebra  
\[ S_{\Z}^{\rm s}(n,r):=\Hom{\Z}{\rsAn{\Z} r}{\Z}\]
of its homogenous summands. 
By theorem \ref{starkes Kriterium} and
\ref{Basissatz}
this is stable under base change, that is, tensoring by a field $K$
gives the symplectic Schur algebra $S_{K}^{\rm s}(n,r)=S_0(n,r)=\Hom K{\rsAn Kr} K$
defined over that field. An
integral form for symplectic Schur algebras exists, as well, by
S.\ Donkin's work on {\em generalized Schur algebras} (see \cite{donk3}).
But his approach is quite different using the theory of Lie algebras in
particular the {\em Kostant $\Z$-form}.
In both cases the notion of symplectic Schur algebras can be extended
to more general integral domains $\gr$ instead of $\Z$ leading to
identical concepts. In our case this is 
$S^{\rm s}_{\gr}(n,r):=\Hom{\gr}{\rsAn{\gr} r}{\gr}$.
By theorem \ref{duale Algebra} we conclude

\begin{cor}
Over any noetherian integral domain $\gr$
the symplectic Schur algebra is
isomorphic to the centralizer algebra
of the Brauer algebra $\br{r}{-n}$.
\end{cor}

For a field of characteristic zero this has been proved by S.\ Doty,
too (\cite{doty} corollary 9.3. (c)). It should be remarked, that the
basis dual to $\B_r$ together with the anti-involution defined by matrix
transposition give a {\em cell datum} for the symplectic Schur algebra
in the sense of J. Graham and G. Lehrer (cf. \cite{doc}, 4.2.5).
Thus, its representation theory can be developed easily to the extent 
of the treatment of {\em cellular algebras} in \cite{graham}.










\section{Proof of theorem \ref{Basissatz}}



Let us first reduce to showing that $\B_r$ is a set of generators for
the $\gr$-module $\rsAn{\gr} r$. This can be done by the following
proposition where notations from the proof of corollary
\ref{Koordinatenring beliebig} are used.


\begin{prop} \label{|B_r|}
$|\B_r|=\Dim{\C}{A_0(n,r)}$.
\end{prop}


\begin{Pf} We use \cite{donk3} p. 74 ff.
First the reader may check that our definition of $A_0(n)$ and
$A_0(n,r)$ is identical to the one given there.
According to
\cite{donk3} and 2.2c in \cite{donk2} we have

\begin{equation} \label{Dim A_0}
\Dim {\C}{A_0(n,r)}=
 \sum_{\lambda\in \pi_0(n,r)} \Dim {\C}{Y_0(\lambda)}^2
\end{equation}

where $Y_0(\lambda):={\rm Ind}_{B_0}^{\gsp n{\C}}({\C}_{\lambda})$ is the
irreducible $\gsp n{\C}$ module induced from
the linear character ${\C}_{\lambda}$
of the Borel subgroup $B_0$ (notations taken from \cite{donk3}). 
Here $\lambda$ runs through the set $\pi_0(n,r)$ of dominant weights
corresponding to the irreducibles occuring in $\ntrts$. If $T_0$ denotes
the  maximal torus of $\gsp n{\C}$ we may consider the
weights $\lambda$ as the group-like elements in its coordinate
ring. More precisely $\lambda\in \pi_0(n,r)$ is of the form

\[ \lambda=\xes 11^{\mu_1}\xes 22^{\mu_2} \ldots \xes mm^{\mu_m} d^l
\]

as can be seen from the argumentation in \cite{donk3}. Here,
$\mu:=(\mu_1, \ldots , \mu_m)\in \prt{m}{r-2l}$ is a partition of
$r-2l$ in not more than $m$ parts and $0\leq l\leq \frac r2$ an
integer. Restricting to the symplectic group $\spn n{\C}$ we have
to set the coefficient of dilation $d$ equal to $1$. Thus the restriction of
$\lambda$ to the maximal torus of $\spn n{\C}$ is just the dominant
weight

\[ \bar{\lambda}=\xes 11^{\mu_1}\xes 22^{\mu_2} \ldots \xes mm^{\mu_m} 
\]


for the symplectic group itself. Furthermore, it is easy to show that
restricting the $\gsp n{\C}$-module structure of $Y_0(\lambda)$ to  
the symplectic group gives
the module $\bar Y(\bar{\lambda}):=
{\rm Ind}_{\bar B}^{\spn n{\C}}({\C}_{\bar{\lambda}})$ induced from the linear
character ${\C}_{\bar{\lambda}}$ of the Borel subgroup 
$\bar B=B_0\cap \spn n{\C}$ of $\spn n{\C}$ (for details see
\cite{doc}, 3.3.3). But the dimension of the latter one is known to
be the cardinality of $\symind{\mu}$ (see \cite{donk1}
theorem 2.3 b for instance). Thus, we obtain


\[ 
\Dim{\C}{A_0(n,r)}=
\sum_{0\leq l\leq \frac r2}\sum_{\mu \in\prt m{r-2l}}
|\symind{\mu} |^2=|\B_r|.
\]

\end{Pf}

Observe that by theorem \ref{base extension} the proof of \ref{Basissatz}
can be reduced to the
case $\gr=\Z$, since the definition of bideterminants over
$\gr$ and $\Z$ respectively commutes with the isomorphism $\zerw{\mu}{\gr}$
when $\gr$ is considered as a $\Z$-algebra. 
Now, suppose we have shown that $\B_r$ generates $\rsAn{\Z}
r\subseteq \rsAn{\C} r$ as a $\Z$-module. 
Then the image of $\B_r$ in $A_0(n,r)$ under the epimorphism
considered in the proof of \ref{Koordinatenring beliebig} is a set of
generators, too. By the above proposition it must be a basis of
$A_0(n,r)$. Consequentely, there can't be any relations among the
elements of $\B_r$, especially none with integer coefficients, giving
the desired result. \Ab

The proof that $\B_r$ is indeed a set of generators will follow from a
symplectic version of the famous {\em straightening formula}. For
convenience of the reader we will first state the algorithm leading to the
classical straightening formula. To do so, we put an order on the set
$\rprt r$ of partitions of $r$ writing $\lambda < \mu$ if the dual $\lambda'$
occurs before the dual $\mu'$ in the lexicographical order.
In this order the fundamental weight $\omega_r:=(1,1, \ldots , 1)\in
\prt rr$ is the largest element, whereas $\alpha_r:=(r)\in \prt 1r$ is the
smallest one. We abbreviate $A:=\rAn{\gr} r$ and define $A(>\lambda)$
resp.\ $A(\geq \lambda)$ to be the $\gr$-linear span in $A$ of all
bideterminants $\bidet{\mu}{\ibf i}{\ibf j}$ such that $\mu > \lambda$
resp.\ $\mu \geq \lambda$. For $\lambda=\omega_r$ we set
$A(>\omega_r):=0$. Clearly $A=A(\geq \alpha_r)$. 

\begin{prop}[Classical Straightening Algorithm] 
\label{Straightening Algorithmus 1}
Let $\lambda \in \rprt r$ be a partition of $r$
and $\ibf j \in \nmind r \ohne \asmind{\lambda}$. Then to each 
$\ibf k\in \nmind r$ satisfying $\ibf k=\ibf jw$ 
\footnote{this is missing in the statement of 2.5.7 in \cite{martin}
  but can be seen directly from the proof given there}
for some $w \in \symg
r$ and
$\ibf k\ll \ibf j$ there is an element
$a_{\ibf j\ibf k} \in \gr$ such that
in $A$ the following congruence  relation holds
 for all $\ibf i\in \nmind r$:


\[
\bidet{\lambda}{\ibf i}{\ibf j} 
\equiv \sum_{\ibf k\ll \ibf j} a_{\ibf j\ibf k}
\bidet{\lambda}{\ibf i}{\ibf k} \; \; \mbox{ mod } \; \; A(>\lambda) .
\]
\end{prop}

Here, the order on $\nmind r$ is the lexicographical one according to the
given order on $\mg n$, in our case $\ll$. A proof of the proposition can be found for
example in \cite{martin}, 2.5.7.\\

Let us now state the symplectic analogue. First consider the
algebra

\[ \shAn{\gr}:=\sAn{\gr}/\left<d \right> \]

where $\left< d\right>$ is the ideal in $\sAn{\gr}$ generated by the
coefficient of dilation. It is graded since $d$ is a homogenous element
but not a bialgebra because an augmentation map is missing. In fact, it turns
out that (in the case where $\gr = K$ is an algebraically closed field)
it is the coordinate ring of the semigroup
$\mmsp nK:=\gspm nK\ohne \gsp nK$ of 
noninvertible elements in the symplectic monoid (see remark \ref{de
  concini}). 
Let us abbreviate its submodule of homogenous elements of
degree $r$ by $A':=\rshAn{\gr} r$ and define $A'(>\lambda)$ and
$A'(\geq \lambda)$ in the same manner as above. Further, 
define a map $f:\nmind r\pfr \N_0^m$ by $f(\ibf i)=(a_1, \ldots ,
a_m)$, where 

\[ a_l:=|\{ j\in \mg r|\; i_j=l \;\mbox{ or } \; i_j=l' \}|, \]

and order $\N_0^m$ writing $(a_1,\ldots , a_m)<(b_1, \ldots , b_m)$ 
if and only if $(b_m, b_{m-1}, \ldots , b_1)$ appears before $(a_m,
a_{m-1}, \ldots , a_1)$ in the lexicographical order.  Next, we
obtain an order $\lhd$ on $\N_0^m\times \nmind r$ in a lexicographical way, as
well:

 \[(a, \ibf i)\lhd (b, \ibf j) :\gdw a < b \mbox{ or }
(a = b \mbox{ and } \ibf i \ll \ibf j).\]

Finally, this gives a new order $\lhd$ on $\nmind r$ via the embedding
$\nmind r\hookrightarrow \N_0^m\times \nmind r$ given by $\ibf
i\mapsto (f(\ibf i), \ibf i)$. Now we are able to state the symplectic
straightening algorithm:


\begin{prop}[Symplectic Straightening Algorithm] 
\label{Straightening Algorithmus 2}
Let $\lambda \in \rprt r$ be a partition of $r$
and $\ibf j \in \nmind r \ohne \symind{\lambda}$. Then to each 
$\ibf k\in \nmind r$ satisfying $\ibf k\lhd \ibf j$ there is an element
$a_{\ibf j\ibf k} \in \gr$ such that
in $A'$ the following congruence relation holds
 for all $\ibf i\in \nmind r$:


\[
\bidet{\lambda}{\ibf i}{\ibf j} 
\equiv \sum_{\ibf k\lhd \ibf j} a_{\ibf j\ibf k}
\bidet{\lambda}{\ibf i}{\ibf k} \; \; \mbox{ mod } \; \; A'(>\lambda) .
\]
\end{prop}


Before proving this, let us deduce that $\B_r$ is a set of
generators for $\rsAn{\gr} r$. First note that multiplication by 
the coefficient of dilation $d$ leads to an exact sequence for $r>1$

\[ \rsAn{\gr}{r-2}\stackrel{\cdot d}{\pfr} \rsAn{\gr} r \pfr \rshAn{\gr} r \pfr 0. \]

Therefore, by induction on $r$ we can reduce to showing that

\[ \{ \bidet{\lambda}{\ibf i}{\ibf j}|\; \lambda \in \prt m{r},\;
\ibf i,\ibf j \in \symind{\lambda}\} \]

is a set of generators for $A'=\rshAn{\gr} r$. For this claim it is
enough to show that

\[ \{ \bidet{\lambda}{\ibf i}{\ibf j}|\;\;
\ibf i,\ibf j \in \symind{\lambda}\} \]

is a set of generators of $A'(\geq \lambda)/A'(>\lambda)$ for each partition
$\lambda$. This can be deduced from the straightening algorithm
 \ref{Straightening Algorithmus 2}: Since $\nmind r$
is a finite set, the elemination of multi-indices $\ibf j$  not being
 $\lambda$-symplectic standard  in an expression
$\bidet{\lambda}{\ibf i}{\ibf j}$
must terminate. This gives the straightening formula
concerning the right-hand-side argument of $\bidet{\lambda}{\ibf
  i}{\ibf j}$:

\begin{cor}[Symplectic Straightening Formula] \label{straightening formula}
Let $\lambda \in \rprt r$ be a partition of $r$ and $\ibf j \in \nmind r$.
Then, to each $\ibf k\in \symind{\lambda}$ there is an element
$a_{\ibf j\ibf k} \in \gr$,
such that in $A'$ we have for all $\ibf i\in \nmind r$:


\[
\bidet{\lambda}{\ibf i}{\ibf j} \equiv \sum_{\ibf k \in \symind{\lambda}}
a_{\ibf j\ibf k}
\bidet{\lambda}{\ibf i}{\ibf k} \; \; \mbox{ mod } A'(>\lambda) .
\]
\end{cor}



Now, there is an algebra automorphism $\theta$ on $\An{\gr}$ induced
by matrix transposition and given by $\theta (\xes ij)=\xes ji$ on
generators. Of course it is an anti-automorphism of coalgebras. 
It can be readily seen that $\theta({\cal F})={\cal F}$ and $\theta
(\spq)=\spq$. Therefore, it factors to an automorphism of $\shAn{\gr}$
which will be denoted by the same symbol. From the definition of
bideterminants we see $\theta(\bidet{\lambda}{\ibf i}{\ibf
  j})=\bidet{\lambda}{\ibf j}{\ibf i}$.\Ab

Applying $\theta$ to the congruence relation in
\ref{straightening formula} we see that
a non $\lambda$-symplectic standard entry $\ibf i$ on the left-hand-side
entry of a bideterminant can be eleminated, too, not
affecting the right-hand-side entry. Thus, it must be possible to
write $\bidet{\lambda}{\ibf i}{\ibf j}$ as a sum of bideterminants
$\bidet{\lambda}{\ibf k}{\ibf l}$ modulo $A'(>\lambda)$ where
$\ibf k, \ibf l \in \symind{\lambda}$. Therefore, the proof of
theorem \ref{Basissatz} is finished as soon as proposition
\ref{Straightening Algorithmus 2} is established.\Ab

\begin{rem} \label{de concini}
A symplectic straightening formula similar to \ref{straightening
formula} can be obtained as a special case of a theorem by de Concini
(\cite{concini}, theorem 2.4)
where $m=2r$ (in the notation taken from there). 
The algebra denoted $A$ in that paper
becomes the coordinate ring of the semigroup denoted $\mmsp nK$ above.
But, note that this result is not strong enough for our purpose
since we don't know wether $A=\shAn K$ or not. On the other hand, the latter
identity follows from theorem \ref{Basissatz} in a similar way to corollary
\ref{Koordinatenring beliebig} using theorem 3.6 of \cite{concini}.
\end{rem}

Also, the symplectic straightening formula is related to the treatment
of symplectic Schur-modules in \cite{donk1} and \cite{iano}, section 6,
as can be seen from the proof of lemma \ref{Rel nicht symplektisch} below.
Concerning the latter paper it should be noted 
that the algebra $A^{sp}_{K}(\overline n)$ defined there
as the coordinate ring of the symplectic group itself is only filtered by
$\sum_{t=0}^r A^{sp}_{K}(\overline n,t)$ but not graded by 
$A^{sp}_{K}(\overline n,r)$. In fact, it can be deduced from the above
remark that $\shAn{K}$ is the corresponding graded algebra, 
that is $A^{sp}_{K}(\overline n,r)=\rshAn{K} r$
as $K$-vector spaces.











\section{Proof of proposition \ref{Straightening Algorithmus 2}}





First, some considerations about the exterior algebra 

\[\aA:= 
\tens{\nt}/\left<\bs i\otimes \bs j+\bs j\otimes \bs i, \bs
  k\otimes \bs k;\; i,j, k\in \mg n \right>
 \] 

are needed. It is easy to see and well known
that this graded $\gr$-algebra is a
{\em comodule algebra} for the bialgebra $\An{\gr}$, i.e.\
it is an $\An{\gr}$-comodule such that multiplication is a
morphism of comodules. In fact, the homogenous summands $\raA
r$ of $\aA$ are comodules for the coalgebras $\rAn{\gr}
r$. Since $\sAn{\gr}$ is an epimorphic image of $\An{\gr}$ the exterior
algebra is a comodule algebra for the latter one, as well.\Ab

As usual we write $\bs i\wedge \bs j$ for the residue class of $\bs
i\otimes \bs j$ in $\aA$ and denote arbitrary multiplications by
$\wedge$, too. For a subset $I:=\{i_1, \ldots , i_r\}\subseteq \mg n$
ordered $i_1 \ll i_2 \ll \ldots \ll i_r$ by the given order
$\ll$ on $\mg n$ (called an ordered subset in the sequel)
we use the abbreviation $\bs
I:=\bs{i_1}\wedge\bs{i_2}\wedge\ldots\wedge
\bs{i_r}$. The elements $\bs I$ give a basis of $\aA$ if $I$ ranges over all
subsets of $\mg n$. A basis for $\raA r$ is obtained when $I$ ranges over all
subsets of cardinality $r$, the collection of which will be denoted by
$\tmg nr$. The comodule structure of $\raA r$ 
can be described in a
simple way using bideterminants for the partition
$\omega_r:=(1^r)=(1,1,\ldots ,1)$. These are just the usual $r\times
r$-minor determinants.
Denoting the structure map by
$\cmd{\wedge}:\aA\pfr \aA\otimes \An{\gr}$ (we will use the same symbol in
the case $\sAn{\gr}$ later on) we explicitly have

\begin{equation}\label{aA comodul}
 \cmd{\wedge}(\bs J)=\sum_{I \in \tmg nr}\bs I\otimes\bidet{\omega_r}{\ibf
i}{\ibf j}
\end{equation}

where $\ibf i=(i_1, \ldots , i_r)$ and $\ibf j=(j_1, \ldots , j_r)$ 
are the multi-indices corresponding to the
ordered subsets $I:=\{i_1, \ldots , i_r\}$ and $J=\{j_1, \ldots , j_r\}$, 
respectively.
Define $d_k:=\bs{k'} \wedge \bs{k}$ and
$d_K:=d_{k_1}\wedge d_{k_2} \wedge \ldots \wedge d_{k_a}$ for a
subset $K:=\{k_1, \ldots, k_a\}\subseteq\mg m$ of cardinality $a$.
Again, we write $\tmg ma$ for the collection of all such subsets $K$.
Note that the $d_k$ are in the center of the exterior algebra and in particular
commute with each other. Thus $d_K$ is defined
independent of the order of the elements of $K$. Set

\[ D_a:=\sum_{K\in \tmg ma} d_K \]

and let $N$ be the ideal in $\aA$ generated by the elements
$D_1, D_2, \ldots , D_m$. One crucial point in the proof of
proposition \ref{Straightening Algorithmus 2} is to show that
the elements $D_a$ are invariant under the bialgebra $\sAn{\gr}$.
%the corresponding group-like elements being the $a$-th powers $d^a$
%of the coefficient of dilation.
But first, let us establish a prototype straightening algorithm inside
the graded algebra $\saA$ with respect to the set 
$\symind{\omega_r}$. We call an ordered subset $I\in \tmg nr$ {\em
  symplectic} if the corresponding multi-index $\ibf i$ is
$\omega_r$-symplectic standard.

\begin{lem} \label{Rel nicht symplektisch}
Let $I\in \tmg nr$ be non symplectic. Then, to each $J\in \tmg nr$
such that the inequallity $f(\ibf j)<f(\ibf i)$ holds
for corresponding multi-indices $\ibf i$ and $\ibf j$ there is 
$a_{IJ}\in \gr$, such that in $\aA$ the
following congruence relation holds:

\[ \bs I\equiv
\sum_{J\in \tmg nr, \; f(\ibf j)<f(\ibf i)}a_{IJ}\bs J  \;\; \mbox{
  mod }
\;\; N \]

\end{lem}
 

\begin{Pf} We follow \cite{donk1}.
Clearly, we may reduce to the case $\gr =\Z$ since 
$\naA{\gr} n\cong \gr \otimes_{\Z} \naA{\Z} n$ and the canonical isomorphism
therein respects the basis elements $\bs I$ and the ideal $N$.
If $K\subseteq \mg m$
then in $\aA$ we calculate as in \cite{donk1}, 2.2.

\begin{equation}
\label{power a}
{(\sum_{k \in K}d_k)}^a=a!\sum_{L\in \tmg ma,\; L\subseteq K} d_L
\end{equation}

In the case $K=\mg m$ this implies $D_1^a=a!D_a$.
Set $x:=\sum_{k \in K}d_k$ and $y:=D_1 -x$.
It follows from (\ref{power a}) that the equation 
\[ x^a=(-1)^ay^a+\sum_{b=1}^a { a \choose b} (-y)^{a-b}b!D_b \]
is divisible by $a!$.
Setting $X^K_a:=\sum_{L\in \tmg ma,\; L\subseteq K}d_L$ and $M:=\mg m\ohne K$
this yields
\[ X^K_a=(-1)^aX^M_a+\sum_{b=1}^a  (-1)^{a-b}X^M_{a-b}D_b\]
since $\naA{\Z} n$ is a free $\Z$-module.
We obtain $X^K_a\equiv (-1)^aX^M_a$ modulo the ideal $N$ and in the
special case $K \in \tmg ma$

\begin{equation}
\label{formel d_J}
d_K\equiv (-1)^a\sum_{L\in \tmg ma,\; L\cap K=\emptyset}d_L \;\;\mbox{
  mod } N.
\end{equation}

The proof can be finished now in a similar way to the proof of the
{\em Symplectic Carter-Lusztig Lemma} in \cite{donk1}:
Since $I$ is not symplectic there is a number $s\in \mg r$
such that $I=\{i_1,\ldots , i_r\}$ contains an element $i_s$ satisfying
$i_s<s$ if $i_s \leq m$ or $i_s'<s$ if $i_s>m$. We assume $s$ to be
as small as possible with this property.
Let $k, l\in \mg m$
be the unique numbers with $\{k, k'\}=\{i_s, i_s'\}$ and
$\{l, l'\}=\{i_{s-1}, i_{s-1}'\}$. By minimality of $s$ we have $l
\geq s-1$. On the other hand, $i_{s-1}\ll i_s$ implies $l\leq k$.
This gives $s-1\leq l \leq k < s$, that is $k=l$ and $s=k+1$.\Ab

Now let $I_s:=\{i_1,\ldots , i_s\}$ be the ordered subset of the first
$s$ entries of $I$ and
$K$ be the set of all $p\in \mg m$ such that both $p$ and $p'$
are contained in $I_s$. Setting $K':=\{p'|\; p\in K\}$ and 
$H:=I\ohne (K\cup K')$
we obtain $\bs I=\bs Hd_K$ since the elements $d_p$ are in the
center of $\naA{\Z} n$. Therefore, by equation (\ref{formel d_J})
it remains to show that for all subsets $L\in \tmg ma$ which
do not intersect $K$ we have $f(\ibf j)<f(\ibf i)$ where
$\ibf j$ is the multi-index corresponding to the ordered set $J:=H\cup L \cup
L'$. Since $\bs J=\bs Hd_L=0$ if $L$ or $L':=\{p'|\; p \in L\}$ contains an
element of $H$ we further may assume $H\cap L=H\cap L'=\emptyset$.
Now, suppose $L\subseteq \mg k$. Let $G$ be the set of all $g \in \mg k\ohne K$
such that $g$ or $g'$ lies in $I_s$. Since $I_s\ohne (K\cup
K')\subseteq H$ the intersection of $L$ and $G$ is empty, as well.
This gives a contradiction

\[ k+1=s=|I_s|=2|K|+|G|=|K|+|L|+|G|=|K\cup L\cup G|\leq k \]

for $K, L$ and $G$ are disjoint subsets of $\mg k$ by assumption on
$L$. Thus, the largest element $t$ of
$L$ must be greater than $k$, whereas all
elements of $K$ are smaller or equal to $k$. If $f(\ibf i)=(b_1,
\ldots , b_m)$ and $f(\ibf j)=(a_1, \ldots, a_m)$ 
it follows that $a_t=b_t+2>b_t$ and
$a_l=b_l$ for $t<l\leq m$. By definition of our order on $\N_0^m$ this
means $f(\ibf j)<f(\ibf i)$ completing the proof.
\end{Pf}


As mentioned before, the crucial point in the proof of proposition
\ref{Straightening Algorithmus 2} is contained in the following

\begin{prop} \label{D_a invariant}
The elements $D_a$ are invariant under the bialgebra $\sAn{\gr}$. The
corresponding group-like elements in $\sAn{\gr}$ are the $a$-th powers of the
coefficient of dilation, more precisely:

\[ \cmd{\wedge}(D_a)=D_a\otimes d^a \;\;\;\;\in \;\;
\nraA{\gr} n{2a} \otimes \rsAn{\gr}{2a}
\]
\end{prop}

\begin{Pf}
In the case $a=1$ this easily follows from (\ref{formel d}) using the
relations given by the set $F$ defined in (\ref{ideal generator}). If we could divide by
$a!$, we would be able to finish the proof right now using
$D_1^a=a!D_a$ and the fact that multiplication is a morphism of
$\sAn{\gr}$-comodules. But, as this is not possible in general (note,
that we don't know if $\sAn{\Z}$ is a free $\Z$-module, yet)
we have to proceed in another way. We set $r=2a$ and

\[ G_{\ibf i}:=\sum_{L\in \tmg ma}\bidet{\omega_r}{\ibf i}{\ibf j(L)}
\]

where $\ibf j(L)$ is the multi-index corresponding to the ordered
subset $L\cup L'=\{l,l'|\; l\in L\}$ of $\mg n$. If $\ibf i$ is the
multi-index corresponding to the ordered subset $I\in \tmg nr$ we also
write $G_I=G_{\ibf i}$. By (\ref{aA comodul})
we have
\[ \cmd{\wedge}(D_a)=\sum_{I\in \tmg nr}\bs I\otimes G_I. \]
Therefore, it remains to show that $G_I=d^a$ if there is a $K\in \tmg ma$
such that $I=K\cup K'$ and $G_I=0$ otherwise.\Ab

If $\ibf i=(i_1,\ldots , i_r)$ is a
multi-index and $l\leq a$ we set $\ibf i^l:=(i_{2l-1},i_{2l})\in
\nmind 2$. Since the result is clear in the case $a=1$ we know

\[
G_{\ibf i^l}=\left\{\begin{array}{ll}
0 & i_{2l-1}'\neq i_{2l} \\
d &  i_{2l-1}'= i_{2l}\leq m\\
-d &  i_{2l-1}'= i_{2l}> m
\end{array} \right. .
\]

Therefore, the element $G_{\ibf i}':=\prod_{l=1}^a G_{\ibf i^l}$ is zero
or $\pm d^a$ depending on $\ibf i$. Let us investigate this in more
detail. If $\sigma\in \symg r$ denotes the involution defined by
$\sigma (2l-1)=2l, \sigma(2l)=2l-1$ for $l\in \mg a$ and $W$ the
centralizer of $\sigma$ in $\symg r$, then clearly $G_{\ibf i}'=0$
if and only if $G_{\ibf iw}'=0$ for all $w \in W$. The group $W$
is isomorphic to the {\em Weyl group} of type $C_a$. It is
a semi-direct product of the normal subgroup $W^1$ consisting of all
$\pi \in W$ which permute neighboured pairs together 
with $W^0$, the
subgroup generated by the transpositions $(2l-1,2l)$ for $l\in \mg a$.
$W^1$ is isomorphic to $\symg a$, whereas the group $W^0$ can be
identified with $(\Z/2\Z)^a$. Choose a set $H$ of left coset
representatives for $W$ in $\symg r$ the element representing $W$
itself being $\id{\mg r}$.\\

%In each left coset $\pi W$ in $\symg r$ there is a unique representative
%$h$ satisfying $h(2l-1)<h(2l)$ for $l \in \mg a$
%and $h(1)< h(3) < \ldots < h(2a-1)$. The first condition can be achieved
%by operating from the right by a transposition $(2l-1, 2l)\in W^0$ for each
%$l$ with $\pi(2l-1)>\pi(2l)$. To achiev the second condition one has
%to operate with elements of $W^1$. Uniqueness follows
%since $h$ is already known by the subset
%$\{min(\pi(2l-1),\pi(2l))|\; l \in \mg a\}$ of $\mg n$ which just depends
%on $\pi W$. We denote the set of all such coset representatives by $H$.\Ab

Now, if $\ibf i$ corresponds to an ordered set $I=K\cup K'$ for some $K \in
\tmg an$ the inequality
$G_{\ibf i\pi}'\neq 0$ holds for a permutation $\pi\in \symg r$ if and only if
$\pi \in W$. Thus, for $h\in H$ we have $G_{\ibf ih}'=0$ if $h\neq
\id{}$ and $G_{\ibf i}'=d^a$. If there is no $K$ such that $I=K\cup K'$ one
clearly has $G_{\ibf ih}'=0$ for all $h\in H$.
Therefore, the proof is finished
as soon as we have shown
\begin{equation} \label{Formel G_I}
G_{\ibf i}=\sum_{h \in H}\sign h G_{\ibf ih}'.
\end{equation}
To this claim
let $\mu :=(a,a) \in \prt{2}{r}$ be the partition of $r$ whose diagram
consists of two rows of length $a$. To a multi-index $\ibf l=(l_1,l_2,
\ldots , l_a)\in \mmind a$ another multi-index $\ibf j(\ibf l):=(l_1', l_1,
l_2', l_2, \ldots , l_a', l_a)\in \nmind r$ can be associated.
Using this notation and reading $\bidet\mu {\ibf{i}}{\ibf j(\ibf l)}$ as a
product of $a$ $2\times 2$-determinants we obtain the formula

\[ G_{\ibf i}'=
\sum_{\ibf l \in \mmind a }
\bidet\mu {\ibf{i}}{\ibf j(\ibf l)}\;\;\;
 \in \rsAn{\gr} r. \]


On the other hand, using {\em Laplace Duality} (see for example
\cite{martin} 2.5.1) we calculate

\begin{equation}\label{laplace klassisch}
\bidet{\omega_r}{\ibf i}{\ibf j}=\sum_{h\in H,\;\pi \in W^1}\sign h
\bidet\mu {\ibf ih\pi}{\ibf j}=
\sum_{h\in H}\sign h\sum_{\pi \in W^1}
\bidet\mu {\ibf ih}{\ibf j\pi}.
\end{equation}


Therein, note that $W^0$ is precisely the column stabilizer of the
basic tableaux $\itab{\mu} b$, whereas the column stabilizer
of $\itab{\omega_r} b$ is all of $\symg r$ (here $\ibf
b:=(1',1,2',2,\ldots ,a',a)$). Also, note that $HW^1$ is a set of left
coset representatives of $W^0$ and that all permutations of $W^1$ are even.
Furthermore, in the right-hand-side equation we have used the
commutativity between the $2\times 2$-determinant factors of 
$\bidet\mu {\ibf ih\pi}{\ibf j}=\bidet\mu {\ibf ih}{\ibf
  j\pi^{-1}}$. Now, the proof of
(\ref{Formel G_I}) can be reduced to the verification of

\begin{equation} \label{Formel mu}
\sum_{L \in \tmg mr}\sum_{\pi \in W^1}\sum_{h \in H} \sign h 
\bidet\mu {\ibf ih}{\ibf j(L)\pi}=
\sum_{\ibf l \in \mmind a}
\sum_{h \in H}\sign h\bidet\mu {\ibf{i}h}{\ibf j(\ibf l)}
\end{equation}

To this claim we associate to a multi-index $\ibf l\in \mmind a$ its
contents $|\ibf l|=\lambda =(\lambda_1,\ldots , \lambda_m)$
which is defined by $\lambda_i:=|\{1\leq t \leq a|\; l_t=i\}|$. It is
a composition of $a$ into $m$ parts, that is an $m$-tuple of non negative
integers $\lambda_i$ summing up to $a$. These compositions count the
set of $\symg a$-orbits in $\mmind a$.
Denoting the set of all such
compositions by $\komp ma$ we can write down the right-hand term in
(\ref{Formel mu}) as a sum of subsums
$\sum_{\lambda \in \komp mr} \Sigma_{\lambda}$ each of which is given by

\[ \Sigma_{\lambda}:=\sum_{|\ibf l|=\lambda}
\sum_{h \in H} \sign{h}
\bidet\mu {\ibf{i}h}{\ibf j(\ibf l)}
\]

Now, the subsum $\Sigma_{\omega_a}$ (for $\omega_a=(1^a)$)
is just the left-hand-side in
(\ref{Formel mu}). Therefore, it remains to show that all other
subsums are zero. To this claim we denote the cardinality 
of the standard Young subgroup $\symg{\lambda}$ of $\symg a$
corresponding to the composition $\lambda \in \komp ma$
by $k_{\lambda}:=|\symg{\lambda}|$. If
$\ibf k=(k_1, \ldots ,k_a)\in \mmind a$ is the unique multi-index
with contents $\lambda$ and $k_1\leq
k_2\leq \ldots \leq k_r$ (the {\em initial index} corresponding to $\lambda$)
then $\symg{\lambda}$ is just the stabilizer of
$\ibf k$ in $\symg a$. Identifying $\symg a$ with $W^1$ it is the stabilizer of
$\ibf j(\ibf k)\in \nmind r$ in $W^1$.
Applying (\ref{laplace klassisch}) again we obtain

\[ \bidet{\omega_r}{\ibf i}{\ibf j(\ibf k)}=\sum_{h\in H}\sign{h}
\sum_{\pi \in W^1}
\bidet\mu {\ibf ih}{\ibf j(\ibf k)\pi}= \]
\[
k_{\lambda}\sum_{h\in H}\sign{h}
\sum_{|\ibf l|=\lambda}
\bidet\mu {\ibf ih}{\ibf j(\ibf l)}=
 k_{\lambda}\Sigma_{\lambda}.
\]

Now, $\bidet{\omega_r}{\ibf i}{\ibf j(\ibf k)}$ must be zero since 
$|\ibf k|\neq \omega_a$ implies that $\ibf k$ contains at least one
number twice. We obtain $k_{\lambda}\Sigma_{\lambda}=0$ and because
this equation is already valid in the free $\Z$-module $\rAn{\Z} r$
we conclude $\Sigma_{\lambda}=0$ for all $\lambda \neq \omega_a$
completing the proof.
\end{Pf}

Let us prove proposition \ref{Straightening Algorithmus 2} in
the case $\lambda=\omega_r$ first. Take 
$\ibf j\in \nmind r\ohne\symind{\omega_r}$. Using the classical
straighening algorithm \ref{Straightening Algorithmus 1} we may
assume $\ibf j\in \asmind{\omega_r}\ohne\symind{\omega_r}$
(observe that $\ibf k=\ibf jw$ implies $f(\ibf k)=f(\ibf j)$). This means,
that $\ibf j$ is a multi-index corresponding to a non symplectic ordered set
$J\in\tmg nr$ in the sense of lemma \ref{Rel nicht
  symplektisch}. Application of the latter one yields

\[ X:=\bs J-
\sum_{K\in \tmg nr, \; f(\ibf k)<f(\ibf j)}a_{\ibf j\ibf k}\bs K  \;\; \in
 N \]

According to proposition \ref{D_a invariant}  $\cmd{\wedge}(X)$ must be
contained in $\aA\otimes <d>$ where $<d>$ denotes the 
ideal in $\sAn{\gr}$ generated by $d$. Applying (\ref{aA comodul})
we obtain the following equation in $\raA r\otimes \rshAn{\gr} r$:

\[ \sum_{I\in \tmg nr}\bs I\otimes
\left(\bidet{\omega_r}{\ibf i}{\ibf j}-\sum_{K\in \tmg nr, \; \ibf k 
\lhd \ibf j}a_{\ibf j\ibf k}
\bidet{\omega_r}{\ibf i}{\ibf k}\right)=0.\]

Since $\{\bs I|\; I \in \tmg nr\}$ is a basis of $\raA r$ each
individual summand in the summation over $\tmg nr$ must be zero.
This gives the desired result in the case of multi-indices $\ibf i$
corresponding to ordered subsets $I\in \tmg nr$, that is 
$\ibf i\in \asmind{\omega_r}$. The general case for $\ibf i$ can be
deduced from this, easily (see \cite{doc}, 3.11.4).\Ab

Now, lets turn to the general case of $\lambda$. Again, we may assume
$\ibf j\in \asmind{\lambda}\ohne\symind{\lambda}$ by the classical
straightening algorithm. Let $\lambda'=(\mu_1, \ldots ,
\mu_p)$  be the dual partition ($p=\lambda_1$). We spilt $\ibf j$
into $p$ multi-indices $\ibf j^l\in\nmind{\mu_l}$ where for
each $l\in \mg p$ the entries of $\ibf j^l$ are taken from the $l$-th
column of $\itab{\lambda} j$. The same thing can be done with
$\ibf i$. Since $\ibf j$ is not $\lambda$-symplectic standard but standard
there must be a column $s$ such that $\ibf j^s$ is not
$\omega_{\mu_s}$-symplectic standard. Applying the result to the
known case of $\bidet{\omega_{\mu_s}}{\ibf i^s}{\ibf j^s}$ we obtain

\[ \bidet{\lambda}{\ibf i}{\ibf j}=
\bidet{\omega_{\mu_1}}{\ibf i^1}{\ibf j^1}
\bidet{\omega_{\mu_2}}{\ibf i^2}{\ibf j^2} \ldots
\bidet{\omega_{\mu_s}}{\ibf i^s}{\ibf j^s} \ldots
\bidet{\omega_{\mu_p}}{\ibf i^p}{\ibf j^p}\]

\[\equiv \sum a_{\ibf j^s\ibf k^s}
\bidet{\omega_{\mu_1}}{\ibf i^1}{\ibf j^1} \ldots
\bidet{\omega_{\mu_s}}{\ibf i^s}{\ibf k^s} \ldots
\bidet{\omega_{\mu_p}}{\ibf i^p}{\ibf j^p}\;\;
= \;\; \sum a_{\ibf j\ibf k}
 \bidet{\lambda}{\ibf i}{\ibf k}.\]

Therein, $\ibf k^s\in \nmind{\mu_s}$ satisfies $\ibf k^s\lhd\ibf j^s$,
$\ibf k\in \nmind r$ is constructed from $\ibf j$ replacing the
entries of $\ibf j^s$ by that of $\ibf k^s$ and $a_{\ibf j\ibf k}$ is
the same as $a_{\ibf j^s\ibf k^s}$ for the corresponding $\ibf k^s$.
One easily checks
$\ibf k\lhd\ibf j$ and the proof of \ref{Straightening Algorithmus 2}
is completed.







\addcontentsline{toc}{chapter}{References}
\begin{thebibliography}{abcd}
\bibitem[Br]{brauer} Brauer,\ R.: On Algebras which are connected with the
        semisimple continuous Groups. Annals of Math. Vol. 38, No. 4 (1937),
        857-871.
\bibitem[Co]{concini} De Concini C.: Symplectic Standard Tableaux. Advances
        in Mathematics 34 (1979), 1-27.
\bibitem[Do1]{donk2} Donkin,\ S.: On Schur Algebras and Related Algebras I,
        Journal of Algebra 104 (1986), 310-328.
\bibitem[Do2]{donk3} Donkin,\ S.: Good Filtrations of Rational Modules for
        Reductive Groups. Arcata Conf. on Repr. of Finite Groups. Proceedings
        of Symp. in Pure Math., Vol. 47 (1987), 69-80.
\bibitem[Do3]{donk1} Donkin,\ S.: Representations of symplectic groups and the
       symplectic tableaux of R.C. King. Linear and Multilinear Algebra, Vol.
       29  (1991), 113-124.
\bibitem[Dt]{doty} Doty,\ S.: Polynomial Representations, Algebraic Monoids,
       and Schur Algebras of Classical Type. J. of 
       Pure and Applied Algebra 123 (1998), 165-199.
\bibitem[GL]{graham} Graham,\ J.J., Lehrer,\ G.I.: Cellular Algebras. 
        Invent. Math. 123 (1996), 1-34.
\bibitem[Gg]{grig} Grigor'ev,\ D. J.: An Analogue of the Bruhat Decomposition
       for the Closure of the Cone of a Chevalley Group of the Classical 
       Series. Soviet Math. Dokl., Vol. 23 (1981), No. 2.
\bibitem[Ha]{hay4} Hayashi,\ T.: Quantum Deformation of Classical Groups. 
       Publ.\ RIMS, Kyoto Univ. 28 (1992), 57-81.
\bibitem[Ia]{iano} Iano-Fletcher,\ M.: Polynomial Representations of
       Symplectic Groups. Thesis, University of Warwick, 1990.
\bibitem[Ki]{king} King,\ R.C.: Weight multiplicity for classical groups., 
       Group Theoretical Methods in Physics (fourth International Colloquium,
       Nijmegen 1975), Lecture Notes in Physics 50, Springer 1975.
\bibitem[Ma]{manin} Manin,\ Y.I.: Quantum Groups and Non-Commutative Geometry.
       CRM, Univesit\`{e} de Montr\`{e}al, 1988.
\bibitem[Mr]{martin} Martin,\ S.: Schur Algebras and Representation Theory.
       Cambridge University Press, 1993.
\bibitem[Oe]{doc} Oehms,\ S.: Symplektische $q$-Schur-Algebren, Thesis, 
       University of Stuttgart.
\bibitem[RTF]{frt} Reshetikhin,\ N. Y., Takhtadjian,\ L. A., Faddeev,\ L. D.:
       Quantization of Lie groups and Lie algebras, Leningrad Math. J. 1
       (1990), 193-225.
\bibitem[Su]{Sudbery} Sudbery,\ A.: Matrix-Element Bialgebras Determined by
       Quadratic Coordinate Algebras. J. of Algebra, Vol 158, (1993), 375-399.
\bibitem[Ta]{tak2} Takeuchi,\ M.:  Matric Bialgebras and Quantum Groups.
       Israel J. of Math., Vol. 72, Nos. 1-2, (1990), 232-251.
\bibitem[We]{wenzl1} Wenzl,\ H.: On the structure of Brauer's centalizer
       algebras. Annals of Math., 128 (1988), 173-193.
\end{thebibliography}


\end{document}
